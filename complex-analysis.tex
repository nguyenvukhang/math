\section{Complex Analysis}\label{f28d4dc}

\Definition{0.0.0}{General terminology}\label{d508dc8}

\textcolor{white}{.}

\textbf{Entire function} is a complex-valued function that is holomorphic on
$\C$.

A \textbf{real-valued} function is any function $f:X\to\R$.

A \textbf{complex-valued} function is any function $f:X\to\C$.

A subset of $\R^n$ or $\C^n$ is called \textbf{compact} if it is closed and
bounded.

The $C^n$ notation:
\begin{itemize}
  \item $C^0$ : continuous
  \item $C^1$ : continuously differentiable
  \item $C^2$ : twice continuously differentiable
\end{itemize}

\Theorem{0.0.1}{Conventional notation}\label{d1055b9}

For this chapter on Complex Analysis.

Let $U\subseteq\C$ be an open set.

Let $D(P,r)$ be the open disc centered at $P$ with radius $r$. Then
\begin{enumerati}
  \item $\partial D(P,r)$ is the (closed) curve at the border of $D(P,r)$
  \item $\overline D(P,r)$ is the closed disc centered at $P$ with
  radius $r$.
\end{enumerati}

\Result{0.0.2}{Basic complex arithmetic}\label{adebe9e}

\begin{align*}
  |z|^2           &=z\bar z                       \\
  |zw|^2          &=|z|^2|w|^2                    \\
  |z+w|^2         &=|z|^2+|w|^2+\Re(z\cdot\bar w) \\
  |z+w|^2+|z-w|^2 &=2|z|^2+2|w|^2
\end{align*}

\begin{proof}
  Let $z:=x+iy$, and $w:=u+iv$.
  $$
    |z|^2=x^2+y^2=(x+iy)(x-iy)=z\bar z
  $$
  \begin{align*}
    |zw|^2
     &= (xu-yv)^2+(xv+yu)^2                        \\
     &=(x^2u^2-2xyvu+y^2v^2)+(x^2v^2+2xyvu+y^2u^2) \\
     &=x^2u^2+y^2v^2+x^2v^2+y^2u^2                 \\
     &=(x^2+y^2)(u^2+v^u)                          \\
     &=|z|^2|w|^2
  \end{align*}
\end{proof}

\Theorem{0.0.3}{Complex differentiability}\label{d1d5d93}

A complex function $f(z):=u(z)+iv(z)$ is complex-differentiable at $z_0$ if and
only if $u$ and $v$ satisfy the \href{fb10fd3}{Cauchy-Riemann Equations} at
$z_0$.

To say a function is \href{e1e08f7}{\textbf{holomorphic}} is much stronger,
since a holomorphic function is complex-differentiable at every point of some
open subset of the complex plane $\C$.

\Definition{1.1.3}{Complex Partials}\label{ffea0ed}

\begin{gather*}
  \frac{\partial f}{\partial z}\equiv\frac12\left(\frac{\partial}{\partial x}-i\frac{\partial}{\partial y}\right)f \\[1em]
  \frac{\partial f}{\partial\bar z}\equiv\frac12\left(\frac{\partial}{\partial x}+i\frac{\partial}{\partial y}\right)f
\end{gather*}

\Definition{1.4.1}{Holomorphic functions}\label{e1e08f7}

Let $U\subset\mathbb C$ be open. Let $f:U\to\C$ be in $C^1(U)$. $f$ is said to
be \textit{holomorphic} if
$$
  \frac{\partial f}{\partial\bar z}=0
$$

\paragraph{Properties of holomorphic functions} If $f$ and $g$ are holomorphic in a domain $U$, then so are $f+g$, $f-g$, $fg$,
and $f\circ g$.

Additionally, if $g$ has no zeros in $U$, then $f/g$ is holomorphic too.

\paragraph{Examples of holomorphic functions}

Here are some building blocks to get started (remember that you can use these
with the properties above to show that other more complicated functions are
holomorphic too):
\begin{enumerati}
  \item $f(z)=1/z$ on $\C\sans{0}$
  \item $f(z)=1/p(z)$ on $\C$ where $p(z)\neq0$
  \item $f(z)=z$ on $\C$
\end{enumerati}

All these can be proved using a destructuring of $z:=x+iy$ and using
\href{ffea0ed}{Definition 1.1.3}.

Here are some functions that are not holomorphic:
\begin{enumerati}
  \item $f(z)=\bar z$
  \item $f(z)=|z|$
\end{enumerati}

\paragraph{Showing that a function is holomorphic}

If we can write $f\equiv u+iv$, and $u$ and $v$ have \textbf{continuous} first
partial derivatives and satisfy the \href{fb10fd3}{Cauchy-Riemann equations},
then $f$ is holomorphic.

\Definition{1.4.2}{Cauchy-Riemann Equations}\label{fb10fd3}

If $f(z) = u(z) + iv(z)$ is \href{e1e08f7}{holomorphic}, then
$$
  \frac{\partial u}{\partial x}=\frac{\partial v}{\partial y}
  \quad\text{and}\quad
  \frac{\partial u}{\partial y}=-\frac{\partial v}{\partial x}
$$

\Proposition{1.4.3}{}\label{d507f47}

If $f:U\to\C$ is $C^1$ and $f$ satisfies the Cauchy-Riemann equations, then
$$
  \frac{\partial f}{\partial z}=\frac{\partial f}{\partial x}=
  -i\frac{\partial f}{\partial y}
$$

\Definition{1.4.4}{Harmonic functions}\label{d53f60f}

Let $U\subset\C$ be open. Let $f:U\to\C$ be in $C^2(U)$. $f$ is said to be
\textit{harmonic} if
$$
  \frac{\partial^2f}{\partial x^2}+\frac{\partial^2f}{\partial y^2}=0
$$

The operator
$$
  \frac{\partial^2}{\partial x^2}+\frac{\partial^2}{\partial y^2}
$$

is called the \textit{Laplace operator}, or \textit{Laplacian}, and is denoted
by $\Delta$. We write
$$
  \Delta f=\frac{\partial^2f}{\partial x^2}+\frac{\partial^2f}{\partial y^2}
$$

\Theorem{1.5.1}{}\label{fdd4521}

Let $f,g\in C^1(U)$ where
$$
  U:=\{ (x,y)\in\R^2:|x-a|<\delta,\ |y-b|<\epsilon \}
$$

and let $\displaystyle\frac{\partial f}{\partial y}=\frac{\partial g}{\partial
  x}$ on $U$. Then there exists a function $h\in C^2(U)$ such that
$$
  \frac{\partial h}{\partial x}=f
  \quad\text{and}\quad
  \frac{\partial h}{\partial y}=g
$$

on $U$. If $f$ and $g$ are real-valued, then we may take $h$ to be real-valued
also.

\Theorem{1.5.3}{}\label{e7808d1}

Let $U\subset\C$ be either an open rectangle or open disc, and let $F$ be
holomorphic on $U$. Then there exists a holomorphic function $H$ on $U$ such
that
$$
  \frac{\partial H}{\partial z}=F
$$
on $U$.

\Definition{2.1.1}{Bounded $C^1$ functions}\label{c1f6d35}

A function $\phi:[a,b]\to\R$ is continuously differentiable (and we write
$\phi\in C^1([a,b])$) if

\begin{enumerata}
  \item $\phi$ is continuous on $[a,b]$
  \item $\phi'$ exists on $(a,b)$
  \item $\phi'$ has a continuous extension to $[a,b]$
\end{enumerata}

In other words, for (c) we require that
$$
  \lim_{t\to a^+}\phi'(t)\quad\text{and}\quad\lim_{t\to b^-}\phi'(t)
$$

both exist.

The motivation for this definition is so if $\phi\in C^1([a,b])$, then we have
\begin{align*}
  \phi(b)-\phi(a)
   &=\lim_{\epsilon\to0^+}\big(\phi(b-\epsilon)-\phi(a+\epsilon)\big) \\
   &=\lim_{\epsilon\to0^+}\int_{a+\epsilon}^{b-\epsilon}\phi'(t)\,dt  \\
   &=\int_a^b\phi'(t)\,dt
\end{align*}
and hence have the \href{b869dc0}{fundamental theorem of calculus}
hold for $\phi\in C^1([a,b])$.

\Definition{2.1.2}{Continuous complex curve}\label{e4132bc}

Let $\gamma:[a,b]\to\C$ be defined by $\gamma(t):=\gamma_1(t) + i\gamma_2(t)$.

Then $\gamma$ is said to be continuous on $[a,b]$ if both $\gamma_1$ and
$\gamma_2$ are.

The curve $\gamma$ is $C^1([a,b])$ if $\gamma_1$ and $\gamma_2$ are
continuously differentiable on $[a,b]$. Under these circumstances we will write
$$
  \gamma'(t)=\frac{d\gamma}{dt}=\frac{d\gamma_1}{dt}+i\frac{d\gamma_2}{dt}
$$

\Definition{2.1.3}{Complex integration}\label{c511702}

Let $\psi:[a,b]\to\C$ be continous on $[a,b]$. Write
$\psi(t)=\psi_1(t)+i\psi_2(t)$. Then we define
$$
  \int_a^b\psi(t)\,dt:=\int_a^b\psi_1(t)\,dt+i\int_a^b\psi_2(t)\,dt
$$

Using this definition along with Definitions \href{c1f6d35}{2.1.1} and
\href{e4132bc}{2.1.2}, we have that if $\gamma\in C^1([a,b])$ is
complex-valued, then
$$
  \gamma(b)-\gamma(a)=\int_a^b\gamma'(t)\,dt
$$

\Proposition{2.1.4}{}\label{f37b676}

Let $U\subseteq\C$ be open and let $\gamma:[a,b]\to U$ be a $C^1$ curve. If
$f:U\to\R$ and $f\in C^1(U)$ and we write
\begin{gather*}
  f:x+iy\mapsto f(x+iy) \\
  \gamma(t)=\gamma_1(t)+i\gamma_2(t)
\end{gather*}

then
\begin{align*}
  f(\gamma(b))-f(\gamma(a))
   &=\int_a^b (f\circ\gamma)'(t)\,dt \\
   &=\int_a^b\left(
  \frac{\partial f}{\partial x}(\gamma(t))\cdot\frac{d\gamma_1}{dt}+
  \frac{\partial f}{\partial y}(\gamma(t))\cdot\frac{d\gamma_2}{dt}
  \right)\,dt                        \\
   &=\int_a^b
  f_x(\gamma(t))\cdot\gamma_1'(t)+f_y(\gamma(t))\cdot\gamma_2'(t)
  \,dt
\end{align*}

This follows from \href{c511702}{Definition 2.1.3} and the \href{d969d46}{chain
rule}.

(the lack of an $i$ term is intentional. Remember that
$f\circ\gamma:\R\to\R$)

\Definition{2.1.5}{Complex line integral}\label{b1e96fc}

Let $U\subseteq\C$ open, $F:U\to\C$ continuous on $U$, and let $\gamma:[a,b]\to
U$ be a $C^1$ curve. Then we define the complex line integral
$$
  \oint_\gamma F(z)\,dz:=\int_a^bF(\gamma(t))\cdot\frac{d\gamma}{dt}\,dt
$$

\Proposition{2.1.6}{Holomorphic line integral}\label{c526c09}

Let $U\subseteq\C$ open, $F:U\to\C$ continuous on $U$, and let $\gamma:[a,b]\to
U$ be a $C^1$ curve. If $f$ is a holomorphic function on $U$, then
$$
  f(\gamma(b))-f(\gamma(a))=\oint_\gamma\frac{\partial f}{\partial z}(z)\,dz
$$

\Definition{2.1.6a}{Complex antiderivative}\label{cf21af2}

A function $f$ has an antiderivative $F$ if and only if, for every
$\gamma:[a,b]\to\C$,
$$
  \oint_\gamma f(z)\,dz=F(\gamma(b))-F(\gamma(a))
$$

This comes from using a holomorphic function on \href{f37b676}{Proposition
2.1.4}, and then applying \href{b1e96fc}{Definition 2.1.5}.

\Proposition{2.1.7}{Moving $||$ into integral}\label{bcec8b1}

Let $\phi:[a,b]\to\C$ be continuous. Then
$$
  \left|\int_a^b\phi(t)\,dt\right|\leq\int_a^b|\phi(t)|\,dt
$$

\Proposition{2.1.8}{Upper bound of line integral}\label{faf3f86}

Let $U\subseteq\C$ be open and $f\in C^0(U)$. Let $\gamma:[a,b]\to U$ be a
$C^1$ curve, and let $\ell(\gamma)$ be given by
$$
  \ell(\gamma):=\int_a^b\left|\frac{d\gamma}{dt}(t)\right|\,dt
$$

Then we have
$$
  \left|\oint_\gamma f(z)\,dz\right|\leq
  \Big(\sup_{t\in[a,b]}|f(\gamma(t))|\Big)\cdot\ell(\gamma)
$$

(Note that $\ell(\gamma)$ is the length of $\gamma$.)

\Proposition{2.1.9}{Parameterization-independence of line integrals}\label{f74efcb}

Let $U\subseteq\C$ be open and $f:U\to\C$ be a continuous function. Let
$\gamma:[a,b]\to U$ be a $C^1$ curve. Suppose that $\phi:[c,d]\to[a,b]$ is a
bijective increasing $C^1$ with a $C^1$ inverse.

% TODO: Is the inverse of a continuous bijective function continuous?
% NO.
% https://math.stackexchange.com/questions/368824/is-the-inverse-of-a-continuous-bijective-function-also-continuous

Let $\tilde\gamma=\gamma\circ\phi$. Then
$$\oint_{\tilde\gamma}f(z)\,dz=\oint_\gamma f(z)\,dz$$

The proof involves the standard change of variable formula from calculus.

\Theorem{2.2.1}{Existence of $f'$ on holomorphic $f$}\label{f75e43c}

Let $U\subseteq\C$ be open and let $f$ be holomorphic on $U$. Then $f'$ exists
at each point of $U$ and
$$f'(z)=\frac{\partial f}{\partial z}$$

for all $z\in U$.

As a result of this theorem, we often will write $f'=\dfrac{\partial
  f}{\partial z}$ when $f$ is holomorphic.

\Theorem{2.2.2}{Holomorphic by existence of derivative}\label{d037b0f}

Let $U\subseteq\C$ be open. If $f\in C^1(U)$ and $f$ has a complex derivative
at each point of $U$, then $f$ is holomorphic on $U$.

In other words, if a continuous, complex-valued function $f$ on $U$ has a
complex derivative at each point and if $f'$ is continuous on $U$, then $f$ is
holomorphic on $U$.

\Theorem{2.2.3}{Holomorphism and directional derivatives}\label{c41052e}

Let $f$ be holomorphic in a neighborhood $P\in\C$. Let $w_1,w_2\in\C$ have unit
modulus. Consider the directional derivatives
\begin{align*}
  D_{w_1}f(P) &:=\lim_{t\to0}\frac{f(P+tw_1)-f(P)}t \\
  D_{w_2}f(P) &:=\lim_{t\to0}\frac{f(P+tw_2)-f(P)}t
\end{align*}

Then
\begin{enumerata}
  \item $|D_{w_1}f(P)|=|D_{w_2}f(P)|$
  \item if $|f'(P)|\neq0$, then the directed angle from $w_1$ to $w_2$ equals the
  directed angle from $D_{w_1}f(P)$ to $D_{w_2}f(P)$.
\end{enumerata}

Note:
\begin{itemize}
  \item 2.2.3(a) alone implies that $f$ is holomorphic.
  \item 2.2.3(b) alone implies that $f$ is holomorphic.
\end{itemize}

\Lemma{2.3.1}{}\label{f8abd8b}

Let $(\alpha,\beta)\subseteq\R$ be an open interval and let
$H,F:(\alpha,\beta)\to\R$ be continuous functions. Let $p\in(\alpha,\beta)$ and
suppose that $dH/dx$ exists and equals $F(x)$ for all
$x\in(\alpha,\beta)\sans{p}$. Then $(dH/dx)(p)$ exists and $(dH/dx)(x)=F(x)$
for all $x\in(\alpha,\beta)$.
$$
  \forall_{x\in(\alpha,\beta)\sans{p}}:\frac{dH}{dx}(x)=F(x)\implies
  \forall_{x\in(\alpha,\beta)}:\frac{dH}{dx}(x)=F(x)
$$

It's as if the continuity fills in the gap at $p$.

\Theorem{2.3.2}{}\label{f017dd0}

Let $U\subseteq\C$ be either an open rectangle or an open disc and let $P\in
U$. Let $f$ and $g$ be continuous, real-valued functions on $U$ which are
continuously differentiable on $U\sans{P}$. Suppose further that
$$
  \frac{\partial f}{\partial y}=\frac{\partial g}{\partial x}
  \quad\text{on }U\sans{P}
$$

Then there exists a $C^1$ function $h:U\to\R$ such that
$$
  \frac{\partial h}{\partial x}=f,\quad
  \frac{\partial h}{\partial y}=g
$$

at every point of $U$ (including $P$).

\Theorem{2.3.3}{Existence of holomorphic antiderivative}\label{b2d9d89}

Let $U\subseteq\C$ be either an open rectangle or an open disc. Let $P\in U$ be
fixed. Suppose that $F$ is continuous on $U$ and holomorphic on $U\sans{P}$.
Then there is a holomorphic $H$ on $U$ such that $\partial H/\partial z=F$.

Note that since $H$ is holomorphic, by \href{f75e43c}{Theorem 2.2.1}, we can
write $H'=F$.

\Lemma{2.4.1}{}\label{c6c594a}

Let $\gamma$ be the boundary of a disc $D(z_0,r)$ in the complex plane,
equipped with the counterclockwise orientation. Let $z$ be a point inside the
circle $\partial D(z_0,r)$. Then
$$
  \frac1{2\pi i}\oint_\gamma\frac1{\zeta-z}\,d\zeta=1
$$

The proof involves considering the function
$$
  I(z):=\oint_\gamma\frac1{\zeta-z}\,d\zeta
$$
and showing that $I(z)$ is independent of $z$, and that $I(z_0)=2\pi i$.

\Theorem{2.4.2}{Cauchy integral formula}\label{e50677f}

Suppose that $U\subseteq\C$ is open and that $f$ is a holomorphic function on
$U$. Let $z_0\in U$ and let $r>0$ such that $\overline D(z_0,r)\subseteq U$ .
Let $\gamma:[0,1]\to\C$ be the $C^1$ curve $\gamma(t)=z_0+r\cos(2\pi
t)+ir\sin(2\pi t)$. Then, for each $z\in D(z_0,r)$,
$$
  f(z)=\frac1{2\pi i}\oint_\gamma\frac{f(\zeta)}{\zeta-z}\,d\zeta
$$

The converse of this theorem is true too: if $f$ is given by the Cauchy
integral formula, then $f$ is holomorphic.

\Example{2.4.2a}{Examples with Cauchy integral formula}\label{da0d68d}

Here's some ground-truth computations to get started. (Almost all problems in
MATH 466 can be re-routed back to these)

$$
  \oint_\gamma\zeta^k\,d\zeta=\begin{cases}
    0      & \text{if }k\neq-1 \\
    2\pi i & \text{if }k=-1
  \end{cases}\with(k\in\Z)
$$

\Theorem{2.4.3}{Cauchy integral theorem}\label{fb87a78}

If $f$ is a holomorphic function on an open disc $U\subseteq\C$, and if
$\gamma:[a,b]\to U$ is a $C^1$ curve in $U$ with $\gamma(a)=\gamma(b)$, then
$$\oint_\gamma f(z)\,dz=0$$

Note that this implies that the \href{e50677f}{Cauchy integral formula} gives a
zero whenever $z$ does not lie in the contour $\gamma$, since the integrand is
holomorphic. (Integrand is holomorphic because numerator is assumed to be
holomorphic, and the denominator is never zero.)

\begin{proof}
  By \href{e7808d1}{Theorem 1.5.3}, there is a holomorphic function
  $G:U\to\C$ with $G'=f$ on $U$. Since $\gamma(a)=\gamma(b)$, we have
  that
  $$
    0=G(\gamma(b))-G(\gamma(a))
  $$

  By \href{c526c09}{Proposition 2.1.6}, this equals
  $$\oint_\gamma G'(z)\,dz=\oint_\gamma f(z)\,dz$$

  (Reminder that since $G$ is holomorphic, $G'=\dfrac{\partial
    G}{\partial z}$ by \href{f75e43c}{Theorem 2.2.1})
\end{proof}

\Definition{2.6.1}{Piecewise $C^1$ curve}\label{baf22ac}

A piecewise $C^1$ curve $\gamma:[a,b]\to\C$ is a continuous function such that
there exists a finitne set of numbers $a_1\leq a_2\leq\ldots\leq a_k$
satisfying $a_1=a$ and $a_k=b$, and with the property that for every $i\leq
j\leq k-1,\ \gamma|_{[a_j,a_{j+1}]}$ is a $C^1$ curve.

$\gamma$ is a piecewise $C^1$ curve in an open set $U$ if
$\gamma([a,b])\subseteq U$.

Note that while joining $C^1(\R)$ curves may not lead to a piecewise $C^1(\R)$
curve, doing it in $\C$ somehow works.

\Definition{2.6.2}{Integrating over a piecewise $C^1$ curve}\label{a99ad34}

If $U\subseteq\C$ is open and $\gamma:[a,b]\to U$ is a piecewise $C^1$ curve in
$U$ and if $f:U\to\C$ is a continuous function on $U$, then

$$
  \oint_\gamma f(z)\,dz:=\sum_{j=1}^k\oint_{\gamma|_{[a_j,a_{j+1}]}}
  f(z)\,dz
$$

where $a_1,a_2,\ldots,a_k$ are as in \href{baf22ac}{Definition 2.6.1}.

\Lemma{2.6.3}{}\label{b6b6d51}

Let $U\subseteq\C$ be open. Let $\gamma:[a,b]\to U$ be a piecewise $C^1$ curve.
Let $\phi:[c,d]\to[a,b]$ be a piecewise $C^1$ strictly monotone increasing
function with $\phi(c)=a$ and $\phi(d)=b$. Let $f:U\to\C$ be a continuous
function on $U$. Then the function $\gamma\circ\phi:[c,d]\to U$ is a piecewise
$C^1$ curve and
$$\oint_\gamma f(z)\,dz=\oint_{\gamma\circ\phi}f(z)\,dz$$

(Really, $\{\gamma(t)\mid t\in[a,b]\}=\{(\gamma\circ\phi)(s)\mid s
\in[c,d]\}$, and there are no added crossovers on the parameterization
of $\gamma\circ\phi$ because $\phi$ is strictly monotone increasing.)

\Lemma{2.6.4}{}\label{cd28a8f}

Let $U\subseteq\C$ be open, $f:U\to\C$ a holomorphic function and
$\gamma:[a,b]\to U$ a piecewise $C^1$ curve. Then
$$
  f(\gamma(b))-f(\gamma(a))=\oint_\gamma f'(z)\,dz
$$

(This is really just \href{c526c09}{Proposition 2.1.6} restated
with a piecewise $C^1$ version of $\gamma$)

\Proposition{2.6.5}{}\label{b52bca5}

If $f:\C\sans{0}\to\C$ is a holomorphic function, and if $\gamma_r$ describes
the circle of radius $r$ around $0$, tranversed once around counterclockwise,
then, for any two positive numbers $r_1<r_2$,
$$
  \oint_{\gamma_{r_1}}f(z)\,dz=\oint_{\gamma_{r_2}}f(z)\,dz
$$

\Proposition{2.6.6}{}\label{ecbc559}

Let $0<r<R<\infty$ and define the annulus $\mathcal A:=\{z\in\C:r<|z|<R\}$. Let
$f:\mathcal A\to\C$ be a holomorphic function. If $r<r_1<r_2<R$ and if for each
$j$ the curve $\gamma_{r_j}$ describes the circle of radius $r_j$ around 0,
traversed once counterclockwise, then we have
$$
  \oint_{\gamma_{r_1}}f(z)\,dz=\oint_{\gamma_{r_2}}f(z)\,dz
$$

(On this annulus (donut), integrating a holomorphic $f$ along any two
circles centered at zero will yield the same value.)

\Theorem{2.6.7}{Cauchy integral formula and theorem: general form}\label{be5c80c}

Let $U\subseteq\C$ be open. Let $f:U\to\C$ be holomorphic. Then
$$\oint_\gamma f(z)\,dz=0$$

for any piecewise $C^1$ closed curve $\gamma$ in $U$ that can be deformed in
$U$ through closed curves to a closed curve lying entirely in a disc contained
in $U$.

In addition, suppose that $\overline D(z,r)\subseteq U$. Then
$$
  \frac1{2\pi i}\oint_\gamma\frac{f(\zeta)}{\zeta-z}\,d\zeta=f(z)
$$

for any piecewise $C^1$ closed curve $\gamma$ in $U\sans{z}$ that can be
continuously deformed in $U\sans{z}$ to $\partial D(z,r)$ equipped with
counterclockwise orientation.

\Theorem{3.1.1}{Analyticity of holomorphic functions}\label{e20a4ed}

Let $U\subseteq\C$ be open and let $f$ be a holomorphic on $U$. Then $f\in
C^\infty(U)$. Moreover, if $\overline D(P,r)\subseteq U$ and $z\in D(P,r)$,
then
$$
  \left(\frac\partial{\partial z}\right)^kf(z)=\frac{k!}{2\pi i}
  \oint_{|\zeta-P|=r}\frac{f(\zeta)}{(\zeta-z)^{k+1}}\,d\zeta
$$

for all $k\in\mathbb{N}_0$.

\Corollary{3.1.2}{Derivative of a holomorphic function is holomorphic}\label{ee189cf}

Let $U\subseteq\C$ be open. If $f:U\to\C$ is holomorphic, then $f':U\to\C$ is
holomorphic.

\Theorem{3.1.3}{}\label{c056da0}

If $\phi$ is a continuous function on $\{\zeta:|\zeta-P|=r\}$, then the
function $f$ given by
$$
  f(z):=\frac1{2\pi i}\oint_{|\zeta-P|=r}\frac{\phi(\zeta)}{\zeta-z}\,d\zeta
$$

is defined and holomorphic on $D(P,r)$.

This theorem induces a very strong way to create a holomorphic function.
Instead of differentiability, we only need a continuous $\phi$ to build a
holomorphic $f$.

\Theorem{3.1.4}{Morera's Theorem}\label{f378992}

Let $U\subseteq\C$ be open. Let $f:U\to\C$ be a continuous function on a
connected open subset $U$ of $\C$. Suppose that for every closed, piecewise
$C^1$ curve $\gamma:[0,1]\to U$, with $\gamma(0)=\gamma(1)$, we have
$$\oint_\gamma f(\zeta)\,d\zeta=0$$

Then $f$ is holomorphic on $U$.

\Lemma{3.2.1}{}\label{db44655}

The sequence $\{a_k\in\C\}$ converges to a limit if and only if for each
$\epsilon>0$ there is an $N_0$ such that $j,k\geq N_0$ implies that
$|a_j-a_k|<\epsilon$.

\Definition{3.2.2}{Complex power series}\label{b9a7c59}

Let $P\in\C$ be fixed. A \textit{complex power series} (centered at $P$) is an
expression of the form
$$
  \sum_{k=0}^\infty a_k(z-P)^k
$$

where $a_k$ for $k=\iter0\infty$ are complex constants.

Note that this power series expansion is only a formal expression. It may or
may not converge.

A \textit{necessary} condition for $\sum a_k(z-P)^k$ to converge is that
$a_k(z-P)^k\to0$.

\Lemma{3.2.3}{Abel's Theorem}\label{d5d5bdc}

If $\sum_{k=0}^\infty a_k(z-P)^k$ converges at some $z$, then the series
converges at each $w\in D(P,r)$, where $r=|z-P|$.

\Definition{3.2.4}{Radius of convergence of power series}\label{da6e337}

Let $\sum_{k=0}^\infty a_k(z-P)^k$ be a power series. Then
$$
  r:=\sup\Set{|w-P|}{\sum_{k=0}^\infty a_k(w-P)^k\text{ converges}}
$$

is called the \textit{radius of convergence} of the power series. We will call
$D(P,r)$ the disc of convergence.

\Lemma{3.2.5}{}\label{c7d0e1d}

If $\sum_{k=0}^\infty a_k(z-P)^k$ is a power series with radius of convergence
$r$, then the series converges for each $w\in D(P,r)$ and diverges for each $w$
such that $|w-P|>r$.

Note that the convergence or divergence question for $|w-P|=r$ is left open.

\Lemma{3.2.6}{Computing radius of convergence}\label{a9ba20f}

Using
$$
  \ell:\limsup_{k\to+\infty}|a_k|^{1/k},
$$

the radius of convergence $r$ of the power series $\sum_{k=0}^\infty
a_k(z-P)^k$ is given by
\begin{equation*}
  r=\begin{cases}
    1/\ell  & \text{if } \ell>0 \\
    +\infty & \text{if } \ell=0 \\
  \end{cases}
\end{equation*}

\Definition{3.2.7}{Uniform convergence of complex functions}\label{bba67e4}

A series $\sum_{k=0}^\infty f_k(z)$ of functions $f_k(z)$ converges uniformly
on a set $E$ to the function $g(z)$ if for each $\epsilon>0$ there is an $N_0$
such that if $N\geq N_0$, then
$$
  \left|g(z)-\sum_{k=0}^N f_k(z)\right|<\epsilon\with\forall z\in E
$$

The point is that $N_0$ does not depend on $z\in E$: There is, for each
$\epsilon$, an $N_0$ depending on $\epsilon$ (but not on $z$) that works for
all $z\in E$.

\Definition{3.2.8}{Uniformly Cauchy series}\label{bfe260e}

Let $\sum_{k=0}^\infty f_k(z)$ be a series of functions on a set $E$. The
series is said to be \textit{uniformly Cauchy} if, for any $\epsilon>0$, there
is a positive integer $N_0$ such that if $m\geq j\geq N_0$, then
$$
  \left|\sum_{k=j}^mf_k(z)\right|<\epsilon\with\forall z\in E
$$

If a series is uniformly Cauchy on a set $E$, then it converges uniformly on
$E$ to some limit function. From this it follows that if $\sum|f_k(z)|$ is
uniformly convergent, then $\sum f_k(z)$ is uniformly convergent (to some limit
function).

\Proposition{3.2.9}{}\label{ec076c1}

Let $\sum_{k=0}^\infty a_k(z-P)^k$ be a power series with radius of convergence
$r$. Then, for any number $R$ with $0\leq R<r$, the series
$\sum_{k=0}^\infty|a_k(z-P)^k|$ \href{bba67e4}{converges uniformly} on
$\overline D(P,R)$.

In particular, the series $\sum_{k=0}^{+\infty}a_k(z-P)^k$ converges uniformly
and absolutely on $\overline D(P,R)$.

\Lemma{3.2.10}{}\label{ccf2595}

If a power series
\begin{equation*}
  \sum_{j=0}^\infty a_j(z-P)^j\Tag{*}
\end{equation*}

has a radius of convergence $r>0$, then the series defines a $C^\infty$
function $f(z)$ on $D(P,r)$. The function $f$ is holomorphic on $D(P,r)$. The
series obtained by termwise differentiation $k$ times of $(*)$,
$$
  \sum_{j=k}^\infty\Big[j(j-1)\ldots(j-k+1)\Big]a_j(z-P)^{j-k}
$$

converges on $D(P,r)$, and its sum is $[\partial/\partial z]^kf(z)$ for each
$z\in D(P,r)$.

\Proposition{3.2.11}{}\label{ea8c930}

If both series $\sum_{j=0}^\infty a_j(z-P)^j$ and $\sum_{j=0}^\infty
b_j(z-P)^j$ converge on a disc $D(P,r)$, $r>0$, and if
$$
  \sum_{j=0}^\infty a_j(z-P)^j=\sum_{j=0}^\infty b_j(z-P)^j
$$

on $D(P,r)$, then $a_j=b_j$ for every $j$.

\Theorem{3.3.1}{}\label{b43209d}

Let $U\subseteq\C$ be open and let $f$ be holomorphic on $U$. Let $P\in U$ and
suppose that $D(P,r)\subseteq U$. Then the complex power series
$$
  \sum_{k=0}^\infty\frac{(\partial^kf/\partial z^k)(P)}{k!}(z-P)^k
$$

has radius of convergence at least $r$. It converges to $f(z)$ on $D(P,r)$.

\Theorem{3.4.1}{The Cauchy estimates}\label{a2d8611}

Let $U\subseteq\C$ be open and $f:U\to\C$ be holomorphic. Let $P\in U$ and
assume that the closed disc $\overline D(P,r)$, $r>0$, is contained in $U$. Set
$$
  M:=\sup_{z\in\overline D(P,r)}|f(z)|.
$$

Then for $k=1,2,3\ldots$ we have
$$
  \left|\frac{\partial^kf}{\partial z^k}(P)\right|\leq\frac{Mk!}{r^k}
$$

\Lemma{3.4.2}{}\label{c2c7fd1}

Let $U\subseteq\C$ be open and connected and $f:U\to\C$ be holomorphic. If
$\partial f/\partial z=0$ on $U$, then $f$ is constant on $U$.

\begin{proof}
  Since $f$ is holomorphic, $\partial f/\partial\bar z=0$. But we have assumed
  that $\partial f/\partial z=0$. Thus $\partial f/\partial x=\partial f/\partial
  y=0$. So $f$ is constant.
\end{proof}

\Theorem{3.4.3}{Liouville's Theorem}\label{cf6d8a9}

A bounded \href{d508dc8}{entire} function is constant.

\begin{proof}
  Let $f$ be entire and assume that $|f(z)|\leq M$ for all $z\in\C$. Fix a
  $P\in\C$ and let $r>0$. We apply the \href{a2d8611}{Cauchy estimate} for $k=1$
  on $\overline D(P,r)$. The result is
  $$
    \left|\frac{\partial f}{\partial z}(P)\right|\leq\frac Mr
  $$

  Since this inequality holds for all $r>0$, we can blow it up to $+\infty$ and
  conclude that
  $$
    \frac{\partial f}{\partial z}(P)=0
  $$

  But since $P$ is arbitrary, we conclude that
  $$
    \frac{\partial f}{\partial z}\equiv0
  $$

  By \href{c2c7fd1}{Lemma 3.4.2}, the proof is complete.
\end{proof}

\Theorem{3.4.4}{}\label{db4ce28}

If $f:\C\to\C$ is an entire function and if for some real number $C$ and some
positive integer $k$ it holds that
$$
  |f(z)|\leq C|z|^k
$$

for all $z\in\C$ with $|z|>1$, then $f$ is a polynomial in $z$ of degree at
most $k$.

\Theorem{3.4.5}{}\label{ae3b10d}

Let $p(z)$ be a non-constant (holomorphic) polynomial. Then $p$ has a root.
That is, there exists an $\alpha\in\C$ such that $p(\alpha)=0$.

This is in fact the fundamental theorem of algebra, and one of the most elegant
applications of \href{cf6d8a9}{Liouville's Theorem}.

\begin{proof}
  Suppose there isn't an $\alpha\in\C$ such that $p(\alpha)=0$. Then
  $$
    g(z):=\frac1{p(z)}
  $$
  is entire. Notice that as $|z|\to\infty$, $|p(z)|\to+\infty$. Thus
  $1/|p(z)|\to0$ as $|z|\to\infty$ and hence $g$ is bounded. By Liouville's
  Theorem, $g$ is constant; hence $p$ is constant. Contradiction!
\end{proof}

\Corollary{3.4.6}{}\label{cae4be0}

If $p(z)$ is a holomorphic polynomial of degree $k$, then there are $k$ complex
numbers $\alpha_1,\ldots,\alpha_k$ (not necessarily distinct) and a non-zero
constant $C$ such that
$$
  p(z)=C(z-\alpha_1)\ldots(z-\alpha_k)
$$

\Theorem{3.5.1}{}\label{c4e6d08}

Let $U\subseteq\C$ be an open set. Let $f_j:U\to\C$, $j=1,2,3\ldots$ be a
sequence of holomorphic functions. Suppose that there is a function $f:U\to\C$
such that, for each compact subset $E$ of $U$, the sequence $f_j|_E$ converges
uniformly to $f|_E$. Then $f$ is holomorphic on $U$. (In particular, $f\in
C^\infty(U)$)

\Corollary{3.5.2}{}\label{f5163f1}

If $f_j,f,U$ are as defined in \href{c4e6d08}{Theorem 3.5.1}, then for any
integer $k\in\{0,1,2\ldots\}$ we have
$$
  \left(\frac\partial{\partial z}\right)^kf_j(z)\to\left(\frac\partial{\partial z}\right)^kf(z)
$$

uniformly on compact sets.

\Theorem{3.6.1}{}\label{bdc2857}

Let $U\subseteq\C$ be a connected open set and let $f:U\to\C$ be holomorphic.
Let $\mathbf Z:=\Set{z\in U}{f(z)=0}$. If there are a $z_0\in\mathbf Z$ and a
sequence $\{z_j\}\subseteq\mathbf Z\sans{z_0}$ such that $z_j\to z_0$, then
$f\equiv0$.

\Corollary{3.6.2}{}\label{b919101}

Let $U\subseteq\C$ be connected and open, and $D(P,r)\subseteq U$ . If $f$ is
holomorphic on $U$ and $f|_{D(P,r)}\equiv0$, then $f\equiv0$ on $U$.

Note the strength of this statement. As long as $f$ is holomorphic, if it's
zero on just a tiny $D(P,r)$, then it is zero on the entire domain.

\Corollary{3.6.3}{}\label{ac6f6ea}

Let $U\subseteq\C$ be connected and open. Let $f,g$ be holomorphic on $U$. If
$\Set{z\in U}{f(z)=g(z)}$ has an \href{b0219cd}{accumulation point} in $U$,
then $f\equiv g$.

\Corollary{3.6.4}{}\label{faf57f7}

Let $U\subseteq\C$ be connected and open and let $f,g$ be holomorphic on $U$.
If $f\cdot g\equiv0$ on $U$, then either $f\equiv0$ on $U$ or $g\equiv0$ on
$U$.

\Corollary{3.6.5}{}\label{dccfe6b}

Let $U\subseteq\C$ be connected and open and let $f$ be holomorphic on $U$. If
there is a $P\in U$ such that
$$
  \left(\frac\partial{\partial z}\right)^jf(P)=0
$$

for every $j$, then $f\equiv0$.

\Corollary{3.6.6}{}\label{bd8ae3b}

If $f$ and $g$ are entire holomorphic functions and if $f(x)=g(x)$ for all
$x\in\R\subseteq\C$, then $f\equiv g$.

\Definition{4.1.0}{Types of singularities}\label{a7f062e}

Let $U\subseteq\C$ be open and $P\in U$. Suppose that $f:U\sans P\to\C$ is
holomorphic.

There are three possibilities for the behavior of $f$ near $P$:
\begin{enumerati}
  \item \textit{(Removable singularity)} $|f(z)|$ is bounded on $D(P,r)\sans{P}$
  for some $r>0$ with $D(P,r)\subseteq U$.
  \item \textit{(Pole)} $\lim_{z\to P}|f(z)|=+\infty$.
  \item \textit{(Essential singularity)} Neither (i) nor (ii) applies.
\end{enumerati}

\Theorem{4.1.1}{The Riemann removable singularities theorem}\label{f42a663}

Let $f:D(P,r)\sans P\to\C$ be holomorphic and bounded. Then
\begin{enumerata}
  \item $\lim_{z\to P}f(z)$ exists
  \item the function $\hat f:D(P,r)\to\C$ defined by
  $$
    \hat f(z)=\begin{cases}
      f(z)                                   & \text{if }z\neq P \\
      \displaystyle\lim_{\zeta\to P}f(\zeta) & \text{if }z=P
    \end{cases}
  $$

  is holomorphic
\end{enumerata}

Notice that, a priori, it is not even clear that $\lim_{z\to P}f(z)$ exists,
or, even if it does, that the function $\hat f$ has any regularity at $P$
beyond just continuity.

\Theorem{4.1.4}{Casorati-Weierstrass}\label{da8365b}

If $f:D(P,r_0)\sans P\to\C$ is holomorphic and $P$ is an essential singularity
of $f$, then $f(D(P,r)\sans P)$ is \href{e14819a}{dense} in $\C$ for any
$0<r<r_0$.

\Definition{4.2.0}{Laurent series}\label{cb20929}

A Laurent series on $D(P,r)$ is a (formal) expression of the form
$$
  \sum_{j=-\infty}^{+\infty} a_j(z-P)^j
$$

where $j$ are integer indices.

We say that the infinite series $\sum_{j=-\infty}^{+\infty}\alpha_j$ converges
if $\sum_{j=0}^{+\infty}\alpha_j$ and $\sum_{j=1}^{+\infty}\alpha_{-j}$
converge. In this case, we set
$$
  \sum_{j=-\infty}^{+\infty}\alpha_j=
  \bigg(\sum_{j=0}^{+\infty}\alpha_j\bigg)+
  \bigg(\sum_{j=1}^{+\infty}\alpha_{-j}\bigg)
$$

This doubly infinite series converges to a complex number $\sigma$ if and only
if for each $\epsilon>0$ there is an $N>0$ such that if $\ell\geq N$ and $k\geq
N$, then $|\sigma-\sum_{j=-k}^\ell\alpha_j|<\epsilon$.

It is important to realize that $\ell$ and $k$ are independent here. In
particular, the existence of the limit
$\lim_{k\to+\infty}\sum_{j=-k}^{+k}\alpha_j$ does not imply in general that
$\sum_{j=-\infty}^{+\infty}\alpha_j$ converges.
% TODO: though as I'm writing this I have no idea why Greene would claim this.
% Apparently it's only clear in an Exercise.

\Lemma{4.2.1}{}\label{d5253d9}

This is the analogue for Laurent series for \href{d5d5bdc}{Lemma 3.2.3}.

If $\sum_{j=-\infty}^{+\infty}a_j(z-P)^j$ converges at $z_1\neq P$ and at
$z_2\neq P$ and if $|z_1-P|<|z_2-P|$, then the series converges for all $z$
with $|z_1-P|<|z-P|<|z_2-P|$.

\Lemma{4.2.2}{}\label{a9a7710}

This is the analogue for Laurent series for \href{c7d0e1d}{Lemma 3.2.5}. Let
$$
  \sum_{j=-\infty}^{+\infty}a_j(z-P)^j
$$

converge at (at least) one point $z_0$. There are unique non-negative numbers
$r_1$ and $r_2$ ($r_1$ or $r_2$ may be $+\infty$) such that the series
converges absolutely for all $z$ with
$$
  r_1<|z-P|<r_2
$$

and diverges for all $z$ with
$$
  |z-P|<r_1\quad\text{or}\quad r_2<|z-P|
$$

\Proposition{4.2.4}{Uniqueness of Laurent expansion}\label{eafd61e}

Let $0\leq r_1<r_2\leq\infty$. If the Laurent series
$\sum_{j=-\infty}^{+\infty}a_j(z-P)^j$ converges on $D(P,r_2)\setminus\overline
D(P,r_1)$ to a function $f$, then, for any $r\in(r_1,r_2)$ and each $j\in\Z$,
we have
$$
  a_j=\frac1{2\pi i}\oint_{|\zeta-P|=r}\frac{f(\zeta)}{(\zeta-P)^{j+1}}\,d\zeta
$$

In particular, the $a_j$'s are uniquely determined by $f$.

\Theorem{4.3.1}{The \href{e50677f}{Cauchy integral formula} for an annulus}\label{ca5d89e}

Suppose that $0\leq r_1<r_2\leq+\infty$ and that $f:D(P,r_2)\setminus\overline
D(P,r_1)\to\C$ is holomorphic. Then, for each $s_1,s_2$ such that
$r_1<s_1<s_2<r_2$ and each $z\in D(P,s_2)\setminus\overline D(P,s_1)$, it holds
that
$$
  f(z)=\frac1{2\pi i}\oint_{|\zeta-P|=s_2}\frac{f(\zeta)}{\zeta-z}\,d\zeta
  -\frac1{2\pi i}\oint_{|\zeta-P|=s_1}\frac{f(\zeta)}{\zeta-z}\,d\zeta
$$

\Theorem{4.3.2}{The existence of Laurent expansions}\label{cd93e84}

If $0\leq r_1<r_2\leq+\infty$ and $f:D(P,r_2)\setminus\overline D(P,r_1)\to\C$
is holomorphic, then there exist complex numbers $a_j$ such that
$$
  \sum_{j=-\infty}^{+\infty}a_j(z-P)^j
$$

converges on $D(P,r_2)\setminus\overline D(P,r_1)$ to $f$. If
$r_1<s_1<s_2<r_2$, then the series converges absolutely and uniformly on
$D(P,s_2)\setminus\overline D(P,s_1)$.

\Proposition{4.3.3}{Laurent expansion of holomorphic functions}\label{e7fa5f8}

If $f:D(P,r)\sans P\to\C$ is holomorphic, then $f$ has a unique Laurent series
expansion.
$$
  f(z)=\sum_{j=-\infty}^{+\infty}a_j(z-P)^j
$$

which converges absolutely for $z\in D(P,r)\sans P$. The convergence is uniform
on compact subsets of $D(P,r)\sans P$. The coefficients are given by
$$
  a_j=\frac1{2\pi i}\oint_{\partial D(P,s)}\frac{f(\zeta)}{(\zeta-P)^{j+1}}\,d\zeta
$$

for any $0<s<r$.

There are three mutually exclusive possibilities for the Laurent series of this
proposition:
\begin{enumerati}
  \item $a_j=0$ for all $j<0$;
  \item for some $k>0$, $a_j=0$ for all $-\infty<j<-k$
  \item neither (i) nor (ii) applies
\end{enumerati}

These three cases correspond exactly to the \href{a7f062e}{three types of
isolated singularities}: (i) $\iff P$ is a removable singularity; (ii) $\iff P$
is a pole; (iii) $\iff P$ is an essential singularity.

\Proposition{4.4.1}{}\label{c1d2d0c}

Let $f$ be holomorphic on $D(P,r)\sans P$ and suppose that $f$ has a pole of
order $k$ at $P$. Then the Laurent series coefficients $a_j$ of $f$ expanded
about $P$, for $j=-k,-k+1,-k+2\ldots$, are given by the formula
$$
  a_j=\frac1{(k+j)!}\left(\frac\partial{\partial z}\right)^{k+j}
  \big((z-P)^k\cdot f\big)\Bigg|_{z=P}
$$

\Definition{4.5.1}{Holomorphically simply connected (HSC)}\label{d20898f}

An open set $U\subseteq\C$ is holomorphically simply connected if $U$ is
connected and if, for each holomorphic function $f:U\to\C$, there is a
holomorphic function $F:U\to\C$ such that $F'\equiv f$.

\Lemma{4.5.2}{}\label{f3a867e}

A connected open set $U$ is holomorphically simply connected if and only if for
each holomorphic function $f:U\to\C$ and each piecewise $C^1$ closed curve
$\gamma$ in $U$,
$$
  \oint_\gamma f(z)\,dz=0.
$$

\Definition{4.5.3a}{Residue of a function at a point}\label{ea3ff58}

The \textbf{residue} of a function $f$ at point $P$ is denoted by $\Res_f(P)$,
and is the coefficient of $(z-P)^{-1}$ in the \href{e7fa5f8}{Laurent expansion}
of $f$ about $P$.

In particular, if $f$ is holomorphic, then $\Res_f(P)$ is given by
$$
  \Res_f(P)=\frac1{2\pi i}\oint_\gamma f(\zeta)\,d\zeta
$$

where $\gamma$ is a counterclockwise simply closed curve around $P$ and not
including any other singularities inside the curve.

\Theorem{4.5.3}{Residue theorem}\label{e1efb5a}

Suppose that $U\subseteq\C$ is a \href{d20898f}{HSC} open set, and that
$\iter{P_1}{P_n}$ are distinct points of $U$. Suppose that
$f:U\sans{\iter{P_1}{P_n}}\to\C$ is a holomorphic function and $\gamma$ is a
closed, piecewise $C^1$ curve in $U\sans{\iter{P_1}{P_n}}$.

Set $R_j$ to be the coefficient of $(z-P_j)^{-1}$ in the Laurent expansion of
$f$ about $P_j$.

Then
$$
  \oint_\gamma f(z)\,dz=\sum_{j=1}^nR_j\cdot\left(\oint_\gamma\frac1{\zeta-P_j}\,d\zeta\right)
$$

Using the notation of \href{ea3ff58}{$\Res_f$} and
\href{bfdcc82}{$\Ind_\gamma$}, we can rewrite this as

$$
  \oint_\gamma f(z)\,dz=2\pi i\sum_{j=1}^n\Res_f(P_j)\cdot\Ind_\gamma(P_j)
$$

\Definition{4.5.4}{Index of a curve}\label{bfdcc82}

Let $\gamma:[a,b]\to\C$ be a piecewise $C^1$ closed curve. Suppose
$P\notin\gamma([a,b])$. Then the \textbf{index} of $\gamma$ with respect to
$P$, is defined as
$$
  \Ind_\gamma(P):=\frac1{2\pi i}\oint_\gamma\frac1{\zeta-P}\,d\zeta
$$

The index is also sometimes called the ``winding number of $\gamma$ about $P$".
As we will see later, $\Ind_\gamma(P)$ coincides with the number of times
$\gamma$ winds about $P$, counting orientation.

\Lemma{4.5.5}{Index of a curve is an integer}\label{ff34baf}

Let $\gamma:[a,b]\to\C$ be a piecewise $C^1$ closed curve. Suppose $P$ is a
point not on the image of that curve, then
$$
  \frac1{2\pi i}\oint_\gamma\frac1{\zeta-P}\,d\zeta
  \equiv
  \frac1{2\pi i}\int_a^b\frac{\gamma'(t)}{\gamma(t)-P}\,dt
$$

is an integer.

\Proposition{4.5.6}{}\label{a264ecd}

Let $f$ be a function with a pole of order $k$ at $P$. Then
$$
  \Res_f(P)=\frac1{(k-1)!}\left(\frac\partial{\partial z}\right)^{k-1}
  \Big((z-P)^kf(z)\Big)\Bigg|_{z=P}
$$

\begin{proof}
  This is the case $j=-1$ of \href{c1d2d0c}{Proposition 4.4.1}.
\end{proof}

\Definition{4.7.1}{Discrete sets}\label{d1b9ae6}

A set $S\in\C$ is discrete if and only if for each $z\in S$ there is a positive
number $r$ (depending on $S$ and $z$) such that
$$
  S\cap D(z,r)=\{z\}
$$

We also say in this circumstance that $S$ consists of isolated points.

\Definition{4.7.2}{Meromorphic functions}\label{cfba843}

A meromorphic function $f$ on and open set $U\subseteq\C$ with singular (as in
``singularity") set $S$ is a function $f:U\setminus S\to\C$ such that
\begin{enumerata}
  \item the set $S$ is closed in $U$ and is \href{d1b9ae6}{discrete},
  \item the function $f$ is holomorphic on $U\setminus S$ (note that $U\setminus S$ is
  necessarily open in $\C$),
  \item for each $z\in S$ and $r>0$ such that $D(z,r)\subseteq U$ and $S\cap
  D(z,r)=\{z\}$, the function
  $$
    f|_{D(z,r)\sans z}
  $$
  has a (finite order) pole at $z$.
\end{enumerata}

\Lemma{4.7.3}{Reciprocal of a holomorphy with zeros is meromorphic}\label{e672e2a}

Let $U\subseteq\C$ be connected and open, and let $f:U\to\C$ be a holomorphic
function with $f\not\equiv0$, then the function
$$
  F:U\setminus\Set{z}{f(z)=0}\to\C
$$

defined by $F(z)=1/f(z)$, is a meromorphic function on $U$ with singular set
(or pole set) equal to $\Set{z\in U}{f(z)=0}$.

\Definition{4.7.4}{}\label{f4cee1f}

Let $U\subseteq\C$ be open, and let $f:U\to\C$ be a holomorphic function.
Suppose that for some $R>0$, we have $\set{z\in U}{|z|>R}\subseteq U$. Define
$G:\set{z\in U}{0<|z|<1/R}\to\C$ by $G(z)=f(1/z)$. Then we say that
\begin{enumerati}
  \item $f$ has a removable singularity at $\infty$ if $G$ has a removable
  singularity at $0$.
  \item $f$ has a pole at $\infty$ if $G$ has a pole at $0$.
  \item $f$ has an essential singularity at $\infty$ if $G$ has an essential
  singularity at $0$.
\end{enumerati}

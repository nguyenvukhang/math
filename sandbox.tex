\def\pp#1#2{\frac{\partial #1}{\partial #2}}

\Result{0.0.1}{Taking partial derivatives w.r.t. $z$ of different functions}\label{f107aed}

Differentiating polynomials work exactly like reals.
$$
  \pp{}z z^n=nz^{n-1}
$$

Differentiating trigonometric identities are also the same as reals.

Product rule works exactly the same as reals.
$$
  \pp{}z(F\cdot G)=\pp Fz\cdot G+F\cdot\pp Gz
$$

Since quotient rule is derived from product rule, it also works the same as
wihth reals.

\Result{0.0.2}{Trying different functions with the Cauchy integral formula}\label{fbc836e}

Let $\gamma:=\partial D(P,r)$ where $D(P,r)$ is an open disc centered at $P$
with radius $r$. Let
$$
  I(z):=\oint_\gamma\frac1{\zeta-z}\,d\zeta
$$

Then it was shown in the textbook that $I(z)$ is constant because
$$
  \pp{}zI(z)=0\quad\text{and}\quad\pp{}{\bar z}I(z)=0
$$

and that
$$
  I(P)=2\pi i
$$
\def\pp#1#2{\frac{\partial #1}{\partial #2}}

\Result{0.0.1}{Taking partial derivatives w.r.t. $z$ of different functions}\label{f107aed}

Differentiating polynomials work exactly like reals.
$$
  \pp{}z z^n=nz^{n-1}
$$

Differentiating trigonometric identities are also the same as reals.

Product rule works exactly the same as reals.
$$
  \pp{}z(F\cdot G)=\pp Fz\cdot G+F\cdot\pp Gz
$$

Since quotient rule is derived from product rule, it also works the same as
wihth reals.

\Result{0.0.2}{Trying different functions with the Cauchy integral formula}\label{fbc836e}

Let $\gamma:=\partial D(P,r)$ where $D(P,r)$ is an open disc centered at $P$
with radius $r$. Let
$$
  I(z):=\oint_\gamma\frac1{\zeta-z}\,d\zeta
$$

Then it was shown in the textbook that $I(z)$ is constant because
$$
  \pp{}zI(z)=0\quad\text{and}\quad\pp{}{\bar z}I(z)=0
$$

and that
$$
  I(P)=2\pi i
$$

---

Show that
$$
  (z-P_1)^2(z-P_2)^2=1+z^4
$$

$P_1:=\frac1{\sqrt2}+\frac i{\sqrt2}$ and $P_2:=-\frac1{\sqrt2}+\frac i{\sqrt2}$

$P_1:=a+ci$ and $P_2:=b+ci$

$w=\frac1{\sqrt2}+\frac i{\sqrt2}$.

Hence $w^2=-1$

\begin{align*}
  (z-w)^2(z-\bar w)^2
   & = (z^2-2zw+i)(z^2-2z\bar w-i) \\
\end{align*}

\newpage
\Result{0.0.3}{Jacobian Matrix}\label{bcd0aa1}

Let $D\subset\R^n$ be open and $F:D\to\R^m$. $F$ is differentiable at $x\in D$
if there exists $J(x)\in\R^{m\times n}$ such that
\begin{equation*}
  \lim_{h\to0}\frac{F(x+h)-F(x)-J(x)h}{\norm h}=0\in\R^m\Tag{*}
\end{equation*}

where $h\in\R^n$.

The existence of the limit in $(*)$ effectively claims that every element of
the $\R^m$ vector on the LHS must go to zero as $h\to0$.

Unpack $F(x)$ and $J(x)$ with
$$
  F(x):=\begin{bmat}
    F_1(x) \\\vdots\\F_m(x)
  \end{bmat}, \Quad
  J(x):=\begin{bmat}
    J_{11} & \ldots & J_{1n} \\
    \vdots & \ddots & \vdots \\
    J_{m1} & \ldots & J_{mn} \\
  \end{bmat}
$$

\def\Ji{\begin{bmat}J_{i1}&\ldots&J_{in}\end{bmat}}

Then for each $i=\iter1m$:
\begin{equation*}
  \lim_{h\to0}\frac{F_i(x+h)-F_i(x)-{\Ji}h}{\norm h}=0\in\R
\end{equation*}

Define $J_i:=\Ji$. Then we have
\begin{align*}
  \lim_{h\to0}\frac{F_i(x+h)-F_i(x)-{J_i}h}{\norm h}
   & =0                                  \\
  \lim_{h\to0}\frac{F_i(x+h)-F_i(x)}{\norm h}
   & =\lim_{h\to0}\frac{{J_i}h}{\norm h}
\end{align*}

Replacing $h$ with $td$ where $t\in\R$ and $d\in\R^n$, and sending $t\to0$, we
have
\begin{equation*}
  \lim_{t\to0}\frac{F_i(x+td)-F_i(x)}{t}
  =\lim_{t\to0}\frac{{J_i}(td)}{t}
  =J_id
\end{equation*}

This shows that $J_i$ is the transpose of the gradient vector of $F_i$.
$$
  J_i=\nabla F_i(x)^T
$$

\newpage
\Result{0.0.4}{Hessian Matrix}\label{bb1bec9}

Firstly, note that at undergrad Y3 mathematics, I will only have to deal with
Hessians of $\R^n\to\R$ functions. Apparently, Hessians of $\R^n\to\R^m$
functions are 3-D matrices.

Let $D\subset\R^n$ be open and $F:D\to\R$. Let $F$ be twice differentiable at
$x\in D$. Consider the first derivative (Jacobian) first, we have
\begin{equation*}
  \lim_{h\to0}\frac{F(x+h)-F(x)-F'(x)h}{\norm h}=0\in\R\Tag{*}
\end{equation*}

This translates to
\begin{equation*}
  \lim_{t\downarrow0}\frac{F(x+td)-F(x)}t=F'(x)d
\end{equation*}

which implies that $F'(x)=\nabla F(x)^T$. Next, moving on to the second
derivative (Hessian).
\begin{align*}
  \lim_{h\to0}\frac{F'(x+h)-F'(x)-H(x)h}{\norm h} & =0\in\R^n
\end{align*}

Now, like before, we unpack this equation into each of the $n$ elements of the
row vectors on both sides of the equation. For each $i=\iter1n$,
\begin{equation*}
  \lim_{h\to0}\frac{F'_i(x+h)-F'_i(x)-(H(x)h)_i}{\norm h}=0
\end{equation*}

Replacing $h$ with $td$ where $t\in\R$ and $d\in\R^n$, we have
\begin{align*}
  \lim_{t\downarrow0}\frac{F'_i(x+td)-F'_i(x)-(H(x)td)_i}t
   & =0         \\
  \lim_{t\downarrow0}\frac{F'_i(x+td)-F'_i(x)}t
   & =(H(x)d)_i
\end{align*}

This implies that
\begin{equation*}
  \nabla F'_i(x)^Td=(H(x)d)_i
\end{equation*}

Let's make sense of this. $F'_i:\R^n\to\R$ since it maps $x$ to just one
element

$\implies F'_i(x)\in\R^n$

$\implies\nabla F'_i(x)\in\R^n$

$\implies\nabla F'_i(x)^Td\in\R$ and hence the equation above is in $\R$.

On the RHS, we have $(H(x)d)_i$

Taking $H_i(x)$ as the $i^\text{th}$ row of $H(x)$, we can rewrite this as
$H_i(x)^Td$, but in this form it's easy to see that we have
\begin{equation*}
  H_i(x)=\nabla F'_i(x)
\end{equation*}

In summary,
\begin{align*}
  H_f
   & =
  \def\p#1#2{\dfrac{\partial^2f}{\partial x_{#1}\partial x_{#2}}}
  \def\P#1{\dfrac{\partial^2f}{\partial x_{#1}^2}}
  \begin{bmat}
    \P1    & \p12   & \ldots & \p1n   \\[1.4em]
    \p21   & \P2    & \ldots & \p2n   \\[1.4em]
    \vdots & \vdots & \ddots & \vdots \\[0.7em]
    \p n1  & \p n2  & \ldots & \P n
  \end{bmat}
  =\begin{bmat}
     \nabla F_1'(x)^T \\[0.2em]
     \vdots           \\[0.4em]
     \nabla F_n'(x)^T
   \end{bmat}
\end{align*}

\newpage

\paragraph{GT1} $H(f(x))=J(\nabla f(x))^T$ (Wikipedia)
\paragraph{GT2} If $f:\R^n\to\R$ is differentiable, then
$$
  \lim_{h\to0}\frac{f(x+h)-f(x)}{\norm h}
  =\lim_{t\downarrow0}\frac{f(x+td)-f(x)}t
  =f'(x;d)
  =\nabla f(x)^Td
$$
\paragraph{GT3} $J=\nabla^Tf$ (Wikipedia)

\paragraph{COR1} Based on \textbf{GT2}, we have
$$
  \lim_{h\to0}\frac{f(x+h)-f(x)-L(x)h}{\norm h}=0\quad\iff\quad L(x)=\nabla f(x)^T
$$

This checks out with \textbf{GT3}

---

Now based on \textbf{GT1}, we have $H^T=J(\nabla f)$ and hence we use the
definition of the derivative:
\begin{align*}
  \lim_{h\to0}\frac{\nabla f(x+h)-\nabla f(x)-H^Th}{\norm h}=0\in\R^n
\end{align*}

Unpack this to the $i^\text{th}$ row:
\begin{align*}
  \lim_{h\to0}\frac{\nabla f_i(x+h)-\nabla f_i(x)-H_i^Th}{\norm h}=0\in\R
\end{align*}

where $H_i^T$ is the $i^\text{th}$ row of $H^T$. Now using \textbf{COR1},
\begin{align*}
  H_i^T
   & =\nabla(\nabla f_i(x))^T                              \\
   & =\nabla\left(\frac{\partial f}{\partial x_i}\right)^T \\
   & =\begin{bmat}
        \dfrac{\partial^2f}{\partial x_i\partial x_1} &
        \ldots                                        &
        \dfrac{\partial^2f}{\partial x_i\partial x_n}
      \end{bmat}
\end{align*}

But due to Clairaut's Theorem, which states that
$$
  \frac{\partial^2f}{\partial x\partial y}=
  \frac{\partial^2f}{\partial y\partial x}
$$

We have $H^T=H$ and hence all is good.
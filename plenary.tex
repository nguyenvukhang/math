\section{Plenary}\label{ce06306}

All the proofs and results I don't want to write twice.

\textbf{1.x.x)} Real analysis \\ %                   __real_analysis__
\textbf{2.x.x)} Calculus \\ %                             __calculus__
\textbf{3.x.x)} One-Liner Definitions \\ %         __one_liner_defns__
\textbf{4.x.x)} Linear Algebra \\ %                 __linear_algebra__
\textbf{5.x.x)} Discrete Mathematics \\ %           __discrete_maths__
\textbf{9.x.x)} General Stuff \\ %                   __general_stuff__

% ╭─────────────────────────────────────────────────────────────────╮
% │ 1. REAL ANALYSIS                              __real_analysis__ │
% ╰─────────────────────────────────────────────────────────────────╯

\Definition{1.1.1}{Supremum/Infimum}\label{b8f715e}

Let $X\subset\R$ be a non-empty. The supremum of $X$ is a real number $M=:\sup
X$ that satisfies
\begin{enumerati}
  \item $M$ is an upper bound of $X$, and
  \item if $M'$ is an upper bound of $X$, then $M'\geq M$
\end{enumerati}

that is, $M$ is the least upper bound of $X$. The infimum of $X$ is the
greatest lower bound of $X$.

\Definition{1.1.2}{Subsequential limit}\label{fd942fa}

Let $\{x_n\in\R\}$ sequence. $\bar x\in\R$ is called a \textbf{subsequential
limit} of $\{x_n\}$ if $\{x_n\}$ has a subsequence $\{x_{n_k}\}$ which
converges to $\bar x$.

\Definition{1.1.3}{Limit superior/limit inferior}\label{f4f2af4}

Let $\{x_n\in\R\}$ sequence, and let $S(x_n)$ be the set of all subsequential
limits of $\{x_n\}$.

Then we define the \textbf{limit superior} of $\{x_n\}$ to be
$$
  \limsup x_n := \sup S(x_n)
$$

and the \textbf{limit inferior} of $\{x_n\}$ to be
$$
  \liminf x_n := \inf S(x_n)
$$

Alternatively, we can also define them by
\begin{align*}
  \limsup x_n &:= \lim_{n\to\infty}\sup\Set{x_k}{k\geq n} \\
  \liminf x_n &:= \lim_{n\to\infty}\inf\Set{x_k}{k\geq n}
\end{align*}

\Definition{1.1.4}{Cluster point}\label{b0219cd}

Let $S$ be a subset of a topological space $X$. A point $x$ in $X$ is a cluster
point of the set $S$ if every neighborhood of $x$ contains at least one point
of $S$ different from $x$ itself.

A cluster point is also called a limit point or accumulation point.

In real analysis, $c\in\R$ is a cluster point of a non-empty set $A\subseteq\R$
if for every $\epsilon>0$ there exists a point $x\in A\sans{c}$ such that
$x\in(c-\epsilon,c+\epsilon)$.

In complex analysis, $c\in\C$ is a cluster point of a non-empty set
$A\subseteq\C$ if for every $\epsilon>0$ there exists a point $z\in A\sans{c}$
such that $z\in B_\epsilon(c)$.

\Definition{1.1.5}{Dense}\label{e14819a}

Informally, a subset $A$ of a topological space $X$ is said to be
\textbf{dense} in $X$ if every point of $X$ either belongs to $A$ or else is
arbitrarily ``close" to a member of $A$.

A subset $A$ if a topological space $X$ is said to be a dense subset of $X$ if
any of the following equivalent conditions are satisfied:
\begin{enumerati}
  \item The smallest \href{deadb92}{closed subset} of $X$ containing $A$ is $X$ itself.
  \item The closure of $A$ in $X$ is equal to $X$. ($\cl_XA=X$).
  \item Every point in $X$ either belongs to $A$ or is a \href{b0219cd}{cluster point}
  of $A$.
\end{enumerati}

\Definition{1.1.6}{Point of closure}\label{f928932}

For $S$ as a subset of a Euclidean space, $x$ is a point of closure of $S$ if
every open ball centered at $x$ contains a point of $S$ (this point can be $x$
itself).

\Definition{1.1.7}{Closure}\label{a07ff74}

The closure of a subset $S$ of points in a topological space can be defined
using any of the following equivalent definitions:
\begin{enumerati}
  \item $\cl S$ is the set of all \href{f928932}{points of closure} of $S$.
  \item $\cl S$ is the set $S$ together with all of its \href{f928932}{limit
  points}.
  \item $\cl S$ is the intersection of all closed sets containing $S$.
  \item $\cl S$ is the smallest closed set containing $S$.
  \item $\cl S$ is the union of $S$ and its boundary $\partial S$
\end{enumerati}

\Definition{1.1.8}{Open sets}\label{dd04b4d}

A subset $U$ of a metric space $(M,d)$ is called open if for any point $x$ in
$U$, there exists a real number $\epsilon>0$ such that any point $y\in M$
satisfying $d(x,y)<\epsilon$ belongs to $U$.

Equivalently, $U$ is open if every point $U$ has a neighborhood contained in
$U$.

An example of a metric space is $(\R^2,\norm\cdot)$.

\Definition{1.1.9}{Closed sets}\label{deadb92}

A subset $A$ of a topological space $(X,\tau)$ is closed if its complement
$X\setminus A$ is an \href{dd04b4d}{open} subset of $(X,\tau)$

A set $A$ is closed in $X$ if and only if it is equal to its closure $\cl A$ in
$X$.

Yet another equivalent definition is that a set is closed if and only if it
contains all of its boundary points.

\Definition{1.2.1}{Monotone sequences}\label{b5fad69}

A sequence $\{x_n\}$ is said to be \textbf{increasing} if $x_0\leq x_1\leq
x_2\leq\ldots$ and \textbf{decreasing} if $x_0\geq x_1\geq x_2\geq\ldots$ and
\textbf{monotone} if it is either increasing or decreasing.

\Theorem{1.2.2}{Monotone convergence theorem}\label{ca25eb7}

If $\{x_n\}$ is monotone and bounded, then $\{x_n\}$ converges.

$$
  \lim_{n\to\infty}=\begin{cases}
    \sup\{x_n:n\in\N\} & \text{if $\{x_n\}$ is increasing} \\
    \inf\{x_n:n\in\N\} & \text{if $\{x_n\}$ is decreasing}
  \end{cases}
$$

\Theorem{1.2.3}{Monotone subsequence theorem}\label{dddb70e}

Every sequence has a monotone subsequence.

\begin{proof}
  \def\xn{\{x_n\}}

  Let $\xn$ be a sequence. We call a term $x_p$ a \textbf{peak term} of $\xn$ if
  $$x_p\geq x_n\quad(\forall n\geq p)$$

  That is, all terms after $x_p$ never go above $x_p$ again. Then there are only
  two cases:

  \textbf{Case 1:} $\xn$ has infinitely many peak terms.

  Then the subsequence formed by all the peak terms form a decreasing subsequence
  of $\xn$.

  \textbf{Case 2:} $\xn$ has finitely many peak terms.

  Let $x_{p_1},x_{p_2},\ldots,x_{p_j}$ be \textbf{all} the peak terms.

  Let $n_1=p_j+1$ be the first term after the last peak term.

  Since $x_{n_1}$ is not a peak term. $\implies\exists n_2>n_1$ such that
  $x_{n_1}<x_{n_2}$.

  Since $x_{n_2}$ is not a peak term, $\implies\exists n_3>n_2$ such that
  $x_{n_2}<x_{n_3}$.

  Continuing indefinitely, we can form an increasing subsequence $\{x_{n_k}\}$.
\end{proof}

\Theorem{1.2.4}{Bolzano-Weierstrass Theorem}\label{d277ad0}

Every bounded sequence has a convergent subsequence.

\begin{proof}
  \def\xn{\{x_n\}}
  \def\xnk{\{x_{n_k}\}}

  Let $\xn$ be a bounded sequence. By the monotone subsequence theorem, $\xn$ has
  a monotone subsequence $\xnk$.

  Since $\xn$ is bounded, so is $\xnk$.

  Since $\xnk$ is both monotone and bounded, it follows from the
  \href{ca25eb7}{monotone convergence theorem} that $\xnk$ converges.
\end{proof}

% proof for MATH 378

\Theorem{1.2.5}{Monotone seq. with a convergent subseq. is convergent}\label{aaf3ba6}

Let $\{x_n\}$ be a monotone sequence with a subsequence $\{x_{n_k}\}$ that
converges to $L$. Then $\{x_n\}$ converges to $L$.

\begin{proof}
  WLOG, assume that $\{x_n\}$ is decreasing.
  Given any $\epsilon>0$, we want to find a $N_\epsilon\in\N$ such that
  $$
    |x_n-L|<\epsilon\quad\forall(n\geq N_\epsilon)
  $$

  Since $\{x_{n_k}\}$ is decreasing and converges to $L$, we can find (and fix) a
  $k_\epsilon$ such that
  \begin{equation*}
    0<x_{n_k}-L<\epsilon\quad\forall(k\geq k_\epsilon)\Tag{*}
  \end{equation*}
  So we take $N_\epsilon=n_{k_\epsilon}$. Then since $\{x_n\}$ is
  decreasing,
  $$
    x_n\leq x_{N_\epsilon}=x_{n_{k_\epsilon}}\quad\forall(n\geq N_\epsilon)
  $$
  Moreover, $L\leq x_n\leq x_{n_{k_\epsilon}}$, and hence
  $$0\leq x_n-L\leq x_{n_{k_\epsilon}}-L$$
  and from $(*)$, we have that this entire inequality $<\epsilon$, and hence
  $$0\leq x_n-L<\epsilon$$
  and finally
  $$|x_n-L|<\epsilon$$
\end{proof}

\Theorem{1.2.6}{Mean value theorem}\label{d37aa2b}

Let $f:[a,b]\to\R$ be continuous on the $[a,b]$, and differentiable on $(a,b)$.
Then there exists $c\in(a,b)$ such that
$$
  f'(c)=\frac{f(b)-f(a)}{b-a}
$$

Generalized to multiple variables, the mean value theorem can be written as:

Let $f:[a,b]\to\R$, where $a,b\in\R^n$, and $[a,b]$ refers to the line segment
connecting $a$ and $b$, namely
$$
  [a,b]:=\{\lambda a+(1-\lambda)b\mid\lambda\in[0,1]\}
$$

Suppose $f$ is continuous on $[a,b]$ and differentiable on $(a,b)$. Then there
exists $c\in[a,b]$ such that
$$
  \nabla f(c)^T(b-a)=f(b)-f(a)
$$

In some arguments, we use $f:[x,x+td]\to\R$ and write that there exists
$\eta\in[x,x+td]$ such that
$$
  \nabla f(\eta)^Td=\frac{f(x+td)-f(x)}t
$$

\Result{1.2.7}{Preprocessed limits}\label{ffc8953}

Let $k,\ell\in\N$ and $a,b,c\in\R$ be fixed.
\begin{enumerata}
  \def\li{\displaystyle\lim_{n\to\infty}}
  \item $\li\frac1{n^k}=0$
  \item $\li b^n=0$ \quad if \quad $|b|<1$
  \item $\li c^{\frac1n}=1$ \quad if \quad $c>0$
  \item $\li n^{\frac1n}=1$
  \item $\li \left(1+\frac1n\right)^n=e$
  \item $\li \left(1-\frac1n\right)^n=\frac1e$
  \item $\li \frac{n^k}{c^n}=0$ \quad if \quad $c>0$
\end{enumerata}

if $k<\ell$ and $1<a<b$, we have
$$
  n^k << n^\ell << a^n << b^n << n!
$$

\Theorem{1.2.8}{Bernoulli's inequality}\label{d44713f}
$$
  (1+x)^r\geq 1+rx
$$

This holds under any of the following conditions:
\begin{itemize}
  \item $r\in\Z,r\geq1$ and $x\in\R,x\geq-1$ (inequality is strict if
        $x\neq0$ and $r\geq2$)
  \item $r\in\Z,r\geq0$ and $x\in\R,x\geq-2$
  \item $r\in\Z$, $r$ is even and $x\in\R$
  \item $r\in\R,r\geq1$ and $x\in\R,x\geq-1$ (inequality is strict if
        $x\neq0$ and $r\neq1$)
\end{itemize}

and separately,
$$
  (1+x)^r\leq 1+rx
$$
for every $r\in\R,0\leq r\leq 1$ and $x\geq-1$.

\Result{1.2.9}{Limit to infinity of a rational function}\label{ccfddb1}

Let $P,Q$ be polynomial functions, where $Q$ is of a higher degree. Then
$$
  \lim_{x\to\infty}\frac{P(x)}{Q(x)}=0
$$

\begin{compute}
  Consider the example of
  $$
    \lim_{x\to\infty}\frac{x^2 - 3x}{x^3 + 2x + 5}
  $$
  We can divide both numerator and denominator by $x^2$ to obtain
  $$
    \lim_{x\to\infty}\frac{1 - \frac3x}{x + \frac2x + \frac5{x^2}}
  $$
  And we can see that the numerator $\to1$ while the denominator
  $\to\infty$.
\end{compute}

\Result{1.2.10}{Limit of $\frac{e^x}x$ as $x\to\infty$}\label{b905ee7}
$$
  \lim_{x\to\infty}\frac{e^x}x=\infty
$$

\begin{proof}
  % TODO: add Taylor series
  Since $e^x$ can be written as a Taylor series
  $$
    e^x=1 + x + \frac{x^2}2 +\ldots
  $$
  We have $e^x\geq 1 + x + x^2$ and hence
  \begin{align*}
    \lim_{x\to\infty}\frac{e^x}x
     &\geq\lim_{x\to\infty}\frac{1+x+\frac{x^2}2}x \\
     &=\lim_{x\to\infty}\frac1x + 1 + \frac{x}2    \\
     &= \infty
  \end{align*}
\end{proof}

\Result{1.2.11}{Limit of $\frac{\ln x}{x}$ as $x\to\infty$}\label{e2e1632}
$$
  \lim_{x\to\infty}\frac{\ln x}x = 0
$$

\begin{proof}
  Given any $\epsilon$, we have to find a $N\in\N$ such that
  $$
    n\geq N\implies\frac{\ln x}x<\epsilon
  $$

  But, if you've been paying attention,
  $$
    \frac{\ln x}x<\epsilon\iff\frac{e^{\epsilon x}}{\epsilon x}>\frac1\epsilon
  $$

  And since $\epsilon x\to+\infty$, using \href{b905ee7}{Result 1.2.10} with
  $\epsilon x$ as the limiting variable tells us that indeed there exists such an
  $N$, hence completing the proof.
\end{proof}

\Result{1.2.12}{Limit of a polynomial divided by an exponential}\label{f3540b0}

Let $a,b\in\R$ be fixed, with $b>1$. Then we have
$$
  \lim_{x\to\infty}\frac{x^a}{b^x}=0
$$

\begin{proof}
  Given any $\epsilon$ we want to find a $N\in\N$ such that
  $$
    n\geq N\implies\frac{x^a}{b^x}<\epsilon
  $$
  But this is equivalent to
  $$
    a\ln x-x\ln b<\ln\epsilon
  $$
  So it suffices to prove that
  $$
    a\ln x-x\ln b\to-\infty.
  $$
  Rewriting, we have
  \begin{align*}
    a\ln x-x\ln b
     &= x\left(a\cdot\frac{\ln x}{x}-\ln b\right)                       \\
     &= \infty(-\ln b) \Quad\because\href{e2e1632}{\frac{\ln x}{x}\to0} \\
     &= -\infty
  \end{align*}

  This completes the proof.
\end{proof}

\Definition{1.2.13}{Norm properties}\label{e0fff96}

Given a vector space $X$ over a subfield $F$ of the complex numbers $\C$, a
\textbf{norm} on $X$ is a real-valued function $p:X\to\R$ with the following
properties, where $|k|$ denotes the absolute value of a scalar $k$.
\begin{enumerate}
  \item [\textbf{(N1)}] \textit{(Positive definiteness)} For all $x\in
        X$, if $p(x)=0$ then $x=0$.
  \item [\textbf{(N2)}] \textit{(Absolute homogeneity)}
        $p(kx)=|k|p(x)$ for all $x\in X$ and scalars $k$.
  \item [\textbf{(N3)}] \textit{(Subadditivity/Triangle inequality)}
        $p(x+y)\leq p(x)+p(y)$ for all $x,y\in X$
\end{enumerate}

\Theorem{1.2.14}{Limit and limit superior/inferior}\label{dbc2c89}

Let $\{x_n\}$ be a bounded sequence. Then $\{x_n\}$ converges to $\bar x$ if
and only if
$$
  \limsup x_n=\liminf x_n=\bar x
$$

In short,
$$
  \lim_{n\to\infty}x_n=\bar x\text{ (exists)}\iff\limsup x_n=\liminf x_n=\bar x
$$

Sidenote: all convergent sequences are bounded, so the boundedness can be taken
for free once convergence is established.

\Result{1.2.15}{Limit of a polynomial divided by its successor}\label{b63d815}

Let $P$ be a polynomial. Show that
$$
  \lim_{x\to\infty}\frac{P(x)}{P(x+1)}=1
$$

\begin{proof}
  \def\one{\left(1+\frac1x\right)}
  We will write $P(x)$ as
  $$
    P(x):=\sum_{i=0}^n a_ix^i
  $$

  where $n$ is the degree of polynomial $P$.
  \begin{align*}
    P(x+1)
     &=P\big(x(1+\tfrac1x)\big)                                                                           \\
     &=a_0+a_1x\one+a_2x^2\one^2+\ldots+a_nx^n\one^n                                                      \\
     &=x^n\left[\frac{a_0}{x^n}+\frac{a_1}{x^{n-1}}\one+\frac{a_2}{x^{n-2}}\one^2+\ldots+a_n\one^n\right] \\[0.5em]
    P(x)
     &=x^n\left(\frac{a_0}{x^n}+\frac{a_1}{x^{n-1}}+\frac{a_2}{x^{n-2}}+\ldots+a_n\right)                 \\[0.5em]
    \implies \frac{P(x)}{P(x+1)}
     &=\frac
    {\frac{a_0}{x^n}+\frac{a_1}{x^{n-1}}\one+\frac{a_2}{x^{n-2}}\one^2+\ldots+a_n\one^n}
    {\frac{a_0}{x^n}+\frac{a_1}{x^{n-1}}+\frac{a_2}{x^{n-2}}+\ldots+a_n}                                  \\[0.5em]
    \implies \lim_{x\to\infty}\frac{P(x)}{P(x+1)}
     &=\frac{a_n}{a_n} = 1
  \end{align*}
\end{proof}

% ╭─────────────────────────────────────────────────────────────────╮
% │ 2. CALCULUS                                        __calculus__ │
% ╰─────────────────────────────────────────────────────────────────╯

\Theorem{2.1.1}{Fundamental theorem of calculus}\label{b869dc0}

\paragraph{First part} Let $f:[a,b]\to\R$ be continuous. Let $F:[a,b]\to\R$ be defined by
$$F(x)=\int_a^xf(t)\,dt$$

Then $F$ is uniformly continuous on $[a,b]$ and differentiable on $(a,b)$, and
$$F'(x)=f(x)$$

on $(a,b)$ so $F$ is an antiderivative of $f$.

\paragraph{Corollary}
$$\int_a^bf(t)\,dt=F(b)-F(a)$$

\paragraph{Second part} Let $f:[a,b]\to\R$. Let $F:[a,b]\to\R$ be continuous and also the
antiderivative of $f$ in $(a,b)$. If $f$ is Riemann integrable on $[a,b]$, then
$$\int_a^bf(t)\,dt=F(b)-F(a)$$

This is stronger than the corollary because it does not assume that $f$ is
continuous.

% ╭─────────────────────────────────────────────────────────────────╮
% │ 3. ONE-LINER DEFINITIONS                    __one_liner_defns__ │
% ╰─────────────────────────────────────────────────────────────────╯

\Definition{3.1.1}{Affine functions}\label{dcb7f73}

An affine function $f:\R^n\to\R^m$ is of the form
$$
  f(x)=Ax-b\with(A\in\R^{m\times n},b\in\R^m)
$$

\Definition{3.1.2}{Coercive functions}\label{e9c7871}

A function $f:\R^n\to\R$ is coercive if
$$
  \lim_{\norm x\to\infty}f(x)=+\infty
$$

\Definition{3.1.3}{Supercoercive functions}\label{a0444cc}

A function $f:\R^n\to\R$ is supercoercive if
$$
  \lim_{\norm x\to\infty}\frac{f(x)}{\norm x}=+\infty
$$

% ╭─────────────────────────────────────────────────────────────────╮
% │ 4. LINEAR ALGEBRA                            __linear_algebra__ │
% ╰─────────────────────────────────────────────────────────────────╯

\Remark{4.1.1}{Thinking about matrix dimensions}\label{d8bd136}

Let $A\in\R^{m\times n}$, $x\in\R^n$, and $b\in\R^m$. We can validly write
$$
  Ax = b
$$

So $A$ is a gadget that takes a $n$-dim vector and returns a $m$-dim vector.

($A$ has $m$ rows and $n$ columns)

\Definition{4.1.2}{Positive (semi)definiteness}\label{e25e722}

A symmetric matrix $A\in\R^{n\times n}$ is positive definite if
$$
  x^TAx>0\with\forall(x\in\R^n)
$$

and positive semidefinite if
$$
  x^TAx\geq0\with\forall(x\in\R^n)
$$

\Definition{4.1.3}{Inner product space}\label{cebd07a}

An inner product space is a vector space $V$ over the field $F$ together with
an \textit{inner product}.

An inner product is a map
$$
  \inner\cdot\cdot:V\times V\to F
$$

that satisfies the following for all $x,y,z\in V$ and $a,b\in F$:
\begin{enumerate}
  \item[\textbf{(I1)}] \textit{(Positive definiteness)} If $x$ is
        non-zero, then
        $$
          \inner xx>0
        $$
  \item[\textbf{(I2)}] \textit{(Linearity in the first argument)}
        $$
          \inner{ax+by}z = a\inner xz + b\inner yz
        $$
  \item[\textbf{(I3)}] \textit{(Conjugate symmetry)}
        $$
          \inner xy=\overline{\inner yx}
        $$
\end{enumerate}

% ╭─────────────────────────────────────────────────────────────────╮
% │ 5. DISCRETE MATHEMATICS                      __discrete_maths__ │
% ╰─────────────────────────────────────────────────────────────────╯

\Definition{5.0.0}{Absolute basics of boolean algebra}\label{ba4e2fa}

\begin{enumerata}
  \item Literal: a boolean variable $x$ or $\neg x$ (or $\bar x$)
  \item Conjunction: $\land$ (and)
  \item Disjunction: $\lor$ (or)
  \item Clause: a disjunction of \textbf{distinct} literals
\end{enumerata}

\Definition{5.1.1}{Conjunctive normal form}\label{ab60bb1}

This is a \href{ba4e2fa}{conjunction} of one or more \href{ba4e2fa}{clauses}.
$$
  (A\lor B)\land (C\lor D\lor E)
$$

\Definition{5.1.2}{Disjunctive normal form}\label{bb41c04}

This is a \href{ba4e2fa}{disjunction} of one or more
\href{ba4e2fa}{conjunctions}.
$$
  (A\land B)\lor (C\land D\land E)
$$

\Proposition{5.1.3}{Extending a CNF to 3 variables}\label{f33a84e}

Given a 1-variable or 2-variable \href{ab60bb1}{CNF}, we want to write a
logically equivalent 3-variable CNF. (Useful for 3-SAT problems). Here's how:

\paragraph{2-var CNF.} Say we have the expression $(x\lor y)$. This is logically equivalent to
$$
  (x\lor y\lor z)\land(x\lor y\lor\bar z)
$$

Notice that if $z$ is TRUE then we can drop the left branch because it's true
and hence
$$
  (x\lor y\lor z)\land(x\lor y\lor\bar z) \equiv(x\lor y\lor\bar z) \equiv (x\lor y)
$$

Similarly if $z$ is FALSE then we drop the right branch and get
$$
  (x\lor y\lor z)\land(x\lor y\lor\bar z) \equiv(x\lor y\lor z) \equiv (x\lor y)
$$

\paragraph{1-var CNF.} Now consider the expression $x$. Instead of adding just one variable we now add
two and get the logically equivalent expression
\begin{equation*}
  (x\lor y\lor z)\land
  (x\lor y\lor\bar z)\land
  (x\lor \bar y\lor z)\land
  (x\lor \bar y\lor\bar z)
\end{equation*}

If $(y,z)=(\text{TRUE},\text{TRUE})$ we can drop all clauses containing $y$ or
$z$, leaving us with
$$
  (x\lor \bar y\lor\bar z)
$$

but then $(\bar y,\bar z)=(\text{FALSE},\text{FALSE})$ and hence it is
logically equivalent to just $x$. Repeating this logic for all combinations of
$(y,z)$, we can see that $(*)$ is logically equivalent to $x$.

% ╭─────────────────────────────────────────────────────────────────╮
% │ 9. GENERAL STUFF                              __general_stuff__ │
% ╰─────────────────────────────────────────────────────────────────╯

\Definition{9.1.1}{Gamma function}\label{ce1fa3f}

The gamma function is defined via a convergent improper integral:
$$
  \Gamma(z):=\int_0^\infty e^{-t}t^{z-1}\,dt\Quad(\Re(z)>0)
$$

Note that ``$\displaystyle\int_0^\infty$" is a shorthand for
``$\displaystyle\lim_{k\to\infty}\int_0^k$".

Observe that $\Gamma(1)=1$.
$$\int_0^\infty e^{-t}\,dt=\Big[-e^{-t}\Big]_0^\infty=1$$

And that $\Gamma(n+1)=n\Gamma(n)$.
\begin{align*}
  \int_0^\infty e^{-t}t^n\,dt
   &= \Big[-e^{-t}\cdot t^n\Big]_0^\infty-\int_0^\infty-e^{-t}\cdot nt^{n-1}\,dt              \\
   &= 0+\int_0^\infty e^{-t}\cdot nt^{n-1}\,dt \Quad\text{(by \href{f3540b0}{Result 1.2.12})} \\
   &= n\Gamma(n)
\end{align*}

\Definition{9.1.2}{Language reductions}\label{e009acb}

If problem $A$ is reducible to problem $B$, we write $A\leq B$.

Reducing $A$ to $B$ by a \textbf{Many-one reduction} is to find a function $f$
which converts inputs $x$ of $A$ into inputs $f(x)$ of $B$, such that
$A(x)=B(f(x))$ under all values of $x$.

Reducing $A$ to $B$ by a \textbf{Turing reduction} is to find a function which
mimics the behavior of $A$ using an oracle of $B$. i.e., $A(x)=\text{TRUE}\iff
B(f(x))=\text{TRUE}$.

$A$ being reducible to $B$ means solving $A$ cannot be harder than the
combined difficulty of solving $B$ and executing the reduction. In
particular, if the reduction runs in constant-time, $A$ cannot be
harder than $B$. In order words, $\leq$ is referring to hardness.

\Definition{9.1.3}{Everything $\mathsf P$-, $\mathsf{NP}$-related}\label{e04bcbc}

This is a compilation of everything $\mathsf P$- and $\mathsf{NP}$-related. For
in-depth definitions, refer to each link below.

A problem $L$ is in $\mathsf P$ if it runs in polynomial time.

A problem $L$ is in $\mathsf{NP}$ if has a polynomial-time verifier.

We say that $L_1\leq_{\mathsf P}L_2$ if there is a polynomial-time
\href{e009acb}{reduction} from $L_1$ to $L_2$.

A problem $L$ is $\mathsf{NP}$-complete when $L\in\mathsf{NP}$, and every
problem $L'$ in $\mathsf{NP}$ has a polynomial-time reduction to it:
$$
  \forall L'\in\mathsf{NP}: L'\leq_{\mathsf P}L
$$

A problem $H$ is $\mathsf{NP}$-hard when for every $L\in\mathsf{NP}$, there is
a polynomial-time reduction from $L$ to $H$:
\begin{equation*}
  \forall L\in\mathsf{NP}: L\leq_{\mathsf P}H\Tag{*}
\end{equation*}

The only difference between $\mathsf{NP}$-complete and $\mathsf{NP}$-hard is
that $\mathsf{NP}$-complete has the extra constraint of having to be in
$\mathsf{NP}$.

$(*)$, based on a \href{e04bcbc}{previous remark}, also implies that
$H$ is at least as hard as the hardest problem in $\mathsf{NP}$.

\Theorem{9.1.4}{Cauchy-Schwarz inequality}\label{c503127}

For all vectors $u$ and $v$ of an inner product space,
$$
  |\inner uv|^2\leq\inner uu\cdot\inner vv
$$

This gives the following corollaries:
\begin{enumerata}

  \item Let $u_i,v_i\in\R$ for $i=\iter1n$ for any integer $n$. Then
  \begin{equation*}
    \def\su{\sum}\def\u{u_i}\def\v{v_i}
    \left(\su\u\v\right)^2\leq\left(\su\u^2\right)\left(\su\v^2\right)
  \end{equation*}

  \item Let $u_k,v_k\in\C$ for $k=\iter1n$ for any integer $n$. Then
  \begin{equation*}
    \def\su{\sum}\def\u{u_i}\def\v{v_i}
    \left|\su\u\v\right|^2\leq\left(\su|\u|^2\right)\left(\su|\v|^2\right)
  \end{equation*}
\end{enumerata}

\begin{proof}
  To prove (a), we observe that $\R^n$ equipped with the standard dot
  product is an \href{cebd07a}{inner product space}. We can build
  vectors $u,v\in\R^n$ by arranging $u_i$ for $i=\iter1n$ into a
  column vector and do the same for $v_i$ to get $v$.

  Then applying the Cauchy-Schwarz inequality with the standard dot product, we
  have
  $$
    |u\cdot v|^2\leq (u\cdot u)(v\cdot v)
  $$

  Which gives the statement in (a) exactly.

  To prove (b), instead of the \href{cebd07a}{inner product space} constructed
  from $\R^n$ and the standard dot product, we use $\C^n$ and the complex inner
  product defined by
  $$
    \inner uw:=u_1\bar w_1+\ldots+u_n\bar w_n
  $$

  Then by the Cauchy-Schwarz inequality, for all $u,w\in\C^n$,
  \begin{align*}
    |\inner uw|^2 &=\left|\sum u_k\bar w_k\right|^2                            \\
                  &\leq\inner uu\cdot\inner ww                                 \\
                  &=\left(\sum u_k\bar u_k\right)\left(\sum w_k\bar w_k\right) \\
                  &=\left(\sum|u_k|^2\right)\left(\sum|w_k|^2\right)
  \end{align*}

  That is
  $$
    |u_1\bar w_1+\ldots+u_n\bar w_n|^2\leq
    \Big(|u_1|^2+\ldots+|u_n|^2\Big)
    \Big(|\bar w_1|^2+\ldots+|\bar w_n|^2\Big)
  $$

  But since $|z|^2=|\bar z|^2$ for all $z\in\C$, we can define a collection
  $\iter{v_1}{v_n}$ such that $v_k=\bar w_k$, then we can rewrite the above
  inequality as
  $$
    |u_1v_1+\ldots+u_nv_n|^2\leq
    \Big(|u_1|^2+\ldots+|u_n|^2\Big)
    \Big(|v_1|^2+\ldots+|v_n|^2\Big)
  $$

  And finally since the collection $w_k$ were arbitrarily chosen, so can the
  collection $v_k$.
\end{proof}

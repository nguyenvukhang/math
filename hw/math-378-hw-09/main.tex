% \paragraph{Problem} \textbf{(LICQ implies MFCQ)} Let $\bar x\in\R^n$
% be feasible for the standard NLP such that LICQ holds at $\bar x$.
% Show that MFCQ holds at $\bar x$.
%
% \paragraph{Solution}

{\large\textbf{Q1}}

Let LICQ hold at $\bar x$. Define $M\in\R^{l\times n}$ by
$$
	M:=\begin{bmatrix}
		\nabla g_1(\bar x)^T \\
		\vdots               \\
		\nabla g_q(\bar x)^T \\
		\nabla h_1(\bar x)^T \\
		\vdots               \\
		\nabla h_p(\bar x)^T \\
	\end{bmatrix}
$$
where $q:=|I(\bar x)|$ is the cardinality of the active set, and $l:=p+q$.

By LICQ, the rows of $M$ are linearly independent, and hence any
system of the form
$$
	Md = b
$$
must have a solution for $d\in\R^n$, given any $b\in\R^l$. (It's the
same setup as having $l$ (linearly independent) equations to solve for
$l$ variables in a system of equations)

Exploiting this, we set $b$ as
$$
	b:=\begin{bmatrix}
		-1 & \ldots & -1 & 0 & \ldots & 0
	\end{bmatrix}^T
$$
Where it has $q$ ``$-1$"s and $p$ ``0"s.

We know that a solution $d\in\R^n$ is guaranteed to exist. Fix this
$d$. Now since
$$
	\begin{bmatrix}
		\nabla g_1(\bar x)^T \\
		\vdots               \\
		\nabla g_q(\bar x)^T \\
		\nabla h_1(\bar x)^T \\
		\vdots               \\
		\nabla h_p(\bar x)^T \\
	\end{bmatrix}d =
	\begin{bmatrix}
		\nabla g_1(\bar x)^Td \\
		\vdots                \\
		\nabla g_q(\bar x)^Td \\
		\nabla h_1(\bar x)^Td \\
		\vdots                \\
		\nabla h_p(\bar x)^Td \\
	\end{bmatrix} =
	\begin{bmatrix}
		-1     \\
		\vdots \\
		-1     \\
		0      \\
		\vdots \\
		0      \\
	\end{bmatrix}
$$

We can inspect it row-wise and see that
$$
  \nabla g_i(\bar x)^Td=-1<0\with(i\in I(\bar x))
$$

and
$$
  \nabla h_j(\bar x)^Td=0\with(j\in J)
$$

And hence MFCQ holds.
\def\TC{T_X(\bar x)}
\def\LC{L_X(\bar x)}
\def\lk{\lim_{k\to\infty}}
\def\nf{\nabla f}
\def\n{\nabla}
\def\f{\frac}
\def\bx{\bar x}

\newpage
{\large\textbf{Q3}}

Let $d\in\R^n$ be an arbitrary element of $T_X$. Then there exists a
sequence $\{x^k\in X\}\to\bar x$ and a sequence $\{t_k\}\downarrow0$
such that $\frac{x^k-\bar x}{t_k}\to d$.

Rearranging, we have
\begin{align*}
	\lim_{k\to\infty}\frac{x^k-\bar x}{t_k}      & =d                              \\
	\lim_{k\to\infty}\frac{x^k-\bar x-t_kd}{t_k} & =0                              \\
	x^k-\bar x-t_kd                              & = o(t_k)                        \\
	x^k                                          & = \bar x + t_kd + o(t_k)\Tag{*}
\end{align*}

Assume that $f$ is continuously differentiable. Then we have
\begin{align*}
	\nf(\bx)^Td
	 & = \lk\f{f(\bx+t_kd)-f(\bx)}{t_k}                   \\
	 & = \lk\f{f(\bx+t_k(d+\f{o(t_k)}{t_k}))-f(\bx)}{t_k}
\end{align*}

The last step is valid because $\f{o(t_k)}{t_k}\to0$ and hence we
still have $\bx+t_k\left(d+\f{o(t_k)}{t_k}\right)\to\bar x$.
\begin{align*}
	\nf(\bx)^Td
	       & =\lk\f{f(\bx+t_kd+o(t_k))-f(\bx)}{t_k} \\
	       & =\lk\f{f(x^k)-f(\bx)}{t_k}             \\
	o(t_k) & =f(x^k)-f(\bx)-t_k\nf(\bx)^Td          \\
	f(x^k) & =f(\bx)+t_k\nf(\bx)^Td+o(t_k)\Tag{**}
\end{align*}

Now use $h_j$ in the place of $f$. $h_j$ is indeed continuously
differentiable in the standard NLP. Moreover, we have we have
$h_j(x)=0$ for all feasible $x$. In particular,
\begin{align*} 0
	 & = \frac{h_j(x^k)}{t_k}                                                \\
	 & \stackrel{(**)}{=}\f1{t_k}\big(h_j(\bx)+t_k\n h_j(\bx)^Td+o(t_k)\big) \\
	 & = \n h_j(\bx)^Td+\frac{o(t_k)}{t_k}
\end{align*}

Taking the limit as $k\to\infty$, we have $\n h_j(\bx)^Td=0$ as
required.

Next, use $g_i$ in the place of $f$, and use a very similar argument:
\begin{align*} 0
	 & \geq\frac{g_i(x^k)}{t_k}                                              \\
	 & \stackrel{(**)}{=}\f1{t_k}\big(g_i(\bx)+t_k\n g_i(\bx)^Td+o(t_k)\big) \\
	 & =\n g_i(\bx)^Td+\frac{o(t_k)}{t_k}
\end{align*}

This time we remove the $g_i(\bx)$ term because $i\in I(\bx)$. Again
taking limit as $k\to\infty$, we have $\n g_i(\bx)^Td\leq0$ as
required.

And hence $\bx\in\LC$, completing the proof.

\newpage
{\large\textbf{Q4}}

Take any $v\in S$, and any $\lambda\geq0$. Our goal is to show that
$\lambda v\in S$.

Since $v\in S$, by definition of a tangent cone, there exists
$\{x^k\in S\}\to\bar x$, $\{t_k\}\downarrow0$ such that
\begin{equation*}
	\frac{x^k-\bar x}{t_k}\to v
\end{equation*}

Multiplying by $\lambda$ on both sides we get
\begin{equation*}
  \frac{x^k-\bar x}{t_k/\lambda}\to\lambda v\Tag{*}
\end{equation*}

Since $\lambda\geq0$ and $t_k\geq0$ for all $k$, we have
$t_k/\lambda\geq0$ for all $k$ and hence $\{t_k/\lambda\}\downarrow0$
too.

So now we've found a $\{x^k\in S\}\to\bar x$ and a
$\{t_k/\lambda\}\downarrow0$ such that $(*)$.
	
By definition of tangent cone $S$, we have $\lambda v\in
S$, completing the proof.
{\large\textbf{Q4}}

Take any $v\in S$, and any $\lambda\geq0$. Our goal is to show that
$\lambda v\in S$.

Since $v\in S$, by definition of a tangent cone, there exists
$\{x^k\in S\}\to\bar x$, $\{t_k\}\downarrow0$ such that
\begin{equation*}
	\frac{x^k-\bar x}{t_k}\to v
\end{equation*}

Multiplying by $\lambda$ on both sides we get
\begin{equation*}
  \frac{x^k-\bar x}{t_k/\lambda}\to\lambda v\Tag{*}
\end{equation*}

Since $\lambda\geq0$ and $t_k\geq0$ for all $k$, we have
$t_k/\lambda\geq0$ for all $k$ and hence $\{t_k/\lambda\}\downarrow0$
too.

So now we've found a $\{x^k\in S\}\to\bar x$ and a
$\{t_k/\lambda\}\downarrow0$ such that $(*)$.
	
By definition of tangent cone $S$, we have $\lambda v\in
S$, completing the proof.

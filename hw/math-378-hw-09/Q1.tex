% \paragraph{Problem} \textbf{(LICQ implies MFCQ)} Let $\bar x\in\R^n$
% be feasible for the standard NLP such that LICQ holds at $\bar x$.
% Show that MFCQ holds at $\bar x$.
%
% \paragraph{Solution}

{\large\textbf{Q1}}

Let LICQ hold at $\bar x$. Define $M\in\R^{l\times n}$ by
$$
	M:=\begin{bmatrix}
		\nabla g_1(\bar x)^T \\
		\vdots               \\
		\nabla g_q(\bar x)^T \\
		\nabla h_1(\bar x)^T \\
		\vdots               \\
		\nabla h_p(\bar x)^T \\
	\end{bmatrix}
$$
where $q:=|I(\bar x)|$ is the cardinality of the active set, and $l:=p+q$.

By LICQ, the rows of $M$ are linearly independent, and hence any
system of the form
$$
	Md = b
$$
must have a solution for $d\in\R^n$, given any $b\in\R^l$. (It's the
same setup as having $l$ (linearly independent) equations to solve for
$l$ variables in a system of equations)

Exploiting this, we set $b$ as
$$
	b:=\begin{bmatrix}
		-1 & \ldots & -1 & 0 & \ldots & 0
	\end{bmatrix}^T
$$
Where it has $q$ ``$-1$"s and $p$ ``0"s.

We know that a solution $d\in\R^n$ is guaranteed to exist. Fix this
$d$. Now since
$$
	\begin{bmatrix}
		\nabla g_1(\bar x)^T \\
		\vdots               \\
		\nabla g_q(\bar x)^T \\
		\nabla h_1(\bar x)^T \\
		\vdots               \\
		\nabla h_p(\bar x)^T \\
	\end{bmatrix}d =
	\begin{bmatrix}
		\nabla g_1(\bar x)^Td \\
		\vdots                \\
		\nabla g_q(\bar x)^Td \\
		\nabla h_1(\bar x)^Td \\
		\vdots                \\
		\nabla h_p(\bar x)^Td \\
	\end{bmatrix} =
	\begin{bmatrix}
		-1     \\
		\vdots \\
		-1     \\
		0      \\
		\vdots \\
		0      \\
	\end{bmatrix}
$$

We can inspect it row-wise and see that
$$
  \nabla g_i(\bar x)^Td=-1<0\with(i\in I(\bar x))
$$

and
$$
  \nabla h_j(\bar x)^Td=0\with(j\in J)
$$

And hence MFCQ holds.

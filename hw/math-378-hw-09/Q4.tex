% \Problem{2}{Tangent cone is a cone} Let
% $S\subset\R^n$ and $\bar x\in S$. Show that $T_S(\bar x)$ is a cone.
%
% \Definition{5.1.1}{Cones}
%
% A non-empty set $K\subset\R^n$ is said to be a cone if
% $$
% 	\lambda v\in K\with(\lambda\geq0,\ v\in K)
% $$
%
% \Definition{5.1.3}{Tangent cone}
%
% Let $S\subset\R^n$ and $\bar x\in S$. Then the set
% $$
% 	T_S(\bar x):=\Set{d\in\R^n}
% 	{\exists\{x^k\in S\}\to\bar x,\{t_k\}\downarrow0:\frac{x^k-\bar x}{t_k}\to d}
% $$
% is called the (Bouligand) tangent cone of $S$ at $\bar x$.
%
% \paragraph{Solution}

{\large\textbf{Q4}}

Take any $v\in S$, and any $\lambda\geq0$. Our goal is to show that
$\lambda v\in S$.

Since $v\in S$, by definition of a tangent cone, there exists
$\{x^k\in S\}\to\bar x$, $\{t_k\}\downarrow0$ such that
\begin{equation*}
	\frac{x^k-\bar x}{t_k}\to v
\end{equation*}

Multiplying by $\lambda$ on both sides we get
\begin{equation*}
  \frac{x^k-\bar x}{t_k/\lambda}\to\lambda v\Tag{*}
\end{equation*}

Since $\lambda\geq0$ and $t_k\geq0$ for all $k$, we have
$t_k/\lambda\geq0$ for all $k$ and hence $\{t_k/\lambda\}\downarrow0$
too.

So now we've found a $\{x^k\in S\}\to\bar x$ and a
$\{t_k/\lambda\}\downarrow0$ such that $(*)$.
	
By definition of tangent cone $S$, we have $\lambda v\in
S$, completing the proof.

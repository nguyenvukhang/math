% vim:ft=tex

\section{Plenary}\label{80eeafc}

All the proofs and results I don't want to write twice.

\textbf{1.x.x)} Real analysis

\Definition{1.1.1}{Monotone sequences}\label{d5142a8}

A sequence $\{x_k\}$ is said to be \textbf{increasing} if $x_0\leq
	x_1\leq x_2\leq\ldots$ and \textbf{decreasing} if $x_0\geq x_1\geq
	x_2\geq\ldots$ and \textbf{monotone} if it is either increasing or
decreasing.

\Theorem{1.1.2}{Monotone subsequence theorem}\label{dddb70e}

Every sequence has a monotone subsequence.

\begin{proof}
	\def\xk{\{x_k\}}

	Let $\xk$ be a sequence. We call a term $x_p$ a \textbf{peak term}
	of $\xk$ if
	$$x_p\geq x_n\quad(\forall n\geq p)$$

  That is, all terms after $x_p$ never go above $x_p$ again. Then
  there are only two cases:

  \textbf{Case 1:} $\xk$ has infinitely many peak terms.

  Then the subsequence formed by all the peak terms form a decreasing
  subsequence of $\xk$.

  \textbf{Case 2:} $\xk$ has finitely many peak terms.

  Let $x_{p_1},x_{p_2},\ldots,x_{p_j}$ be \textbf{all} the peak terms.

  Let $n_1=p_j+1$ be the first term after the last peak term.

  Since $x_{n_1}$ is not a peak term. $\implies\exists n_2>n_1$ such
  that $x_{n_1}<x_{n_2}$.

  Since $x_{n_2}$ is not a peak term, $\implies\exists n_3>n_2$ such
  that $x_{n_2}<x_{n_3}$.

  Continuing indefinitely, we can form an increasing subsequence
  $\{x_{n_k}\}$.
\end{proof}

\Theorem{1.1.1}{Convergence of a monotone subsequence}

Let $\{x_k\}$ be a \hyperref[d5142a8]{monotone sequence}

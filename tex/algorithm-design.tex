% vim:ft=tex

\def\NP{\mathsf{NP}}
\def\P{\mathsf{P}}
\def\coNP{\mathsf{coNP}}
\def\SAT{\mathsf{SAT}}
\def\VC{\mathsf{VC}}
\def\CONT{\mathsf{CONT}}
\def\Yes{\text{Yes}}
\def\No{\text{No}}

\section{Algorithm Design}\label{41ab460}

\Definition{0.0.1}{Vertex cover}\label{b530c81}

Let $G=(V,E)$ be an undirected graph. A vertex cover $U\subseteq V$ satisfies
$$
	(u,v)\in E\implies u\in U\lor u\in U.
$$

In other words, every edge in $E$ has at least one endpoint in the
vertex cover $U$. Such a set is said to \textit{cover} the edges of
$G$.

\Definition{1.1}{Flow network}

A \textit{flow network} is a directed graph $G=(V,E)$ with a single
\textit{source node} $s$ and a single \textit{target node} $t$, as
well as a positive number $c(e)$ for each edge $e\in E$, called the
capacity of $e$.

\Definition{1.3}{Flow}

Let $G$ be a flow network. A \textit{flow} on $G$ is given by a positive
number $f(e)$ for each edge $e$ in $G$ satisfying the following two
constraints:
\begin{itemize}
	\item \textbf{Capacity constraints.} For each edge $e\in E$, we have
	      $0\leq f(e)\leq c(e)$
	\item \textbf{Flow conservation.} For each vertex $v\in V$ that is
	      not the source or target vertex,
\end{itemize}
$$
	\sum_{e\text{ into }v}f(e)=\sum_{e\text{ leaving }v}f(e)
$$

The \textit{value} of flow $f$ is all of the flow leaving $s$:
$$
	\text{val}(f):=\sum_{e\text{ leaving }s}f(e)
$$

where $s$ is the source node of $G$.

\Problem{1}{Max Flow}

\textbf{Input:} A flow network $G$ with source $s$, target $t$, and positive
edge capacities $c(e)$ for $e\in E$.

\textbf{Output:} A flow $f$ with the maximum value.

\Definition{1.4}{Residual graph}

Let $G$ be a flow network and let $f$ be a flow on $G$. The
\textit{residual graph} of $G$ and $f$, denoted by $G_f$, is the
directed graph defined as follows:

The vertices of $G_f$ are the same as the vertices of $G$.

For each edge $e=(u,v)$ in $G$, if $f(e)<c(e)$ then we add the edge
$(u,v)$ to $G_f$, labelled with the number $c(e)-f(e)$. If $f(e)>0$,
then we also add the edge $(v,u)$ to $G_f$, labelled with the number
$f(e)$.

All paths from $s$ to $t$ in the residual graph correspond to a
sequence where flow can be re-routed to increase its value.

\Definition{3.0}{Binary representation}
\def\binrep{\{0,1\}^*}

Let
$$\binrep:=\{\epsilon,0,1,00,01,10,11,100,101,\ldots\}$$

be the set of all finite binary strings. (where $\epsilon$ is the
empty string.)

\Definition{3.1}{Decision problem}

A decision problem $L$ is a subset of $\binrep$. The computational
task corresponding to deciding if $L$ is ``Given a string
$x\in\binrep$, is $x\in L$?"

\Problem{8}{$L$-membership problem}

\textbf{Input:} A boolean string $x$.

\textbf{Output:} Decide if $x\in L$.

\Example{3.1.1}{Rewriting problems as $L$-membership problems}

\begin{itemize}
	\def\B{\binrep}
	\item\textit{Graph Connectivity.} Given a graph $G=(V,E)$, is it
	      connected?
	      \begin{align*}
		      L & =\{x\in\B\mid x\text{ encodes a connected graph}\} \\
		        & =\{G \mid G \text{ is a connected graph}\}
	      \end{align*}
	\item\textit{Max Flow (Decision Version).} Given a flow network
	      $G$ and a postive integer $k$, does the max flow on $G$ have
	      value $\geq k$?
	      \begin{align*}
		      L & =\{x\in\B\mid x\text{ encodes a $(G,k)$ such that $\text{val}(G)\geq k$}\} \\
		        & =\{(G,k)\mid\text{val}(G)\geq k \}
	      \end{align*}
	\item\textit{Sum.} Given $a,b,c\in\Z$, does $a+b=c$?
	      \begin{align*}
		      L & =\{x\in\B\mid x\text{ encodes $(a,b,c)$ such that $a+b=c$}\} \\
		        & =\{(a,b,c)\mid a+b=c\}
	      \end{align*}
\end{itemize}

Choice of encoding is important and definitely affects runtime.
However, for discussion we will assume that the most natural and
succint encoding is chosen.

\Definition{3.2}{Polynomial-time algorithms}\label{ee3be28}

An algorithm $A$ runs in \textit{polynomial time} if $\exists c\in\R$
s.t. $\forall x\in\binrep$, $A$ terminates after $O(|x|^c)$
computation steps.

A decision problem $L$ is \textit{polynomial-time computable} if there
exists a polynomial-time $A$ s.t. $\forall x\in\binrep$, $x\in L\iff
	A(x)=\Yes$.

We define
$$
	\P:=\{L\subseteq\binrep\mid L\text{ is polynomial-time computable}\}
$$

The complexity class $\P$ is our proxy for \textit{efficiently
	computable languages}. (``language" is another way to refer to $L$, in
addition to ``problem".)

\Definition{3.3.1}{Vertex Cover Problem}

$$
	\VC:=\{(G,k)\mid G\text{ is a graph with a
		\hyperref[b530c81]{vertex cover} of size}\leq k\}
$$

\Definition{3.3.2}{Satisfiability Problem}\label{d6893fe}

$$
	\mathsf{FORMSAT}:=\{F\mid F\text{ is a satisfiable boolean formula}\}
$$

\Definition{3.3}{Nondeterministic polynomial-time algorithms}\label{d93050e}

A decision problem $L$ has a polynomial-time verifier if there is a
polynomial time algorithm $B$ taking two strings $x,y$ as input, and a
polynomial $p(n)$ such that
$$
	x\in L\iff\exists y\in\binrep,|y|\leq p(|x|):B(x,y)=\Yes
$$
The complexity class
$$
	\mathsf{NP}:=\{L\subseteq\binrep\mid L\text{ has a polynomial-time verifier}\}
$$

\begin{itemize}
	\item A polynomial-time verifier for $\VC$ would take a graph
	      $(G,k)$ and a proposed vertex cover $U$ and check if $|U|\leq k$
	      and that $U$ is a vertex cover.
	\item A polynomial-time verifier for $\SAT$ would take in a boolean
	      formula $F$ and a proposed assignment $x$ and check if $F(x)=\Yes$.
\end{itemize}

Observe that $\P\subseteq\NP$. If $L$ has a polynomial-time algorithm,
then it also has a polynomial-time verifier.

\begin{proof}
	If $L\in\P$, then \hyperref[ee3be28]{by definition} there exists a polynomial-time algorithm $A$ with
	$$x\in L\iff A(x)=\Yes$$

	Then, following the \hyperref[d93050e]{defintion of $\NP$}, we need
	to find a polynomial-time verifier $B$ such that
	$$
		x\in L\iff\exists y\in\binrep,|y|\leq p(|x|):B(x,y)=\Yes
	$$
	But we can simply use $B(x,y):=A(x)$.
\end{proof}

\Definition{3.4}{Complement of a decision problem}

The complement of a decision problem $L$ is defined as
$$
	\bar L=\binrep\setminus L:=\{x\in\binrep\mid x\notin L\}
$$

Note that $\binrep=L\cup\bar L$ for any decision problem $L$.

\Exercise{3.5}{} Prove that if $L\in\P$ then $\bar L\notin\P$

% TODO: compute this

\Definition{3.6}{coNP}

The complexity class $\mathsf{coNP}$ is defined as
$$
	\mathsf{coNP}:=\{L\mid\bar L\in\mathsf{NP}\}
$$

For example, recall that $\mathsf{\hyperref[d6893fe]{SAT}}=\{F\mid
	F\text{ is a satisfiable boolean formula}\}$. Then
$$
	\overline{\SAT}=\left\{x\in\binrep\, \middle\vert
	\begin{array}{l}
		\text{$x$ is an invalid encoding of a formula, or} \\
		\text{$x$ encodes an unsatisfiable boolean formula}
	\end{array}
	\right\}
$$

But given $x\in\binrep$, it is easy to test its validity as a boolean
formula, hence we focus on the second constraint:
$$
	\CONT:=\{F\mid F\text{ is an unsatisfiable boolean formula}\}
$$

Note that since $\overline{\CONT}=\SAT\in\NP$, we have that
$\overline{\CONT}\in\NP$.

However, is $\CONT\in\NP$? Observe that $F\in\CONT$ if and only if
\textit{for every} assignment $x$ to the variables of $F$, we have
that $F(x)=\No$. Since there are $2^n$ assigments to check, it is not
clear how to encode this checking procedure into a single
polynomial-sized certificate. For this reason, many researchers
conjecture that $\NP\neq\coNP$.

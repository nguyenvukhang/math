% vim:ft=tex

\section{Algorithm Design}\label{41ab460}

\Definition{1.1}{Flow network}

A \textit{flow network} is a directed graph $G=(V,E)$ with a single
\textit{source node} $s$ and a single \textit{target node} $t$, as
well as a positive number $c(e)$ for each edge $e\in E$, called the
capacity of $e$.

\Definition{1.3}{Flow}

Let $G$ be a flow network. A \textit{flow} on $G$ is given by a positive
number $f(e)$ for each edge $e$ in $G$ satisfying the following two
constraints:
\begin{itemize}
	\item \textbf{Capacity constraints.} For each edge $e\in E$, we have
	      $0\leq f(e)\leq c(e)$
	\item \textbf{Flow conservation.} For each vertex $v\in V$ that is
	      not the source or target vertex,
\end{itemize}
$$
	\sum_{e\text{ into }v}f(e)=\sum_{e\text{ leaving }v}f(e)
$$

The \textit{value} of flow $f$ is all of the flow leaving $s$:
$$
	\text{val}(f):=\sum_{e\text{ leaving }s}f(e)
$$

where $s$ is the source node of $G$.

\Problem{1}{Max Flow}

\textbf{Input:} A flow network $G$ with source $s$, target $t$, and positive
edge capacities $c(e)$ for $e\in E$.

\textbf{Output:} A flow $f$ with the maximum value.

\Definition{1.4}{Residual graph}

Let $G$ be a flow network and let $f$ be a flow on $G$. The
\textit{residual graph} of $G$ and $f$, denoted by $G_f$, is the
directed graph defined as follows:

The vertices of $G_f$ are the same as the vertices of $G$.

For each edge $e=(u,v)$ in $G$, if $f(e)<c(e)$ then we add the edge
$(u,v)$ to $G_f$, labelled with the number $c(e)-f(e)$. If $f(e)>0$,
then we also add the edge $(v,u)$ to $G_f$, labelled with the number
$f(e)$.

All paths from $s$ to $t$ in the residual graph correspond to a
sequence where flow can be re-routed to increase its value.

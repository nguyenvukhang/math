% vim:ft=tex

\section{Complex Analysis}\label{f28d4dc}

\Definition{0.0.0}{General terminology}

\textbf{Entire function} is a complex-valued function that is holomorphic on $\C$.

A \textbf{real-valued} function is any function $f:X\to\R$.

A \textbf{complex-valued} function is any function $f:X\to\C$.

The $C^n$ notation:
\begin{itemize}
	\item $C^0$ : continuous
	\item $C^1$ : continuously differentiable
	\item $C^2$ : twice continuously differentiable
\end{itemize}

\Definition{1.1.3}{Complex Partials}\label{ffea0ed}

$$
	\frac{\partial f}{\partial z} :=
	\frac12\left(
	\frac{\partial}{\partial x}-i\frac{\partial}{\partial y}
	\right)f
$$

$$
	\frac{\partial f}{\partial\bar z} :=
	\frac12\left(
	\frac{\partial}{\partial x}+i\frac{\partial}{\partial y}
	\right)f
$$

\Definition{1.4.1}{Holomorphic functions}\label{e1e08f7}

Let $U\subset\mathbb C$ be open. Let $f:U\to\C$ be in $C^1(U)$. $f$ is
said to be \textit{holomorphic} if
$$
	\frac{\partial f}{\partial\bar z}=0
$$

\paragraph{Properties of holomorphic functions} If $f$ and $g$ are
holomorphic in a domain $U$, then so are $f+g$, $f-g$, $fg$, and
$f\circ g$.

Additionally, if $g$ has no zeros in $U$, then $f/g$ is holomorphic
too.

\paragraph{Examples of holomorphic functions}

Here are some building blocks to get started (remember that you can
use these with the properties above to show that other more
complicated functions are holomorphic too):

\begin{enumerate}
	\item $f(z)=1/z$ on $\C\setminus\{0\}$
	\item $f(z)=z$ on $\C$
\end{enumerate}

(I really thought this list was gonna be longer. To be continued.)

All these can be proved using a destructuring of $z:=x+iy$ and using
\hyperref[ffea0ed]{Definition 1.1.3}.

\Definition{1.4.2}{Cauchy-Riemann Equations}\label{fb10fd3}

If $f(z) = u(z) + iv(z)$ is \hyperref[e1e08f7]{holomorphic}, then
$$
	\frac{\partial u}{\partial x}=\frac{\partial v}{\partial y}
	\quad\text{and}\quad
	\frac{\partial u}{\partial y}=-\frac{\partial v}{\partial x}
$$

\Proposition{1.4.3}{}

If $f:U\to\C$ is $C^1$ and $f$ satisfies the Cauchy-Riemann equations,
then
$$
	\frac{\partial f}{\partial z}=\frac{\partial f}{\partial x}=
	-i\frac{\partial f}{\partial y}
$$

\Definition{1.4.4}{Harmonic functions}

Let $U\subset\C$ be open. Let $f:U\to\C$ be in $C^2(U)$. $f$ is said
to be \textit{harmonic} if
$$
	\frac{\partial^2f}{\partial x^2}+\frac{\partial^2f}{\partial y^2}=0
$$

The operator
$$
	\frac{\partial^2}{\partial x^2}+\frac{\partial^2}{\partial y^2}
$$

is called the \textit{Laplace operator}, or \textit{Laplacian}, and is
denoted by $\Delta$. We write
$$
	\Delta f=\frac{\partial^2f}{\partial x^2}+\frac{\partial^2f}{\partial y^2}
$$

\Theorem{1.5.1}{}

Let $f,g\in C^1(U)$ where
$$
	U:=\{ (x,y)\in\R^2:|x-a|<\delta,\ |y-b|<\epsilon \}
$$

and let $\displaystyle\frac{\partial f}{\partial y}=\frac{\partial
		g}{\partial x}$ on $U$. Then there exists a function $h\in C^2(U)$
such that
$$
	\frac{\partial h}{\partial x}=f
	\quad\text{and}\quad
	\frac{\partial h}{\partial y}=g
$$

on $U$. If $f$ and $g$ are real-valued, then we may take $h$ to be
real-valued also.

\Theorem{1.5.3}{}\label{e7808d1}

Let $U\subset\C$ be either an open rectangle or open disc, and let $F$
be holomorphic on $U$. Then there exists a holomorphic function $H$ on
$U$ such that
$$
	\frac{\partial H}{\partial z}=F
$$
on $U$.

\Definition{2.1.1}{Bounded $C^1$ functions}\label{c1f6d35}

A function $\phi:[a,b]\to\R$ is continuously differentiable
(and we write $\phi\in C^1([a,b])$) if

\begin{enumerate}[label=(\alph*)]
	\item $\phi$ is continuous on $[a,b]$
	\item $\phi'$ exists on $(a,b)$
	\item $\phi'$ has a continuous extension to $[a,b]$
\end{enumerate}

In other words, for (c) we require that
$$
	\lim_{t\to a^+}\phi'(t)\quad\text{and}\quad\lim_{t\to b^-}\phi'(t)
$$

both exist.

The motivation for this definition is so if $\phi\in C^1([a,b])$, then
we have
\begin{align*}
	\phi(b)-\phi(a)
	 & =\lim_{\epsilon\to0^+}\big(\phi(b-\epsilon)-\phi(a+\epsilon)\big) \\
	 & =\lim_{\epsilon\to0^+}\int_{a+\epsilon}^{b-\epsilon}\phi'(t)\,dt  \\
	 & =\int_a^b\phi'(t)\,dt
\end{align*}
and hence have the \hyperref[b869dc0]{fundamental theorem of calculus}
hold for $\phi\in C^1([a,b])$.

\Definition{2.1.2}{Continuous complex curve}\label{e4132bc}

Let $\gamma:[a,b]\to\C$ be defined by $\gamma(t):=\gamma_1(t) +
	i\gamma_2(t)$.

Then $\gamma$ is said to be continuous on $[a,b]$ if both $\gamma_1$
and $\gamma_2$ are.

The curve $\gamma$ is $C^1([a,b])$ if $\gamma_1$ and $\gamma_2$ are
continuously differentiable on $[a,b]$. Under these circumstances we
will write
$$
	\gamma'(t)=\frac{d\gamma}{dt}=\frac{d\gamma_1}{dt}+i\frac{d\gamma_2}{dt}
$$

\Definition{2.1.3}{Complex integration}\label{c511702}

Let $\psi:[a,b]\to\C$ be continous on $[a,b]$. Write
$\psi(t)=\psi_1(t)+i\psi_2(t)$. Then we define
$$
	\int_a^b\psi(t)\,dt:=\int_a^b\psi_1(t)\,dt+i\int_a^b\psi_2(t)\,dt
$$

Using this definition along with Definitions \hyperref[c1f6d35]{2.1.1}
and \hyperref[e4132bc]{2.1.2}, we have that if $\gamma\in C^1([a,b])$
is complex-valued, then
$$
	\gamma(b)-\gamma(a)=\int_a^b\gamma'(t)\,dt
$$

\Proposition{2.1.4}{}\label{f37b676}

Let $U\subseteq\C$ be open and let $\gamma:[a,b]\to U$ be a $C^1$
curve. If $f:U\to\R$ and $f\in C^1(U)$ and we write
$$
	f:x+iy\mapsto f(x+iy)\\\gamma(t)=\gamma_1(t)+i\gamma_2(t)
$$

then
\begin{align*}
	f(\gamma(b))-f(\gamma(a))
	 & =\int_a^b (f\circ\gamma)'(t)\,dt \\
	 & =\int_a^b\left(
	\frac{\partial f}{\partial x}(\gamma(t))\cdot\frac{d\gamma_1}{dt}+
	\frac{\partial f}{\partial y}(\gamma(t))\cdot\frac{d\gamma_2}{dt}
	\right)\,dt                         \\
	 & =\int_a^b
	f_x(\gamma(t))\cdot\gamma_1'(t)+f_y(\gamma(t))\cdot\gamma_2'(t)
	\,dt
\end{align*}

This follows from \hyperref[c511702]{Definition 2.1.3} and the
\hyperref[d969d46]{chain rule}.

(the lack of an $i$ term is intentional. Remember that
$f\circ\gamma:\R\to\R$)

\Definition{2.1.5}{Complex line integral}\label{b1e96fc}

Let $U\subseteq\C$ open, $F:U\to\C$ continuous on $U$, and let
$\gamma:[a,b]\to U$ be a $C^1$ curve. Then we define the complex line
integral
$$
	\oint_\gamma F(z)\,dz:=\int_a^bF(\gamma(t))\cdot\frac{d\gamma}{dt}\,dt
$$

\Proposition{2.1.6}{Holomorphic line integral}\label{c526c09}

Let $U\subseteq\C$ open, $F:U\to\C$ continuous on $U$, and let
$\gamma:[a,b]\to U$ be a $C^1$ curve. If $f$ is a holomorphic function
on $U$, then
$$
	f(\gamma(b))-f(\gamma(a))=\oint_\gamma\frac{\partial f}{\partial z}(z)\,dz
$$

This comes from using a holomorphic function on
\hyperref[f37b676]{Proposition 2.1.4}, and then applying
\hyperref[b1e96fc]{Definition 2.1.5}.

\Proposition{2.1.7}{Moving $||$ into integral}

Let $\phi:[a,b]\to\C$ be continuous. Then
$$
	\left|\int_a^b\phi(t)\,dt\right|\leq\int_a^b|\phi(t)|\,dt
$$

\Proposition{2.1.8}{Upper bound of line integral}

Let $U\subseteq\C$ be open and $f\in C^0(U)$. Let $\gamma:[a,b]\to U$
be a $C^1$ curve, and let $\ell(\gamma)$ be given by
$$
	\ell(\gamma):=\int_a^b\left|\frac{d\gamma}{dt}(t)\right|\,dt
$$

Then we have
$$
	\left|\oint_\gamma f(z)\,dz\right|\leq
	\Big(\sup_{t\in[a,b]}|f(\gamma(t))|\Big)\cdot\ell(\gamma)
$$

(Note that $\ell(\gamma)$ is the length of $\gamma$.)

\Proposition{2.1.9}{Parameterization-independence of line integrals}

Let $U\subseteq\C$ be open and $f:U\to\C$ be a continuous function.
Let $\gamma:[a,b]\to U$ be a $C^1$ curve. Suppose that
$\phi:[c,d]\to[a,b]$ is a bijective increasing $C^1$ with a $C^1$
inverse.

% TODO: Is the inverse of a continuous bijective function continuous?
% NO.
% https://math.stackexchange.com/questions/368824/is-the-inverse-of-a-continuous-bijective-function-also-continuous

Let $\tilde\gamma=\gamma\circ\phi$. Then
$$\oint_{\tilde\gamma}f(z)\,dz=\oint_\gamma f(z)\,dz$$

The proof involves the standard change of variable formula from
calculus.

\Theorem{2.2.1}{Existence of $f'$ on holomorphic $f$}\label{f75e43c}

Let $U\subseteq\C$ be open and let $f$ be holomorphic on $U$. Then
$f'$ exists at each point of $U$ and
$$f'(z)=\frac{\partial f}{\partial z}$$

for all $z\in U$.

As a result of this theorem, we often will write $f'=\dfrac{\partial
		f}{\partial z}$ when $f$ is holomorphic.

\Theorem{2.2.2}{Holomorphic by existence of derivative}

Let $U\subseteq\C$ be open. If $f\in C^1(U)$ and $f$ has a complex
derivative at each point of $U$, then $f$ is holomorphic on $U$.

In other words, if a continuous, complex-valued function $f$ on $U$ has
a complex derivative at each point and if $f'$ is continuous on $U$,
then $f$ is holomorphic on $U$.

\Theorem{2.2.3}{Holomorphism and directional derivatives}

Let $f$ be holomorphic in a neighborhood $P\in\C$. Let $w_1,w_2\in\C$
have unit modulus. Consider the directional derivatives
\begin{align*}
	D_{w_1}f(P) & :=\lim_{t\to0}\frac{f(P+tw_1)-f(P)}t \\
	D_{w_2}f(P) & :=\lim_{t\to0}\frac{f(P+tw_2)-f(P)}t
\end{align*}
Then
\begin{enumerate}[label=(\alph*)]
	\item $|D_{w_1}f(P)|=|D_{w_2}f(P)|$
	\item if $|f'(P)|\neq0$, then the directed angle from $w_1$ to
	      $w_2$ equals the directed angle from $D_{w_1}f(P)$ to $D_{w_2}f(P)$.
\end{enumerate}
Note:
\begin{itemize}
	\item 2.2.3(a) alone implies that $f$ is holomorphic.
	\item 2.2.3(b) alone implies that $f$ is holomorphic.
\end{itemize}

\Lemma{2.3.1}{}

Let $(\alpha,\beta)\subseteq\R$ be an open interval and let
$H,F:(\alpha,\beta)\to\R$ be continuous functions. Let
$p\in(\alpha,\beta)$ and suppose that $dH/dx$ exists and equals $F(x)$
for all $x\in(\alpha,\beta)\setminus\{p\}$. Then $(dH/dx)(p)$ exists
and $(dH/dx)(x)=F(x)$ for all $x\in(\alpha,\beta)$.
$$
	\forall_{x\in(\alpha,\beta)\setminus\{p\}}:\frac{dH}{dx}(x)=F(x)\implies
	\forall_{x\in(\alpha,\beta)}:\frac{dH}{dx}(x)=F(x)
$$

It's as if the continuity fills in the gap at $p$.

\Theorem{2.3.2}{}

Let $U\subseteq\C$ be either an open rectangle or an open disc and let
$P\in U$. Let $f$ and $g$ be continuous, real-valued functions on $U$
which are continuously differentiable on $U\setminus\{P\}$. Suppose
further that
$$
	\frac{\partial f}{\partial y}=\frac{\partial g}{\partial x}
	\quad\text{on }U\setminus\{P\}
$$

Then there exists a $C^1$ function $h:U\to\R$ such that
$$
	\frac{\partial h}{\partial x}=f,\quad
	\frac{\partial h}{\partial y}=g
$$

at every point of $U$ (including $P$).

\Theorem{2.3.3}{Existence of holomorphic antiderivative}

Let $U\subseteq\C$ be either an open rectangle or an open disc. Let
$P\in U$ be fixed. Suppose that $F$ is continuous on $U$ and
holomorphic on $U\setminus\{P\}$. Then there is a holomorphic $H$ on
$U$ such that $\partial H/\partial z=F$.

Note that since $H$ is holomorphic, by \hyperref[f75e43c]{Theorem
	2.2.1}, we can write $H'=F$.

\Lemma{2.4.1}{}

Let $\gamma$ be the boundary of a disc $D(z_0,r)$ in the complex
plane, equipped with the counterclockwise orientation. Let $z$ be a
point inside the circle $\partial D(z_0,r)$. Then
$$
	\frac1{2\pi i}\oint_\gamma\frac1{\zeta-z}\,d\zeta=1
$$

The proof involves considering the function
$$
	I(z):=\oint_\gamma\frac1{\zeta-z}\,d\zeta
$$
and showing that $I(z)$ is independent of $z$, and that $I(z_0)=2\pi i$.

\Theorem{2.4.2}{Cauchy integral formula}

Suppose that $U\subseteq\C$ is open and that $f$ is a holomorphic
function on $U$. Let $z_0\in U$ and let $r>0$ such that $\bar
	D(z_0,r)\subseteq U$ . Let $\gamma:[0,1]\to\C$ be the $C^1$ curve
$\gamma(t)=z_0+r\cos(2\pi t)+ir\sin(2\pi t)$. Then, for each $z\in
	D(z_0,r)$,
$$
	f(z)=\frac1{2\pi i}\oint_\gamma\frac{f(\zeta)}{\zeta-z}\,d\zeta
$$

The converse of this theorem is true too: if $f$ is given by the
Cauchy integral formula, then $f$ is holomorphic.

\Theorem{2.4.3}{Cauchy integral theorem}

If $f$ is a holomorphic function on an open disc $U\subseteq\C$, and
if $\gamma:[a,b]\to U$ is a $C^1$ curve in $U$ with
$\gamma(a)=\gamma(b)$, then
$$\oint_\gamma f(z)\,dz=0$$

\begin{proof}
  By \hyperref[e7808d1]{Theorem 1.5.3}, there is a holomorphic
  function $G:U\to\C$ with $G'=f$ on $U$. Since $\gamma(a)=\gamma(b)$,
  we have that
	$$0=G(\gamma(b))-G(\gamma(a))$$

  By \hyperref[c526c09]{Proposition 2.1.6}, this equals
	$$\oint_\gamma G'(z)\,dz=\oint_\gamma f(z)\,dz$$

	(Reminder that since $G$ is holomorphic, $G'=\dfrac{\partial
			G}{\partial z}$ by \hyperref[f75e43c]{Theorem 2.2.1})
\end{proof}

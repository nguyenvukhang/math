% vim:ft=tex

\section{Ordinary Differential Equations}\label{82569c2}

\subsection{First order differential equations}

\Result{1.1.1}{Separable}

$$\frac{dy}{dx}=\frac{p(x)}{q(y)}$$

To solve, we do
$$
	\int p(x)\,dx=\int q(y)\,dy
$$

\Result{1.1.2}{Homogeneous}

$$\frac{dy}{dx}=g\left(\frac yx\right)$$

An easier way to check if $\displaystyle\frac{dy}{dx}$ satisfies this
form is by letting $f(x,y):=\displaystyle\frac{dy}{dx}$ and verifying
that $f(x,y)=f(kx,ky)$.

To solve, let $u:=y/x$ and rewrite the DE in terms of $u$ and $x$.

\Result{1.1.3}{Linear}\label{b8c7e19}

$$\frac{dy}{dx}+p(x)y=q(x)$$

To solve, let $\ln u:=\int p(x)\,dx$ and jump to
$$
	\frac d{dx}uy=u\cdot q(x)
$$

(which is just the product rule of differentiation)

\Result{1.1.4}{Bernoulli}

$$\frac{dy}{dx}+p(x)y=q(x)\cdot y^n$$

If $n\in\{0,1\}$, we have the \hyperref[b8c7e19]{linear case}.

Use $u:=y^{1-n}$, eliminate all $y$s, and reduce to a linear DE in
$u$:

$$\frac{du}{dx}+(1-n)p(x)u=(1-n)q(x)$$

\Result{1.1.5}{Riccati}

$$\frac{dy}{dx}=p(x)y^2+q(x)y+r(x)$$

To solve, first find a basic solution $y_1$. (By inspection,
hopefully. Usually a solution will be given as part of the homework
problem)

Then let $\displaystyle y_2:=y_1+\frac1u$, substitute it into the
original DE, and reduce it to a linear DE in $u$:

$$-u'=(2py_1+q)u+p$$

\begin{compute}
	First we obtain $\displaystyle y_2'=y_1'-\frac{u'}{u^2}$. Then substitute into the original DE:
	\begin{align*}
		y_2' & = py_2^2+qy_2+r                                                                  \\
		y_1'-\frac{u'}{u^2}
		     & = p\left(y_1+\frac1u\right)^2+q\left(y_1+\frac1u\right)+r                        \\
		     & = p\left(y_1^2+\frac{2y_1}{u}+\frac1{u^2}\right)+q\left(y_1+\frac1u\right)+r     \\
		     & = (py_1^2+qy_1+r)+p\left(\frac{2y_1}{u}+\frac1{u^2}\right)+q\left(\frac1u\right)
	\end{align*}
	Since $y_1$ is a solution to the original DE,
	\begin{align*}
		-\frac{u'}{u^2} & = p\left(\frac{2y_1}{u}+\frac1{u^2}\right)+q\left(\frac1u\right) \\
		-u'             & =(2py_1+q)u+p
	\end{align*}
	\QED
\end{compute}

\Result{1.1.6}{Exact}

$$M(x,y)+N(x,y)\frac{dy}{dx}=0$$

Criteria: $M_y=N_x$.

The idea is to work towards a function $F$ where $F_x=M$, and $F_y=N$.
(Because then based on the original DE we'll have $F_x=0$)

To solve, integrate to find $F(x,y)=\int M(x,y)\,dx=\int N(x,y)\,dy$.

Then we have $F(x,y)=C$ for some constant $C$.

\subsection{Second order differential equations}

\begin{itemize}
	\item $ay''+by'+cy=g(t)$ : \hyperref[af8932c]{Method of undetermined
		      coefficients}
	\item $ax^2y''+bxy'+cy=0$ : \hyperref[c15a777]{Euler equations}
	\item $y''+p(x)y'+q(x)y=r(x)$ : Variation of parameters for either a
	      \hyperref[cc51fcb]{particular} or
	      \hyperref[d359b97]{complementary} solution.
\end{itemize}

\Definition{2.1.1}{Wronskian}\label{b70073b}

The Wronskian of two differentiable functions $f$ and $g$ is
$W(f,g):=fg'-gf'$.

More generally, for $n$ complex-valued functions $f_1,\ldots,f_n$
which are $n-1$ times differentiable on an interval $I$, the Wronskian
$W(f_1,\ldots,f_n)$ is a function on $x\in I$ defined by

$$
	W(f_1,\ldots,f_n)(x):=\det\begin{bmatrix}
		f_1(x)         & f_2(x)         & \ldots & f_n(x)         \\
		f_1'(x)        & f_2'(x)        & \ldots & f_n'(x)        \\
		\vdots         & \vdots         & \ddots & \vdots         \\
		f_1^{(n-1)}(x) & f_2^{(n-1)}(x) & \ldots & f_n^{(n-1)}(x) \\
	\end{bmatrix}
$$

\Result{2.1.2}{Method of undetermined coefficients}\label{af8932c}

$$ay''+by'+cy=g(t)$$

Define $\Sigma_{n,t}(A):=A_0t^n+A_1t^{n-1}+\ldots+A_n$ and $P_n(t):=a_0t^n+a_1t^{n-1}+\ldots+a_n$.

\begin{center}
	\def\squirl{\begin{cases}\sin\beta t\\\cos\beta t\end{cases}}
	\def\S#1{\Sigma_{n,t}({#1})}
	\begin{tabular}{|c|c|}
		\hline
		$g(t)$                      & $y(t)$                                                                 \\
		\hline
		$P_n(t)$                    & $\S A$                                                                 \\
		\hline
		$P_n(t)e^{\alpha t}$        & $t^se^{\alpha t}\cdot\S A$                                             \\
		\hline
		$P_n(t)e^{\alpha t}\squirl$ & $t^s\big[\S Ae^{\alpha t}\cos\beta t+\S Be^{\alpha t}\sin\beta t\big]$ \\
		\hline
	\end{tabular}
\end{center}

Here, $s$ is the smallest non-negative integer that ensures that no
term in $y(t)$ is a solution of the corresponding homogeneous
equation.

\Result{2.1.3}{Variation of parameters (particular: $y_p$)}\label{cc51fcb}

import \hyperref[b70073b]{Wronskian} for $W$.

$$y''+p(x)y'+q(x)y=r(x)$$

Suppose we already know that $y_1$ and $y_2$ satisfy the corresponding
homogeneous equation.

Then we can jump to
$$y_p=v_1y_1+v_2y_2$$

where $\displaystyle v_1:=\frac{-y_2r}{W(y_1,y_2)}$ and $\displaystyle
	v_2:=\frac{y_1r}{W(y_1,y_2)}$.


\Result{2.1.4}{Variation of parameters (complementary: $y_c$)}\label{d359b97}

$$y''+p(x)y'+q(x)y=r(x)$$

Suppose we already know that $y_1$ is a solution. Then let
$$y_2:=vy_1$$

Then substituting $y_2$ back into the original DE, we have
$$v''y_1+v'(2y_1'+py_1)=0$$

\begin{compute}
	$$y_2:=vy_1;\quad y_2'=v'y_1+vy_1';\quad y_2''=v''y_1+2v'y_1'+vy_1''$$

	Substituting this back into the original DE:
	$$(v''y_1+2v'y_1'+vy_1'')+p(v'y_1+vy_1')+q(vy_1)=r$$

	But since $y_1$ is known to be a solution:
	\begin{align*}
		(v''y_1+2v'y_1')+p(v'y_1) & =0 \\
		v''y_1+v'(2y_1'+py_1)     & =0
	\end{align*}
	\QED
\end{compute}

Which is a first-order linear equation in $v'$. Use $u:=v'$ to solve
for $u$ and then substitute everything back to find $y$.

\Result{2.1.5}{Euler equations}\label{c15a777}

$$ax^2y''+bxy'+cy=0$$

Try $y=x^r$ for some $r\in\C$ to be found.

If two distinct roots: $y:=Ax^{r_1}+Bx^{r_2}$.

If one distinct root: $y:=Ax^r+B\ln(x)x^r$.

If complex roots ($r=\alpha\pm\beta i$): $y:=Ax^\alpha\cos(\beta\ln x)+Bx^\alpha\sin(\beta\ln x)$

\subsection{Higher order differential equations}

% Lecture 12 page 6

\begin{equation*}\def\y#1{y^{(#1)}}
	\y n+p_1\y{n-1}+p_2\y{n-2}+\ldots+p_{n-1}y'+p_ n y=q
\end{equation*}

where $\iter{p_1}{p_n},q:\R\to\R$.

For these we split into a few cases:
\begin{itemize}
	\item\hyperref[cd8a29a]{Constant coefficients}
	\item\hyperref[a0f8e0c]{Euler equations}
	\item\hyperref[de785b7]{Variation of parameters}
\end{itemize}

\Result{3.1.1}{Constant coefficients}\label{cd8a29a}

\begin{equation*}\def\y#1{y^{(#1)}}
	\y n+a_1\y{n-1}+a_2\y{n-2}+\ldots+a_{n-1}y'+a_ny=0\tag*{($*$)}
\end{equation*}
where $\iter{a_1}{a_n}\in\R$ fixed.

Then try $y=e^{rt}$. Substituting that into $(*)$, we'll obtain
$$a_0r^n+a_1r^{n-1}+\ldots+a_n=0$$

Since $\iter{a_1}{a_n}$ are given, we can solve for $r$.

If all the roots of $r$ are real and no two are equal, then we have
$n$ distinct solutions $\iter{e^{r_1t}}{e^{r_nt}}$ of equation $(*)$.

Moreover, if these solutions are linearly independent, then the
general solution to $(*)$ is
$$y_g=c_1e^{r_1t}+c_2e^{r_2t}+\ldots+c_ne^{r_nt}$$

One way to establish the linear independence is to evaluate their
\hyperref[b70073b]{Wronskian}.

\Result{3.1.2}{Euler equations}\label{a0f8e0c}

\begin{equation*}\def\y#1{x^{#1}y^{(#1)}}
	a_0\y{n}+a_1\y{n-1}+\ldots+a_ny=0\tag*{($*$)}
\end{equation*}
where $\iter{a_1}{a_n}\in\R$ fixed.

Then try $y=x^{r}$. Substituting that into $(*)$, we'll obtain
$$
	\sum_{i=0}^n\left(\prod_{j=r-i+1}^rj\right) a_ix^r=0
$$

Which is then
$$
	\sum_{i=0}^n\left(\prod_{j=r-i+1}^rj\right) a_i=0
$$

\Result{3.1.3}{Variation of parameters}\label{de785b7}

\Result{3.1.4}{Radius of convergence of power series}\label{ea9daf1}

Consider the power series
$$
	\sum_{k=0}^n(ax)^k=1+ax+a^2x^2+a^3x^3+\ldots
$$

As $n\to\infty$,
$$
	\sum_{k=0}^n(ax)^k\begin{cases}
		\text{is convergent on }\dfrac1{1-ax} & \text{if }|ax|<1 \\
		\text{is divergent}                   & \text{otherwise}
	\end{cases}
$$

We use this as a benchmark to check for convergence of an arbitrary
power series:
$$
	\sum_{k=0}^\infty b_kx^k
$$
where $b_1,b_2,\ldots\in\R$.

If for all $k\in\N_0$ we have $|b_k|<a^k$, and the geometric series
converges for $x$, then the arbitrary series converges too.
% TODO: prove why |bₖ|<aᵏ implies convergence of the arbitrary series

This is the same as requiring $(b_k)^{1/k}<a$ and $|x|<\frac1{|a|}$,
and hence we obtain the radius of convergence of the arbitrary power
series:
$$
	R:=\frac{1}{\displaystyle\lim_{k\to\infty}(b_k)^{1/k}}
$$

The series converges if $|x|<R$, and diverges if $|x|>R$.

\Proposition{3.1.5}{Power series at zero}

Suppose we have an arbitrary power series that is zero
$$
	\sum_{k=0}^na_kx^k = 0
$$

Then we must have
$$
a_k=0\quad\forall(k\in\N_0)
$$

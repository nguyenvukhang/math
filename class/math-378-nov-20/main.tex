With $I:=\iter1m$ and $J:=\iter1p$, the Standard NLP is
\begin{equation*}
	\begin{array}{l l l l}
		\displaystyle \min_{x\in\R^n}f(x)
		 & \text{s.t.} & g_i(x)\leq0 & \forall i\in I \\
		 &             & h_j(x)=0    & \forall j\in J
	\end{array}\tag*{(5.1)}
\end{equation*}

We define the active set $I(\bar x)$ as
$$
	I(\bar x):=\Set{i\in I}{g_i(\bar x)=0}
$$

Let $X$ be the feasible set of the NLP.

TANGENT CONE:
$$
	T_X(\bar x):=
	\Set{d\in\R^n}{\exists \{x^k\in X\}\to\bar x,\{t_k\}\downarrow0:\frac{x^k-\bar x}{t_k}\to d}
$$

LINEARIZED CONE:
$$
	L_X(\bar x):=\left\{ d\in\R^n\ \middle\vert
	\begin{array}{l l}
		\nabla g_i(\bar x)^Td\leq0 & \forall i\in I(\bar x) \\
		\nabla h_j(\bar x)^Td=0    & \forall j\in J
	\end{array}
	\right\}
$$

\Definition{5.1.22}{}

We say that the affine constraint qualification (Affine CQ) holds for
(5.1) if all constraints are affine. That is, there exists
$$
	\begin{array}{l l l}
		a_i\in\R^n & \alpha_i\in\R & \forall i\in I \\
		b_i\in\R^n & \beta_j\in\R  & \forall j\in J
	\end{array}
$$
such that for all $x\in\R^n$,
$$
	\begin{array}{l l}
		g_i(x)=a_i^T-\alpha_i & \forall i\in I \\
		h_j(x)=b_i^T-\beta_j  & \forall j\in J
	\end{array}
$$

\Proposition{5.1.23}{}

Let the Affine CQ hold for $(5.1)$. Then ACQ holds at every feasible
point.

\begin{proof}
	\def\LC{L_X(\bar x)}
	\def\TC{T_X(\bar x)}
	\def\Active{I(\bar x)}

	Let $\bar x\in X$. In view of Lemma 5.1.18 we only have to show that
	$\LC\subset\TC$. Let $d\in\LC$. Then by defn of the linearized cone
	we have
	$$
		\begin{array}{l l}
			\alpha_i^Td\leq0 & \forall i\in I \\
			\beta_j^Td=0     & \forall j\in J
		\end{array}
	$$
	Now, let $\{t_k\}\downarrow0$ and put $x^k:=\bar x+t_kd$. Then
	$$
		x^k\to\bar x \Quad\text{and}\Quad \frac{x^k-\bar x}{t_k}\to d
	$$
	Hence, we still need to show that $x^k\in X$ ($\forall k$ suff.
	large)

	\paragraph{For $i\notin\Active$} Then $a_i^T\bar x<\alpha_i$, hence
	by continuity
	$$
		a_i^Tx^k<\alpha_i\Quad\forall(x\text{ suff. large})
	$$

	\paragraph{For $i\in\Active$} Then
	$$
		a_i^Tx^k = a_i^T\bar x + t_ka_i^Td \leq a_i^T\bar x = \alpha_i
	$$
	because $a_i^Td\leq0$ by definition of $\LC$.

	\paragraph{For $j\in J$} Then
	$$
		b_j^T x^k=b_j^T\bar x+t_kb_j^Td=\beta_j
	$$
	because $b_j^Td=0$ by definition of $\LC$.

	These three together show that $x^k\in X$ for $k$ sufficiently
	large. This completes the proof.
\end{proof}

\textbf{5.1.5 CONVEX PROBLEMS}

Consider
\begin{equation*}
	\begin{array}{l l l l}
		\min f(x) & \text{s.t.} & g_i(x)\leq0             & \forall i\in I \\
		          &             & h_j(x)=b_j^Tx-\beta_j=0 & \forall j\in J
	\end{array}
	\tag*{(5.2)}
\end{equation*}
where $f,g_i,h_j\in\R^n\to\R$ are cont. diff \textbf{and convex}, and
$b_j\in\R^n,\beta_j\in\R$. Then
$$
	X=\Set{x\in\R^n}{
		\begin{array}{l l}
			g_i(x)\leq0 & \forall i\in I \\
			h_j(x)=0    & \forall j\in J
		\end{array}
	}
$$
is convex (see Midterms).

\Theorem{5.1.24}{}

Let $\bar x$ be feasible for (5.2), and consider the following statements:
\begin{enumerata}
	\item There exists $(\bar\lambda,\bar\mu)\in M(\bar x)$
	\item $\bar x$ is a global minimizer of (5.2)
\end{enumerata}

Then (a) implies (b). Hence, if a CQ holds at $\bar x$, then (a) iff
(b).


% cl A is the set of all {d|exists seq x in A s.t. x goes to d}
% yes. that is ANY CQ.
% A CQ, by definition, is a condition that makes the KKT a
% necessary optimality condition.
%
% A CQ guarantees that the local minimizer is a KKT point.
%
% NB: make sure the "necessary" and "sufficient" are correctly placed.

\begin{proof}
	\def\bx{\bar x}\def\bl{\bar\lambda}\def\bm{\bar\mu}
	\def\Act{I(\bx)}
	Let $(\bx,\bl,\bm)$ be a KKT point of (5.2), and let $\bar x\in X$.

	Then, by Theorem 2.2.1,
	$$
		f(x)\geq f(\bx) + \nabla f(\bx)^T(x-\bx)
	$$

	But with the KKT conditions, we can rewrite $\nabla f(\bx)$ as
	$$
		\nabla f(\bx)=-\sum_{i\in\Act}^m\bl_i \nabla g_i(\bx)-\sum_{j=1}^p\bm_j\nabla h_j(\bx)
	$$
	So then
	\begin{align*}
		f(x)\geq f(\bx)-\sum_{i\in\Act}^m\bl_i\nabla g_i(\bx)^T(x-\bx)-\sum_{j=1}^p\bm_j\nabla h_j(\bx)^T(x-\bx)\Tag{*}
	\end{align*}

	Again by Theorem 2.2.1, we have
	$$
		\nabla g_i(\bx)^T(x-\bx)\leq g_i(x)-g_i(\bx)
	$$
	Separately,
	$$
		\nabla h_j(\bx)^T(x-\bx)=0
	$$

	Putting everything back to $(*)$ and noting that $g_i(\bar x)=0$ by
	definition of an active set,
	$$
		f(x)\geq f(\bx)-\sum_{i\in\Act}^m\bl_ig_i(x)
	$$
	But since $\bl_i\geq0$ and $g_i(x)\leq0$, we have
	$$
		f(x)\geq f(\bx)
	$$

	and hence $\bar x$ is a global minimum of (5.2). Hence shown that
	(a) implies (b). The converse direction is the definition of a CQ.

	And hence (a) iff (b).
\end{proof}

\Definition{5.1.25}{Slater constraint qualification}

We say that Slater CQ holds for (5.2) if there exists $\hat x$ such that
$$
	\begin{array}{l l}
		g_i(\hat x)<0 & \forall i\in I \\
		h_j(\hat x)=0 & \forall j\in J
	\end{array}
$$

We call such an $\hat x$ a Slater point.

\Proposition{5.1.26}{}

Let SCQ hold for (5.2). Then ACQ holds at every feasible point.

\begin{proof}
	\def\bx{\bar x}\def\bl{\bar\lambda}\def\bm{\bar\mu}
	\def\LC{L_X(\bar x)}
	\def\TC{T_X(\bar x)}
	\def\Act{I(\bx)}

	Let $\bx\in X$ and set
	$$
		F(\bx):=\Set{d\in\R^n}{
			\begin{array}{l l}
				\nabla g_i(\bx)^T\leq0 & \forall i\in\Act \\
				b_j^Td=0               & \forall j\in J
			\end{array}
		}
	$$

	\Lemma{}{$F(\bx)\subset T_X(\bx)$} Let $d\in F(\bx)$, take
	$\{t_k\}\downarrow0$. Set $x^k:=\bar x+t_kd$. Then $\dfrac{x^k-\bar
			x}{t_k}\to d$. Moreover,

	For $i\notin\Act$, $g_i(x^k)<0$ for $k$ sufficiently large.

	For $i\in\Act$, $\dfrac{g_i(x^k)-g_i(\bx)}{t_k}\to\nabla
		g_i(\bx)<0$, and $g_i(\bx)=0$, so we ahve $g_i(x^k)<0$ for $k$
	sufficiently large.

	For $j\in J$, $h_j(x^k)=b_j^Tx^k-\beta_j=t_k\nabla h_j(\bx)^Td-\beta_j$
	\begin{align*}
		h_j(x^k) & =b_j^Tx^k-\beta_j                                 \\
		         & =t_k\nabla h_j(\bx)^Td+\nabla h_j(\bx)\beta_j = 0
		% !!! TODO: fix the line above! this was not read
		% properly in class.
	\end{align*}

	Thus
	$$
		cl F(\bx)\subset cl\TC=\TC
	$$

	We now show $\LC\subset dF(\bx)$. To this end, let $d\in\LC$, and
	let $\hat x$ be a Slater point.

	Set $\hat d$ to be $\hat x-\bar x$. Then by Theorem 2.2.1,
	\begin{equation*}
		\nabla g_i(\bx)^T\hat d\leq g_i(\hat x) - g_i(\bx)<0\Quad\forall(i\in\Act)\Tag{*}
	\end{equation*}
	$<0$ because of definitions of Slater and Active Set.

	Moreover, by the affine-ness of $h_j$, we have
	\begin{align*}
		\nabla h_j(\bx)^T\hat d = h_j(\hat x)-h_j(\bx)=0\Quad\forall(j\in J)\Tag{**}
	\end{align*}

	because both $h_j(\hat x)$ and $h_j(\bx)$ are zero.

	Now we take a small pertubation of $d$ using $\hat d$:
	$$
		d(\delta):= d + \delta\hat d\Quad(\delta>0)
	$$
	Then $d(\delta)\in F(\bx)$, since
	\begin{align*}
		\nabla g_i(\bx)^Td(\delta) & =\nabla g_i(\bx)^Td + \delta\nabla g_i(\bx)^T\hat d<0 \Quad\forall(i\in\Act) \\
		\nabla h_j(\bx)^Td(\delta) & =\nabla h_j(\bx)^Td + \delta\nabla h_j(\bx)^T\hat d=0 \Quad\forall(j\in J)
	\end{align*}
	$<0$ because the 1st term $\leq0$ and 2nd term $<0$, because $d$ is
	in the linearized cone, and $\hat d$...? \\
	And $=0$ because both terms $=0$.

	Hence,
	$$
		d=\lim_{\delta\downarrow0}d(\delta) \subset clF(\bx)
	$$
\end{proof}
% \begin{align*}
% \end{align*}

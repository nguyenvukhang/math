\section{Nonlinear Optimization, Part II: Constrained Optimization}\label{fc7da02}

\def\pd{\succ} % positive definite
\def\psd{\succeq} % positive semidefinite
\def\NlpStdForm{\begin{array}{l l r l}
    \displaystyle\min_{x\in\R^n}f(x)
     & \text{s.t.} & g_i(x)\leq0 & \forall i\in I \\
     &             & h_j(x)=0    & \forall j\in J
  \end{array}}
\def\NlpStdCond{\begin{array}{r l}
    g_i(x)\leq0 & \forall i\in I \\
    h_j(x)=0    & \forall j\in J
  \end{array}}

\Definition{5.0.0}{Standard Nonlinear Program}\label{aeac31f}

\begin{equation*}
  \NlpStdForm\Tag{5.1}\label{a85283d}
\end{equation*}
Where $f,g_i,h_j:\R^n\to\R$ are continuously differentiable.

We call \ref{a85283d} a nonlinear program (NLP) in standard form.

By convention, we let the feasible set of \ref{a85283d} be denoted by $X$, with
\begin{equation*}
  X:=\Set{x\in\R^n}{\begin{array}{r l}
      g_i(x)\leq0 & \forall i\in I \\
      h_j(x)=0    & \forall j\in J
    \end{array}\Tag{5.2}}\label{ea8fbff}
\end{equation*}

By the continuity of the constraint functions, $X$ is closed.

We will use $I:=\{\iter1m\}$ and $J:=\{\iter1p\}$, and define the
\textbf{active set} at $\bar x\in X$ as
$$
  I(\bar x):=\Set{i\in I}{g_i(\bar x)=0}
$$

\Definition{5.1.1}{Cones}\label{ced12c7}

A non-empty set $K\subset\R^n$ is said to be a cone if
$$
  \lambda v\in K\with(\lambda\geq0,\ v\in K)
$$

i.e. $K$ is a cone if and only if it is closed under multiplication with
non-negative scalars.

\Example{5.1.2}{Examples of cones}\label{c36113c}

\begin{enumerata}
  \item \textit{(Non-negative Orthant)} For all $n\in\N$, the
  non-negative orthan $\R^n_+$ is a convex cone, which is also a polyhedron as
  $$
    \R^n_+=\Set{x\in\R^n}{(-e_i)^Tx\leq 0,\ \forall i=\iter1n}
  $$
  \item \textit{(Cone complimentary constraints)} Let $K\subset \R^n$ be
  a cone. Then the set
  $$
    \Lambda:=\Set{(x,y)\in\R^n\times\R^n}{x,y\in K,\ \inner xy=0}
  $$
  is a cone. A prominent example is $K=\R^n$, in which case $\Lambda$
  is called the complementary constraint set.
  \item \textit{(Positive semidefinite matrices)} For $n\in\N$, the set
  of positive semidefinite $n\times n$ matrices is a convex cone in
  the space of $n\times n$ symmetric matrices.
\end{enumerata}

\Definition{5.1.3}{Tangent cone}\label{add7a4b}

Let $S\subset\R^n$ and $\bar x\in S$. Then the set
$$
  T_S(\bar x):=\Set{d\in\R^n}
  {\exists\{x^k\in S\}\to\bar x,\{t_k\}\downarrow0:\frac{x^k-\bar x}{t_k}\to d}
$$
is called the (Bouligand) tangent cone of $S$ at $\bar x$.

\Proposition{5.1.4}{}\label{fe814cc}

Let $S\subset\R^n$ and $x\in S$. Then $T_S(x)$ is a closed cone.

\Theorem{5.1.5}{Basic first-order optimality condition}\label{c8e5836}

Let $\bar x$ be a local minimizer of $f\in C^1$ over $S$. Then the following
hold:
\begin{enumerata}
  \item $\nabla f(\bar x)^Td\geq0\quad(d\in T_S(\bar x))$
  \item If $S$ is convex then
  $$
    \nabla f(\bar x)^T(x-\bar x)\geq 0\with(x\in S)
  $$
\end{enumerata}

\Definition{5.1.6}{Projection on a set}\label{fb41457}

Let $S\subset\R^n$ be non-empty and $x\in\R^n$. Then we define the projection
of $x$ on $S$ by
$$
  P_S(x):=\argmin_{y\in S}\norm{x-y}
$$

\Proposition{5.1.7}{Projection on a closed convex set}\label{ce30ae7}

Let $x\in\R^n$ and $S\subset\R^n$ be a non-empty, closed, and convex set. Then
the following hold:
\begin{enumerata}
  \item $P_S(x)$ has exactly one element, i.e. $P$ is a function
  $\R^n\to S$
  \item $P_S(x)=x$ if and only if $x\in S$
  \item $(P_S(x)-x)^T(y-P_S(x))\geq0\with(y\in S)$
\end{enumerata}

\begin{proof}
  (a) follows immediately from \href{f546fc9}{Theorem 2.2.7(d)} since
  the function $y\mapsto\frac12\norm{x-y}^2$ is strongly convex.

  (b) is obvious.

  (c) follows from \href{c8e5836}{Theorem 5.1.5(b)} applied to
  $f:y\mapsto\frac12\norm{x-y}^2$.
\end{proof}

\Lemma{5.1.8}{}\label{d2dff14}

Let $B\in\R^{l\times n}$. Then
$$
  K:=\Set{B^Tx}{x\geq0}
$$

is a (non-empty) closed, convex cone.

\Theorem{5.1.9}{Farkas' Lemma}\label{d64b0db}

Let $B\in\R^{l\times n}$ and $h\in\R^n$. Then the system
$$
  B^Tx=h\with(x\in\R^l,\ x\geq0)
$$

has a solution if and only if $h^Td\geq0$ for all $d\in\R^n$ such that
$Bd\geq0$.

\begin{proof}
  Proving that (1) implies (2):

  Let $x\geq0$ such that $B^Tx=h$. Then for any $d$ such that $Bd\geq0$, we have
  $$
    h^Td=(B^Tx)^Td=x^TBd
  $$

  But $x^TBd\geq0$ because $x\geq0$ and $Bd\geq0$.

  Proving that (2) implies (1) by contrapositive.

  Assume that (1) is false. Then
  $$
    h\notin\{B^Tx \mid x\geq0\}=:K
  $$

  By \href{d2dff14}{Lemma 5.1.8}, $K$ is a closed convex cone.

  Set $\bar s:=P_K(h)$ and $\bar d:=\bar s-h$. Note that $\bar s\in K$, and
  $h\notin K$, and hence $\bar d\neq0$.

  By \href{ce30ae7}{Proposition 5.1.7(c)},
  \begin{equation*}
    \bar d^T(s-\bar s)\geq0\quad\forall(s\in K)\Tag{*}
  \end{equation*}

  Substituting $s:=0$ and $s:=2\bar s$, we obtain two simultaneous inequalities
  \begin{align*}
    \bar d^T\bar s\leq0\Quad\text{and}\Quad\bar d^T\bar s\geq0
  \end{align*}

  And hence $\bar d^T\bar s=0$. Using this with $(*)$ gives
  $$
    \bar d^Ts\geq0
  $$

  Then by definition of cone $K$, for all $x\geq0$,
  \begin{align*}
    \bar d^T B^Tx     &\geq0 \\
    \then(B\bar d)^Tx &\geq0
  \end{align*}

  Inserting $x:=e_i$ (where $e_i$ is the $i^\text{th}$ component vector) for
  $i=\iter1n$ implies $(B\bar d)^T\geq0$.

  On the other hand (recall $\bar d^T\bar s=0$ from above)
  \begin{align*}
    h^T\bar d &=(\bar s-\bar d)^T\bar d        \\
              &=\bar s^T\bar d-\norm{\bar d}^2 \\
              &=-\norm{\bar d}^2               \\
              &\leq 0
  \end{align*}

  But since $\bar d\neq0$, we have the strict inequality $h^T\bar d<0$.

  Therefore, $B\bar d\geq0$, but $h^T\bar d<0$, i.e. (2) does not hold.
\end{proof}

\Definition{5.1.10}{Karush-Kuhn-Tucker conditions}\label{b38093d}

Consider the standard NLP in \ref{a85283d}. and let $X$ be the feasible set of
\ref{a85283d}.

\begin{enumerate}
  \item The function $L:\R^n\times\R^m\times\R^p\to\R$ defined by
        \begin{align*}
          L(x,\lambda,\mu)
           &=f(x)+\sum_{i=1}^m\lambda_ig_i(x)+\sum_{j=1}^p\mu_jh_j(x) \\
           &=f(x)+\lambda^Tg(x)+\mu^Th(x)
        \end{align*}
        is called the Lagrangian (function) of \ref{a85283d}.
  \item The set of conditions
        $$
          \begin{array}{l l}
            \nabla_xL(x,\lambda,\mu)                 & =0, \\[0.2em]
            h(x)                                     & =0, \\[0.2em]
            \lambda\geq0,\ g(x)\leq0,\ \lambda^Tg(x) & =0
          \end{array}
        $$
        are called the Karush-Kuhn-Tucker conditions for \ref{a85283d}, where
        \begin{align*}
          \nabla_xL(x,\lambda,\mu)
           &= \nabla f(x)+\sum_{i=1}^m\lambda_i\nabla g_i(x)+\sum_{j=1}^p\mu_j\nabla h_j(x) \\
           &= \nabla f(x)+\lambda^T\nabla g(x)+\mu^T\nabla h(x)
        \end{align*}
  \item A triple $(\bar x,\bar\lambda,\bar\mu)$ that satisfies the KKT conditions is
        called a KKT point.
  \item Given $\bar x$, a feasible point for \ref{a85283d}, we define
        $$
          M(\bar x):=\{(\lambda,\mu)\mid(\bar x,\lambda,\mu)\text{ is a KKT point of \ref{a85283d}}\}
        $$
        as the set of all KKT multipliers (possibly empty) at $\bar x$.
\end{enumerate}

\Definition{5.1.10a}{Constraint qualification (CQ)}\label{d533941}

We define a condition about the feasible set $X$ of a standard NLP
\ref{a85283d} that guarantees that the \href{b38093d}{KKT conditions} hold at a
local minimizer as a constraint qualification.

If a CQ holds on $\bar x\in X$, then KKT is necessary for $\bar x$ to be a
local minimizer.

If a CQ holds on $\bar x\in X$, then $\bar x$ being a local minimizer implies
that there exists a $(\bar\lambda,\bar\mu)$ such that $(\bar
x,\bar\lambda,\bar\mu)$ is a KKT point.

% A CQ, by definition, is a condition that makes the KKT a
% necessary optimality condition.
%
% A CQ guarantees that the local minimizer is a KKT point.

\Definition{5.1.11}{Linearized cone}\label{ca4f471}

Let $X$ be the feasible set of \ref{a85283d}. The linearized cone (of $X$) at
$\bar x\in X$ is defined by
$$
  L_X(\bar x):=\left\{ d\in\R^n\ \middle\vert
  \begin{array}{r l}
    \nabla g_i(\bar x)^Td\leq0 & \forall i=I(\bar x) \\
    \nabla h_j(\bar x)^Td=0    & \forall j=\iter1p
  \end{array}
  \right\}
$$

\Definition{5.1.12}{Abadie constraint qualification (ACQ)}\label{adc266e}

We say that the ACQ holds at $\bar x\in X$ if
$$
  T_X(\bar x)=L_X(\bar x)
$$

That is, the \href{add7a4b}{tangent cone} is exactly the
\href{ca4f471}{linearized cone}.

\Theorem{5.1.13}{\href{b38093d}{KKT} conditions under \href{adc266e}{ACQ}}\label{b1c5437}

Let $\bar x\in X$ be a local minimizer of \ref{a85283d} such that
\href{adc266e}{ACQ} holds at $\bar x$. Then there exists
$(\bar\lambda,\bar\mu)\in\R^m\times\R^p$ such that $(\bar
x,\bar\lambda,\bar\mu)$ is a \href{b38093d}{KKT} point of \ref{a85283d}.

\begin{proof}
  \def\bm{\bar\mu}\def\bl{\bar\lambda}\def\bx{\bar x}
  By \href{c8e5836}{Theorem 5.1.5},
  \begin{equation*}
    \nabla f(\bx)^Td\geq0\quad\forall(d\in T_X(\bx))\Tag{*}
  \end{equation*}

  Set
  $$
    B:=\begin{pmatrix}
      -\nabla g_i(\bx)^T & (i=\iter1m) \\
      -\nabla h_j(\bx)^T & (j=\iter1p) \\
      \nabla h_j(\bx)^T  & (j=\iter1p)
    \end{pmatrix}\in\R^{(m+2p)\times{n}}
  $$

  Then purely from the definition of a \href{ca4f471}{linearized cone} and how we
  set $B$, we have
  $$
    d\in L_X(\bx)\iff Bd\geq0
  $$

  By \href{adc266e}{ACQ}, we have $d\in T_X(\bx)\iff Bd\geq0$

  Combined with $(*)$, we have
  $$
    \nabla f(\bx)^Td\geq0\quad\forall(d:Bd\geq0)
  $$

  (Think $h=\nabla f(\bx)$ and apply the Farkas Lemma.)

  By the Farkas Lemma,
  $$
    \exists y=
    \begin{pmatrix}
      y^1\in\R^m \\y^2\in\R^p\\y^3\in\R^p
    \end{pmatrix}
  $$

  such that $y\geq0$, and $B^Ty=\nabla f(\bx)$

  Define $\bl\in\R^n,\bm\in\R^p$ by
  $$
    \bl_i=\begin{cases}
      y^1_i & \text{if $i=\iter1m$} \\
      0     & \text{otherwise}
    \end{cases}
  $$
  and
  $$
    \bm_i=\begin{cases}
      y^2_j - y^3_j & \text{if $j=\iter{m+1}{m+2p}$} \\
      0             & \text{otherwise}
    \end{cases}
  $$

  Then $(\bx,\bl,\bm)$ is a KKT point.

  MORE NOTES

  \begin{align*}
    0
     &= \nabla f(\bx)
    +\sum_{i=0}^my^1_i\nabla g_i(\bx)
    +\sum_{j=m+1}^{m+2p}(y^2_j-y^3_j)\nabla h_j(\bx) \\
     &= \nabla f(\bx)
    +\sum_{i=0}^m \bl_i\nabla g_i(\bx)
    +\sum_{j=m+1}^{m+2p}\bm_j\nabla h_j(\bx)
  \end{align*}

  and then there is a line with a tick/check next to it:
  $$
    \bl^Tg(\bx)=\sum_{i=0}^m\bl_ig_i(\bx)=0
  $$
\end{proof}

\Definition{5.1.14}{Constraint qualifications}\label{e8fa554}

A condition on $X$ (i.e. on $g$ and $h$) that ensures that the KKT conditions
hold at a local minimizer is called a \textbf{constraint qualification}.

\Definition{5.1.15}{LICQ and MFCQ}\label{fed784a}

Let $\bar x$ be feasible for (1). We say that
\begin{enumerata}
  \item \textbf{(LICQ)} the linear independence constraint
  qualification holds at $\bar x$ if the gradients
  \begin{align*}
    \nabla g_i(\bar x) &\quad(i\in I(\bar x)), \\
    \nabla h_j(\bar x) &\quad(j\in J)
  \end{align*}
  are linearly indepdendent.
  \item \textbf{(MFCQ)} the Mangasarian-Fromovitz constraint
  qualification holds at $\bar x$ if the gradients
  $$
    \nabla h_j(\bar x)\quad(j\in J)
  $$
  are linearly indepdendent, and $\exists d\in\R^n$ such that
  \begin{align*}
    \nabla g_i(\bar x)^Td &<0\quad(i\in I(\bar x)) \\
    \nabla h_j(\bar x)^Td &=0\quad(j\in J)
  \end{align*}
\end{enumerata}

\Proposition{5.1.16}{LICQ implies MFCQ}\label{a7ef3f5}

Let $\bar x$ be feasible for (1) such that \href{fed784a}{LICQ} holds at $\bar
x$. Then \href{fed784a}{MFCQ} holds.

% \Lemma{5.1.17}{}\label{a9bea04}

% Let $\bar x$ be feasible for (1) such that there exists $d\in\R^n$
% such that
% $$
% 	\nabla h_j(\bar x)\quad\forall(j\in J)
% $$
% are linearly indepdendent, and
% \begin{gather*}
% 	\nabla g_i(\bar x)^T < 0  \quad\forall(i\in I(\bar x))\\
% 	\nabla h_j(\bar x)^Td = 0 \quad\forall(j\in J)
% \end{gather*}

% Then there exists $\epsilon>0$ and a $C^1$-curve
% $\gamma:(-\epsilon,\epsilon)\to\R^n$ such that
% \begin{gather*}
% 	\gamma(t) \in X\quad\forall(t\in(-\epsilon,\epsilon)) \\
% 	\gamma(0) = \bar x \\
% 	\gamma'(0)=d
% \end{gather*}

% \Lemma{5.1.18}{}\label{a08cd35}

% Let $\bar x$ be feasible for (1). Then
% $$
% 	T_X(\bar x)\subset L_X(\bar x)
% $$

% \begin{proof}
% 	Exercise 9.3
% \end{proof}

% \Proposition{5.1.19}{MFCQ $\implies$ ACQ}\label{eddce03}

% Let $\bar x$ be feasible for (1) such that MFCQ holds at $\bar x$.
% Then ACQ holds at $\bar x$.

% \begin{proof}
% 	Only need to show $L_X(\bar x)\subset T_X(\bar x)$.

% 	Let $d\in L_X(\bar x)$. By MFCQ, there exists $\hat d$ such that
% 	% Note: we can't use d directly instead of d(∂) because d doesn't
% 	% have the strict inequality required by Lemma 5.1.17
% 	\begin{gather*}
% 		\nabla g_i(\bar x)^T\hat d<0\quad\forall(i\in I(\bar x)) \\
% 		\nabla h_j(\bar x)^T\hat d=0\quad\forall(j\in J)
% 	\end{gather*}
% 	and $\nabla h_j(\bar x)\ \forall(j\in J)$ are linearly indepdendent.

% 	Set $d(\delta):= d + \delta\hat d$. Then, $\forall\delta>0$:
% 	\begin{gather*}
% 		\nabla g_i(\bar x)^T{d(\delta)}<0\quad\forall(i\in I(\bar x)) \\
% 		\nabla h_j(\bar x)^T{d(\delta)}=0\quad\forall(j\in J)
% 	\end{gather*}

% 	Applying Lemma 5.1.17 to $d(\delta)$ yields a $C^1$-curve
% 	$\gamma:(-\epsilon,\epsilon)\to\R^n$ such that
% 	\begin{gather*}
% 		\gamma(t) \in X\quad\forall(t\in(-\epsilon,\epsilon)) \\
% 		\gamma(0) = \bar x \\
% 		\gamma'(0)=d(\delta)
% 	\end{gather*}

% 	Let $\{t_k\}\downarrow0$ and set $x^k:=\gamma(t_k)$. Then
% 	$\{x^k\in X\}\to\bar x$.

% 	And $d(\delta)=\gamma'(0)$ and hence
% 	$$
% 		d(\delta)=\lim_{k\to\infty}\frac{\gamma(t_k)-\gamma(0)}{t_k}=
% 		\lim_{k\to\infty}\frac{x^k-\bar x}{t_k}
% 	$$

% 	And since
% 	$$
% 		\lim_{\delta\to0} d(\delta) = d
% 	$$
% 	and both lie in $T_X(\bar x)$ due to an unproved argument using
% 	closedness.

% 	this completes the proof with
% 	$$
% 		\lim_{k\to\infty}\frac{x^k-\bar x}{t_k}=d
% 	$$
% \end{proof}

% \Corollary{3.1.20}{KKT under MFCQ}\label{b39f0fe}

% Let $\bar x$ be a local minimum of (1) such that MFCQ holds at $\bar x$. Then:

% \begin{enumerata}
% 	\item There exists $(\bar\lambda, \bar\mu)\in\R^m\times\R^p$  such
% 	that $(\bar x,\bar\lambda, \bar\mu)$ is a KKT point of (1).
% 	\item $M(\bar x)$ is bounded where
% 	$$
% 		M(\bar x):=\{(\lambda,\mu)\mid(\bar x,\lambda,\mu)\text{ is a KKT point of (1)} \}
% 	$$
% \end{enumerata}

% \begin{proof}
% 	Proving (a) requires Proposition 5.1.19 (MFCQ $\implies$ ACQ) and
% 	Theorem 5.1.13

% 	Proving (b):

% 	Suppose $M(\bar x)$ were unbounded, i.e. there exists
% 	$$
% 		\{(\lambda^k,\mu^k)\in M(\bar x)\} : \norm{(\lambda^k,\mu^k)}\to+\infty
% 	$$

% 	Then WLOG,
% 	$$
% 		\frac{(\lambda^k,\mu^k)}{\norm{(\lambda^k,\mu^k)}}\to(\tilde\lambda,\tilde\mu)
% 	$$

% 	Note that every element in the above sequence has norm 1. And hence
% 	we know that it doesn't converge to the zero vector, since that has
% 	norm 0.

% 	Since $(\bar x,\lambda^k,\mu^k)$ is a KKT point of (1), we have
% 	$$
% 		0 = \frac{\nabla f(\bar x) + \sum_{i\in I(\bar x)}\lambda^k_i\nabla g_i(\bar x)
% 			+ \sum_{j\in J}\mu^k_j\nabla h_j(\bar x)}{\norm{(\lambda^k,\mu^k)}}
% 	$$

% 	Then as $h\to\infty$,

% 	\begin{equation*}
% 		0=\sum_{i\in I(\bar x)}\tilde\lambda_i\nabla g_i(\bar x)
% 		+ \sum_{j\in J}\tilde\mu_j\nabla h_j(\bar x)
% 	\end{equation*} % cognitive, but just raw algebra

% 	% Note: taking ||.|| of a product space (X × Y)
% 	% ||(a,b)|| = ||a||+||b||
% \end{proof}

% Now multiply with $d$ from MFCQ at $\bar x$.
% \begin{equation*}
% 	0=\sum_{i\in I(\bar x)}\tilde\lambda_i\nabla g_i(\bar x)^Td
% 	+ \sum_{j\in J}\tilde\mu_j\nabla h_j(\bar x)^Td \Tag{*}
% \end{equation*}

% And by MFCQ, the second term is $=0$. Hence
% $$
% 	0=\sum_{i\in I(\bar x)}\tilde\lambda_i\nabla g_i(\bar x)^Td
% $$

% \textbf{Case 1} $\exists i_0 \in I(\bar x): \tilde\lambda_{i_0}>0$. Then
% $0<0$. Contradiction!
% $$
% 	0=\sum_{i\in I(\bar x)}\tilde\lambda_i\nabla g_i(\bar x)^Td
% 	\;\leq\;\tilde\lambda_{i_0}\nabla g_{i_0}(\bar x)^Td<0
% $$
% \textbf{Case 2} $\forall i\in I(\bar x):\tilde\lambda_i=0$. Then $(*)$ reads
% $$
% 	0=\sum_{j\in J}\tilde\mu_j\nabla h_j(\bar x)^Td
% $$
% which is a contradiction against MFCQ.

% \Corollary{5.1.21}{}\label{c55a5f3}

% Let $\bar x$ be a local minimum of (1) such that LICQ holds at $\bar
% 	x$. Then:
% \begin{enumerata}
% 	\item There exists $(\bar\lambda,\bar\mu)\in M(\bar x)$.
% 	\item $M(\bar x) = \{(\bar\lambda,\bar\mu)\}$.
% \end{enumerata}

% \begin{proof}
% 	(a) follows from Proposition 5.1.16 + Corollary 5.1.19

% 	(b)

% 	Assume that $(\tilde\lambda,\tilde\mu)\in M(\bar x)$. Then
% 	\begin{align*}
% 		0          & =\nabla f(\bar x)+\sum_{i\in I(\bar x)}\tilde\lambda_i\nabla g_i(\bar x)+\sum_{j\in J}\tilde\mu_j\nabla h_j(\bar x)                             \\
% 		           & =\nabla f(\bar x)+\sum_{i\in I(\bar x)}\bar\lambda_i\nabla g_i(\bar x)+\sum_{j\in J}\bar\mu_j\nabla h_j(\bar x)                                 \\
% 		\implies 0 & =\nabla f(\bar x)+\sum_{i\in I(\bar x)}(\tilde\lambda_i-\bar\lambda_i)\nabla g_i(\bar x)+\sum_{j\in J}(\tilde\mu_j-\bar\mu_j)\nabla h_j(\bar x)
% 	\end{align*}

% 	Then by LICQ,
% 	\begin{gather*}
% 		\tilde\lambda_i - \bar\lambda_i = 0\quad\forall(i\in I(\bar x)) \\
% 		\tilde\mu_j - \bar\mu_j = 0\quad\forall(j\in J) \\
% 	\end{gather*}

% 	Then since $\tilde\lambda_i = \bar\lambda_i\ \forall(i\notin I(\bar
% 		x))$, this shows $\tilde\lambda=\bar\lambda$ and $\tilde\mu=\bar\mu$.
% \end{proof}

% LAGRANGIAN DUALITY

With $I:=\iter1m$ and $J:=\iter1p$, the Standard NLP is
\begin{equation*}
  \begin{array}{l l l l}
    \displaystyle \min_{x\in\R^n}f(x)
     & \text{s.t.} & g_i(x)\leq0 & \forall i\in I \\
     &             & h_j(x)=0    & \forall j\in J
  \end{array}\Tag{5.1}
\end{equation*}

We define the active set $I(\bar x)$ as
$$
  I(\bar x):=\Set{i\in I}{g_i(\bar x)=0}
$$

Let $X$ be the feasible set of the NLP.

TANGENT CONE:
$$
  T_X(\bar x):=
  \Set{d\in\R^n}{\exists \{x^k\in X\}\to\bar x,\{t_k\}\downarrow0:\frac{x^k-\bar x}{t_k}\to d}
$$

LINEARIZED CONE:
$$
  L_X(\bar x):=\left\{ d\in\R^n\ \middle\vert
  \begin{array}{l l}
    \nabla g_i(\bar x)^Td\leq0 & \forall i\in I(\bar x) \\
    \nabla h_j(\bar x)^Td=0    & \forall j\in J
  \end{array}
  \right\}
$$

\Definition{5.1.22}{}\label{e55c332}

We say that the affine constraint qualification (Affine CQ) holds for (5.1) if
all constraints are affine. That is, there exists
$$
  \begin{array}{l l l}
    a_i\in\R^n & \alpha_i\in\R & \forall i\in I \\
    b_i\in\R^n & \beta_j\in\R  & \forall j\in J
  \end{array}
$$
such that for all $x\in\R^n$,
$$
  \begin{array}{l l}
    g_i(x)=a_i^T-\alpha_i & \forall i\in I \\
    h_j(x)=b_i^T-\beta_j  & \forall j\in J
  \end{array}
$$

\Proposition{5.1.23}{}\label{f5d546e}

Let the Affine CQ hold for $(5.1)$. Then ACQ holds at every feasible point.

\begin{proof}
  \def\LC{L_X(\bar x)}
  \def\TC{T_X(\bar x)}
  \def\Active{I(\bar x)}

  Let $\bar x\in X$. In view of Lemma 5.1.18 we only have to show that
  $\LC\subset\TC$. Let $d\in\LC$. Then by defn of the linearized cone we have
  $$
    \begin{array}{l l}
      \alpha_i^Td\leq0 & \forall i\in I \\
      \beta_j^Td=0     & \forall j\in J
    \end{array}
  $$
  Now, let $\{t_k\}\downarrow0$ and put $x^k:=\bar x+t_kd$. Then
  $$
    x^k\to\bar x \Quad\text{and}\Quad \frac{x^k-\bar x}{t_k}\to d
  $$
  Hence, we still need to show that $x^k\in X$ ($\forall k$ suff.
  large)

  \paragraph{For $i\notin\Active$} Then $a_i^T\bar x<\alpha_i$, hence by continuity
  $$
    a_i^Tx^k<\alpha_i\Quad\forall(x\text{ suff. large})
  $$

  \paragraph{For $i\in\Active$} Then
  $$
    a_i^Tx^k = a_i^T\bar x + t_ka_i^Td \leq a_i^T\bar x = \alpha_i
  $$
  because $a_i^Td\leq0$ by definition of $\LC$.

  \paragraph{For $j\in J$} Then
  $$
    b_j^T x^k=b_j^T\bar x+t_kb_j^Td=\beta_j
  $$
  because $b_j^Td=0$ by definition of $\LC$.

  These three together show that $x^k\in X$ for $k$ sufficiently large. This
  completes the proof.
\end{proof}

\textbf{5.1.5 CONVEX PROBLEMS}

Consider
\begin{equation*}
  \begin{array}{l l l l}
    \min f(x) & \text{s.t.} & g_i(x)\leq0             & \forall i\in I \\
              &             & h_j(x)=b_j^Tx-\beta_j=0 & \forall j\in J
  \end{array}
  \Tag{5.2}
\end{equation*}
where $f,g_i,h_j\in\R^n\to\R$ are cont. diff \textbf{and convex}, and
$b_j\in\R^n,\beta_j\in\R$. Then
$$
  X=\Set{x\in\R^n}{
    \begin{array}{l l}
      g_i(x)\leq0 & \forall i\in I \\
      h_j(x)=0    & \forall j\in J
    \end{array}
  }
$$
is convex (see Midterms).

\Theorem{5.1.24}{}\label{d0453f3}

Let $\bar x$ be feasible for (5.2), and consider the following statements:
\begin{enumerata}
  \item There exists $(\bar\lambda,\bar\mu)\in M(\bar x)$
  \item $\bar x$ is a global minimizer of (5.2)
\end{enumerata}

Then (a) implies (b). Hence, if a CQ holds at $\bar x$, then (a) iff (b).

% cl A is the set of all {d|exists seq x in A s.t. x goes to d}
% yes. that is ANY CQ.
% A CQ, by definition, is a condition that makes the KKT a
% necessary optimality condition.
%
% A CQ guarantees that the local minimizer is a KKT point.
%
% NB: make sure the "necessary" and "sufficient" are correctly placed.

\begin{proof}
  \def\bx{\bar x}\def\bl{\bar\lambda}\def\bm{\bar\mu}
  \def\Act{I(\bx)}
  Let $(\bx,\bl,\bm)$ be a KKT point of (5.2), and let $\bar x\in X$.

  Then, by Theorem 2.2.1,
  $$
    f(x)\geq f(\bx) + \nabla f(\bx)^T(x-\bx)
  $$

  But with the KKT conditions, we can rewrite $\nabla f(\bx)$ as
  $$
    \nabla f(\bx)=-\sum_{i\in\Act}^m\bl_i \nabla g_i(\bx)-\sum_{j=1}^p\bm_j\nabla h_j(\bx)
  $$
  So then
  \begin{align*}
    f(x)\geq f(\bx)-\sum_{i\in\Act}^m\bl_i\nabla g_i(\bx)^T(x-\bx)-\sum_{j=1}^p\bm_j\nabla h_j(\bx)^T(x-\bx)\Tag{*}
  \end{align*}

  Again by Theorem 2.2.1, we have
  $$
    \nabla g_i(\bx)^T(x-\bx)\leq g_i(x)-g_i(\bx)
  $$
  Separately,
  $$
    \nabla h_j(\bx)^T(x-\bx)=0
  $$

  Putting everything back to $(*)$ and noting that $g_i(\bar x)=0$ by definition
  of an active set,
  $$
    f(x)\geq f(\bx)-\sum_{i\in\Act}^m\bl_ig_i(x)
  $$
  But since $\bl_i\geq0$ and $g_i(x)\leq0$, we have
  $$
    f(x)\geq f(\bx)
  $$

  and hence $\bar x$ is a global minimum of (5.2). Hence shown that (a) implies
  (b). The converse direction is the definition of a CQ.

  And hence (a) iff (b).
\end{proof}

\Definition{5.1.25}{Slater constraint qualification}\label{a3b15e9}

We say that Slater CQ holds for (5.2) if there exists $\hat x$ such that
$$
  \begin{array}{l l}
    g_i(\hat x)<0 & \forall i\in I \\
    h_j(\hat x)=0 & \forall j\in J
  \end{array}
$$

We call such an $\hat x$ a Slater point.

\Proposition{5.1.26}{}\label{c78f4af}

Let SCQ hold for (5.2). Then ACQ holds at every feasible point.

\begin{proof}
  \def\bx{\bar x}\def\bl{\bar\lambda}\def\bm{\bar\mu}
  \def\LC{L_X(\bar x)}
  \def\TC{T_X(\bar x)}
  \def\Act{I(\bx)}

  Let $\bx\in X$ and set
  $$
    F(\bx):=\Set{d\in\R^n}{
      \begin{array}{l l}
        \nabla g_i(\bx)^T\leq0 & \forall i\in\Act \\
        b_j^Td=0               & \forall j\in J
      \end{array}
    }
  $$

  \Lemma{}{$F(\bx)\subset T_X(\bx)$} Let $d\in F(\bx)$, take
  $\{t_k\}\downarrow0$. Set $x^k:=\bar x+t_kd$. Then $\dfrac{x^k-\bar
    x}{t_k}\to d$. Moreover,

  For $i\notin\Act$, $g_i(x^k)<0$ for $k$ sufficiently large.

  For $i\in\Act$, $\dfrac{g_i(x^k)-g_i(\bx)}{t_k}\to\nabla g_i(\bx)<0$, and
  $g_i(\bx)=0$, so we ahve $g_i(x^k)<0$ for $k$ sufficiently large.

  For $j\in J$, $h_j(x^k)=b_j^Tx^k-\beta_j=t_k\nabla h_j(\bx)^Td-\beta_j$
  \begin{align*}
    h_j(x^k) &=b_j^Tx^k-\beta_j                                 \\
             &=t_k\nabla h_j(\bx)^Td+\nabla h_j(\bx)\beta_j = 0
    % !!! TODO: fix the line above! this was not read
    % properly in class.
  \end{align*}

  Thus
  $$
    cl F(\bx)\subset cl\TC=\TC
  $$

  We now show $\LC\subset dF(\bx)$. To this end, let $d\in\LC$, and let $\hat x$
  be a Slater point.

  Set $\hat d$ to be $\hat x-\bar x$. Then by Theorem 2.2.1,
  \begin{equation*}
    \nabla g_i(\bx)^T\hat d\leq g_i(\hat x) - g_i(\bx)<0\Quad\forall(i\in\Act)\Tag{*}
  \end{equation*}
  $<0$ because of definitions of Slater and Active Set.

  Moreover, by the affine-ness of $h_j$, we have
  \begin{align*}
    \nabla h_j(\bx)^T\hat d = h_j(\hat x)-h_j(\bx)=0\Quad\forall(j\in J)\Tag{**}
  \end{align*}

  because both $h_j(\hat x)$ and $h_j(\bx)$ are zero.

  Now we take a small pertubation of $d$ using $\hat d$:
  $$
    d(\delta):= d + \delta\hat d\Quad(\delta>0)
  $$
  Then $d(\delta)\in F(\bx)$, since
  \begin{align*}
    \nabla g_i(\bx)^Td(\delta) &=\nabla g_i(\bx)^Td + \delta\nabla g_i(\bx)^T\hat d<0 \Quad\forall(i\in\Act) \\
    \nabla h_j(\bx)^Td(\delta) &=\nabla h_j(\bx)^Td + \delta\nabla h_j(\bx)^T\hat d=0 \Quad\forall(j\in J)
  \end{align*}
  $<0$ because the 1st term $\leq0$ and 2nd term $<0$, because $d$ is
  in the linearized cone, and $\hat d$...? \\
  And $=0$ because both terms $=0$.

  Hence,
  $$
    d=\lim_{\delta\downarrow0}d(\delta) \subset clF(\bx)
  $$
\end{proof}

Consider the standard NLP:
\begin{equation*}
  \NlpStdForm\Tag{1}
\end{equation*}
But now assume that $f,g_i,h_j:\R^n\to\R$ with no smoothness.

The Lagrangian of (1) is
\begin{align*}
  L(x,\lambda,\mu)
   &=f(x)+\sum_{i\in I}\lambda_ig_i(x) + \sum_{j\in J}\mu_jh_j(x) \\
   &=f(x)+\lambda^Tg(x)+\mu^Th(x)
\end{align*}

\textbf{The Dual Problem}

Observe that if $x$ is a feasible point,
\begin{align*}
  \sup_{\lambda\in\R^m_+,\ \mu\in\R^p}=
  \begin{cases}
    f(x)    & \text{if }x\in X,   \\
    +\infty & \text{if }x\notin X
  \end{cases}
\end{align*}

% Infinity because we can blow up both Lambda and Mu terms to
% +infinity, and hence sup is infinity.
%
% 0 because Mu is 0 when feasible and Lambda maxes out at 0 when
% feasible.

Therefore the primal problem (1) is equivalent to
$$
  \min_{x\in\R^n}\sup_{\lambda\in\R^m_+,\ \mu\in\R^p}L(x,\lambda,\mu)
$$

\textbf{Q.} When can we switch min(inf) and sup?

\Definition{6.1.1}{Lagrangian dual}\label{f6ab91b}

The Lagrangian dual of (1) is given by
$$
  \begin{array}{l l l l}
    \max_{}d(\lambda,\mu) & \text{s.t.} & \lambda\geq0
  \end{array}
$$

where the dual objective is given by $d:\R^m\times\R^p\to\R\cup\{+\infty\}$ and
$$
  d(\lambda,\mu):=\inf_{x\in\R^n}L(x,\lambda,\mu)
$$

The function $p:\R^n\to\R\cup\{+\infty\}$ given by
$$
  p(x):=\sup_{\lambda\in\R^m_+,\ \mu\in\R^p}
  L(x,\lambda,\mu)
$$

is called the \textbf{primal objective}.

\Example{6.1.2}{LP duality}\label{ab8429b}

Consider the standard linear program (LP):
$$
  \begin{array}{l l l}
    \displaystyle \min_{x\in\R^n}c^Tx
     & \text{s.t.} & Ax=b   \\
     &             & x\geq0
  \end{array}
$$
with $A\in\R^{m\times n}$, $b\in\R^m$, $c\in\R^n$.

The Lagrangian reads
\begin{align*}
  L(x,\lambda,\mu)
   &=c^Tx-\lambda^Tx+\mu^T(b-Ax)         \\
   &=(c-\lambda-A^T\mu)^Tx + b^Tu\Tag{*}
\end{align*}

% so g_i corresponds to -x<=0
% and h_j corresponds to Ax-b=0

So then
\begin{align*}
  \nabla_xL(x,\lambda,\mu)
   &=c-\lambda-A^T\mu
\end{align*}

The function that takes $x\mapsto L(x,\lambda,\mu)$ is affine (from $(*)$), and
in particular it is convex. And hence it takes its minimum if and only if
$\nabla_xL(x,\lambda,\mu)=0$, in which case,
$$
  \inf_{x\in\R^n}L(x,\lambda,\mu)=b^T\mu
$$

otherwise if $\nabla_xL(x,\lambda,\mu)\neq0$, the infimum must be $-\infty$.

So then
$$
  d(\lambda,\mu)=\begin{cases}
    b^T\mu  & \text{if }c=A^Tu+\lambda, \\
    -\infty & \text{otherwise}
  \end{cases}
$$

Therefore, the dual problem reads
$$\begin{array}{l l l}\displaystyle
    \max_{\lambda,\,\mu}d(\lambda,\mu) & \text{s.t.} & \lambda\geq0
  \end{array}$$

Which is the same as
$$\begin{array}{l l l}\displaystyle
    \max_{\lambda,\,\mu}b^T\mu & \text{s.t.} & \lambda\geq0,A^T\mu+\lambda=c
  \end{array}$$

and again,
$$\begin{array}{l l l}\displaystyle
    \max_\mu b^T\mu & \text{s.t.} & A^T\mu\leq c
  \end{array}$$

% WEAK AND STRONG DUALITY

\Theorem{6.2.1}{Weak duality}\label{ad0b792}

Let $\hat x$ be feasible for (P) and $(\hat\lambda,\hat\mu)$ be feasible for
(D). Then
$$
  p(\hat x)\geq d(\hat\lambda,\hat\mu)
$$

\begin{proof}
  \def\hx{\hat x}
  \def\hl{\hat\lambda}
  \def\hm{\hat\mu}
  We have
  $$
    p(\hx) = f(\hx)\with(\hx\in X)
  $$

  and hence
  \begin{align*}
    p(\hx) &\geq f(\hx) + \hl^Tg(\hx) + \hm^Th(\hx) \\
           &=L(\hx,\hl,\hm)                         \\
           &\geq\inf_{x\in\R^n}L(x,\hl,\hm)         \\
           &= d(\hl,\hm)
  \end{align*}

  \paragraph{Remark} If $p(\hx)=d(\hl,\hm)$, then $\hx$ solves (P), and $(\hl,\hm)$ solves (D).
\end{proof}

From weak duality, if we define
\begin{align*}
  \bar p:=\inf_{x\in\R^n}p(x)\geq
  \sup_{\lambda\in\R^m_+,\ \mu\in\R^p}d(\lambda,\mu)
  =:\bar d
\end{align*}

Then $\bar p-\bar d\geq0$

\Example{6.2.2}{Non-zero duality gap}\label{c4aaa1f}

% Strong duality in LP but we don't have that in NLP

Consider the following objective function:
\begin{align*}
  \min f(x):=\begin{cases}
               x^2-2x & x\geq0           \\
               x      & \text{otherwise}
             \end{cases}
  \quad \text{s.t.}\quad g(x):=-x\leq0
\end{align*}

% Note that the global minimizer is at 1.
% we already know that p_bar is 1. We will be computing d_bar now.

The Lagrangian reads
$$
  L(x,\lambda)=\begin{cases}
    x^2-(2+\lambda)x & \text{if } x\geq0 \\
    (1-\lambda)x     & \text{otherwise}
  \end{cases}
$$
% there is no mu because there is no equality constraints

A short computation shows that
$$
  d(\lambda)=\begin{cases}
    -\frac{(2+\lambda)^2}{4} & \text{if }\lambda\geq1 \\
    -\infty                  & \text{otherwise}
  \end{cases}
$$

Therefore,
\begin{align*}
  \bar d=d(1)=-\frac94<1=\bar p
\end{align*}

Hence the duality gap
$$
  \bar p-\bar d=\frac54>0
$$

\Definition{6.3.1}{Saddle point}\label{e1525d8}

The triple $(\bar x,\bar\lambda,\bar\mu)\in\R^n\times\R^m_+\times\R^p$ is
called a saddle point of the Lagrangian $L$ of (P) if
$$
  L(\bar x,\lambda,\mu)\leq
  L(\bar x,\bar\lambda,\bar\mu)\leq
  L(x,\bar\lambda,\bar\mu)
$$

\Theorem{6.3.2}{}\label{dc4af95}

The following are equivalent:
\begin{enumerati}
  \item $(\bar x,\bar\lambda,\bar\mu)$ is a saddle point of (P)
  \item $\bar x$ solves (P); $(\bar\lambda,\bar\mu)$ solves (D)
\end{enumerati}

\begin{proof}
  \paragraph{(i) $\implies$ (ii):}

  If $(\bar x,\bar\lambda,\bar\mu)$ is a saddle point of (P), then
  \begin{align*}
    L(\bar x,\bar\lambda,\bar\mu)
     &\stackrel{\text{S.P.}}{\leq}\inf_{x}L(x,\bar\lambda,\bar\mu)      \\
     &\leq\sup_{\lambda\in\R^m_+,\ \mu\in\R^p}\inf_{x}L(x,\lambda,\mu)  \\
     &\leq\inf_{x}\sup_{\lambda\in\R^m_+,\ \mu\in\R^p}L(x,\lambda,\mu)  \\
     &\leq\sup_{\lambda\in\R^m_+,\ \mu\in\R^p}L(\bar x,\lambda,\mu)     \\
     &\stackrel{\text{S.P.}}{\leq} L(\bar x,\bar\lambda,\bar\mu)\Tag{*}
  \end{align*}

  % TODO: look for a problem in the notes that flip inf and sup and have
  % a resulting inequality. This will convince you of the argument
  % above.

  Then,
  \begin{align*}
    d(\bar\lambda,\bar\mu)
     &=\inf_{x} (x,\bar\lambda,\bar\mu)                                        \\
     &\stackrel{(*)}=\sup_{\lambda\in\R^m_+,\ \mu\in\R^p}L(\bar x,\lambda,\mu) \\
     &= p(\bar x) = \bar p<+\infty
  \end{align*}

  Hence if $x\in X$, and by weak duality, $\bar x$ solves (P), and
  $(\bar\lambda,\bar\mu)$ solves (D).

  \paragraph{(ii) $\implies$ (i):}

  Observe that
  \begin{align*}
    L(\bar x,\bar\lambda,\bar\mu)
     &\stackrel{\bar x\in X}{\leq} f(\bar x)                                                 \\
     &\stackrel{\bar x\in X}= p(\bar x)                                                      \\
     &\stackrel{\text{defn. of }p}=\sup_{\lambda\in\R^m_+,\ \mu\in\R^p}L(\bar x,\lambda,\mu) \\
     &=d(\bar\lambda,\bar\mu)                                                                \\
     &=\inf_{x} (x,\bar\lambda,\bar\mu)                                                      \\
     &\leq L(\bar x,\bar\lambda,\bar\mu)                                                     \\
  \end{align*}

  But that's just the original LHS value, and hence all lines are equal. Hence
  \begin{align*}
    L(\bar x,\bar\lambda,\bar\mu)
     &=\inf_{x} (x,\bar\lambda,\bar\mu)                          \\
     &=\sup_{\lambda\in\R^m_+,\ \mu\in\R^p}L(\bar x,\lambda,\mu) \\
  \end{align*}

  And hence $(\bar x,\bar\lambda,\bar\mu)$ is a saddle point.
\end{proof}

% in class, Prof forgot to talk about strong duality in class.

% Exact penalty functions

Consider again
\begin{equation*}
  \begin{array}{l l l}
    \min f(x) & \text{s.t.} & x\in X
  \end{array}\Tag{1}
\end{equation*}

and a penalty function
\begin{equation*}
  P_\alpha^r:=f+\alpha r,\with \alpha>0\Tag{2}
\end{equation*}

with $r\geq0$, $r(x)=0\iff x\in X^r$

\Definition{7.2.1}{}\label{de3d5ee}

The penalty function $P_\alpha^r$ is called exact at a local min $\bar x$ of
(1) if there exists $\bar\alpha$ such that $\bar x$ is a local min of
$P_\alpha^r$ for all $\alpha>\bar\alpha$

Consider now the standard NLP
\begin{equation*}
  \NlpStdForm\Tag{3}
\end{equation*}

with $f,g_i,h_j:\R^n\to\R$ at least cont. diff.

A whole class of penalty functions in the sense of (2) for problem (3) is
defined via
$$
  r_q(x):=\norm{(\max\{g(x),0\},h(x))}_q
$$

Where the $\max$ is interpreted component-wise, and then we're taking the
$q$-norm. The value in $(\cdot)$ is a vector with the first $i$ elements being
$\max\{g_i(x),0\}$ and the last $j$ elements being $h_j(x)$.

and
$$
  \norm{z}_q=\begin{cases}
    \displaystyle\left(\sum_{i=1}^\ell (z_i)^q \right)^{\frac1q} \ldots
                                          & \text{if } q\in[1,\infty) \\[2em]
    \displaystyle\max_{i=\iter1\ell}|z_i| & \text{if } q=+\infty
  \end{cases}
$$

we focus on $q=1$:
$$
  P_\alpha^1(x)=f(x)+\alpha\sum_{j=1}^p|h_j(x)| + \alpha\sum_{i=1}^m\max\{g_i(x),0\}
$$

\Theorem{7.2.2}{}\label{d2d5221}

Let $(\bar x,\bar\lambda,\bar\mu)$ be a KKT point of the convex NLP
\begin{equation*}
  \NlpStdForm\Tag{4}
\end{equation*}

with $f,g_i:\R^n\to\R$ convex and cont. diff, and $h_j:\R^n\to\R$ affine (and
hence convex).

Then $\bar x\in\argmin_XP_\alpha^1(x)$, for all
$\alpha\geq\norm{(\bar\lambda,\bar\mu)}_\infty$.

In particular, $P_\alpha^1$ is exact at $\bar x$ if a CQ holds.

\begin{proof}
  By Theorem 5.1.23 (KKT for convex problems), $\bar x$ is a global minimizer of
  (4). Therefore, by Theorem 6.3.2 (Saddle point theorem), $\bar x$ is a global
  minimizer of the Lagrangian $L(\cdot,\bar\lambda,\bar\mu)$.

  Therefore, for all $x\in\R^n$ and for all
  $\alpha\geq\norm{(\bar\lambda,\bar\mu)}_\infty$, we have
  % TODO: what is the shape of $(\bar\lambda,\bar\mu)$? is it a single column
  % vector? Norming it above here suggests that it is
  $$
    P_\alpha^1(\bar x)=f(\bar x)+\alpha\sum_{j=1}^p|h_j(\bar x)| + \alpha\sum_{i=1}^m\max\{g_i(\bar x),0\}
  $$

  For reference,
  $$
    \norm{(\bar\lambda,\bar\mu)}_\infty:=\max\{
    \iter{|\bar\lambda_1|}{|\bar\lambda_m|},
    \iter{|\bar\mu_1|}{|\bar\mu_p|}
    \}
  $$

  But remember that since the point is feasible, the second and third terms are
  both zero and hence
  \begin{align*}
    P_\alpha^1(\bar x)
     &=f(\bar x)                                                                               \\
     &\stackrel{\text{KKT}}{=}f(\bar x)+\bar\lambda^Tg(\bar x)+\bar\mu^Th(\bar x)              \\
     &= L(\bar x,\bar\lambda,\bar\mu)                                                          \\
     &\stackrel{\text{Thm 6.3.2}}{\leq} L(x,\bar\lambda,\bar\mu)                               \\
     &= f(x)+\sum_{i=1}^m\bar\lambda_ig_i(x)+\sum_{j=1}^p\bar\mu_jh_j(x)                       \\
     &\leq f(x)+\sum_{i=1}^m\bar\lambda_i\max\{g_i(x),0\}+\sum_{j=1}^p|\bar\mu_j|\cdot|h_j(x)| \\
     &\leq f(x)+\alpha\sum_{i=0}^m\max\{g_i(x),0\}+\alpha\sum_{j=1}^p|h_j(x)|                  \\
     &= P_\alpha^1(x)
  \end{align*}

  Hence $\bar x$ is the global minimizer of $P_\alpha^1$, that is for any
  $\alpha\geq\norm{(\bar\lambda,\bar\mu)}_\infty$
  $$
    \bar x\in\argmin_XP_\alpha^1
  $$
\end{proof}

% LAST CHAPTER: SQP methods

\subsection{SQP Methods}

Consider
\begin{equation*}
  \begin{array}{l l l}
    \min f(x) & \text{s.t.} & h_j(x)=0\with\forall j\in J
  \end{array}\Tag{1}
\end{equation*}

with $f,h_j:\R^n\to\R$ twice cont diff.

Define $\Phi:\R^n\times\R^p\to\R^n\times\R^p$ by
$$
  \Phi(x,\mu):=\begin{bmat}
    \nabla_xL(x,\mu) \\ h(x)
  \end{bmat}
$$

where $L:\R^n\times\R^p\to\R$ is the Lagrangian of (1). Then
\begin{equation*}
  (x,\mu)\text{ is a KKT point of (1)}\iff\Phi(x,\mu)=0\Tag{*}
\end{equation*}

where $\Phi$ is $C^1$.

Idea: Apply Newton's method to $(*)$.

\Algorithm{8.1.1}{Lagrange-Newton method}\label{faafd8f}

\begin{enumerate}
  \item [\textbf{(S0)}] Choose $(x^0,\mu^0)\in\R^n\times\R^p$ and set $k:=0$.
  \item [\textbf{(S1)}] If $\Phi(x^k,\mu^k)=0$: STOP
  \item [\textbf{(S2)}] Determine $(\Delta x^k,\Delta\mu^k)$ as solution of
        $$
          \Phi'(x^k,\mu^k)\begin{pmatrix}\Delta x\\\Delta\mu\end{pmatrix}=-\Phi(x^k,\mu^k)
        $$
  \item [\textbf{(S3)}] Put $(x^{k+1},\mu^{k+1}):=(x^k,\mu^k)+(\Delta x^k,\Delta\mu^k)$, $k\gets k+1$, and go to (S1).
\end{enumerate}

``Hessian is the Jacobian of the gradient"

Crucial part for well-definedness is to have $\Phi'$ be invertible at
$(x^k,\mu^k)$

\Theorem{8.2.1}{}\label{c471dcc}

Let $(\bar x,\bar\mu)\in\R^n\times\R^p$ be a KKT point of (1), i.e. a root of
the function $\Phi(x,\mu)=0$ such that:
\begin{enumerati}
  \item (LICQ) The vectors $\nabla h_j(x)$ are linearly indepdendent, for $j\in J$
  \item (2nd order sufficient condition) We have
  $$
    d^T \nabla_{xx}^2 L(\bar x,\bar\mu)d>0\with\forall d\neq0\land\nabla h_j(\bar
    x)^Td=0
  $$
\end{enumerati}

\begin{proof}
  Goal is to show that kernel is trivial.

  Sidenote:
  $$
    h'(x)^T=\begin{bmat}
      \nabla h_1(x) & \ldots & \nabla h_p(x)
    \end{bmat}
    \in\R^{n\times p}
  $$

  Observe that
  % Example of computing jacobians!
  $$
    \Phi(x,\mu)=\begin{pmatrix}
      \nabla_{xx}^2L(x,\mu) & h'(x)^T \\
      h'(x)                 & 0
    \end{pmatrix}\with\forall (x,\mu)\in\R^n\times\R^p
  $$

  Hence,
  \begin{align*}
    % TODO: which space is this exactly?
    \Phi'(\bar x,\bar\mu){\begin{pmat}q\\r\end{pmat}}
     &= 0 \\
    \iff \nabla_{xx}^2L(\bar x,\mu)q + h'(\bar x)^Tr
     &=0  \\
    \text{and}\Quad h'(\bar x)^Tq
     &=0
  \end{align*}

  Note: $\Phi'\in\R^{(n+p)\times(n+p)}$
\end{proof}

\subsection{Nov 30 class}

\subsection{Lagrange-Newton Equation}

Update with $(x^{k+1},\mu^{k+1}):=(x^k,\mu^k)+(\Delta x^k,\Delta \mu^k)$

\begin{align*}
  \Phi'(x^k,\mu^k)\begin{pmat}\Delta x\\\Delta\mu\end{pmat}
   &=-\Phi(x^k,\mu^k) \\
  \iff\begin{bmat}
        \nabla_{xx}^2 L(x^k,\mu^k) & h'(x^k)^T \\
        h'(x^k)                    & 0
      \end{bmat}\begin{pmat}\Delta x\\\Delta\mu\end{pmat}
   &=
  -\begin{bmat}
     \nabla f(x^k)-h'(x^k)^T \\
     h'(x^k)
   \end{bmat}
\end{align*}
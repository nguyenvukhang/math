\section{Real Analysis}\label{cd94400}

\Definition{1.1.1}{Number systems}

\begin{enumerati}
  \item $\N:=$ set of all natural numbers $\{1,2,3,\ldots\}$
  \item $\Z:=$ set of all integers $\{\ldots,-2,-1,0,1,2,\ldots\}$
  \item $\mathbb Q:=$ set of all rational numbers $\Set{\dfrac pq}{p,q\in\Z,q\neq0}$
  \item $\mathbb R:=$ set of all real numbers
\end{enumerati}

We have
$$
  \N\subseteq\Z\subseteq\mathbb Q\subseteq\R.
$$

The set of irrational numbers is denoted by $\R\setminus\mathbb Q$.

\Theorem{1.1.2}{$\sqrt2$ is an irrational number}

\begin{proof}
  Suppose $\sqrt2$ is rational. Then we can write
  $$\sqrt2=\frac ab$$

  where $a$ and $b$ are integers with no common factor other than 1. Then
  $$2=\frac{a^2}{b^2}$$

  and
  $$2b^2=a^2$$

  This says that $a^2$ is even. So $a$ is also even, and $a=2k$ for some integer
  $k$. Then we get
  $$2b^2=4k^2$$

  So
  $$b^2=2k^2$$

  But this says that $b^2$ is even, so $b$ is even. It follows that 2 is a common
  factor for $a$ and $b$. This contradicts our assumption of $a$ and $b$, and
  hence $\sqrt2$ is not rational.
\end{proof}

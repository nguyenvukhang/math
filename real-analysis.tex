\section{Real Analysis}\label{cd94400}

\Definition{1.1.1}{Number systems}\label{d52c6b7}

\begin{enumerati}
  \item $\N:=$ set of all natural numbers $\{1,2,3,\ldots\}$
  \item $\Z:=$ set of all integers $\{\ldots,-2,-1,0,1,2,\ldots\}$
  \item $\mathbb Q:=$ set of all rational numbers $\Set{\dfrac pq}{p,q\in\Z,q\neq0}$
  \item $\mathbb R:=$ set of all real numbers
\end{enumerati}

We have
$$
  \N\subseteq\Z\subseteq\mathbb Q\subseteq\R.
$$

The set of irrational numbers is denoted by $\R\setminus\mathbb Q$.

\Theorem{1.1.2}{$\sqrt2$ is an irrational number}\label{c2585a1}

\begin{proof}
  Suppose $\sqrt2$ is rational. Then we can write
  $$\sqrt2=\frac ab$$

  where $a$ and $b$ are integers with no common factor other than 1. Then
  $$2=\frac{a^2}{b^2}$$

  and
  $$2b^2=a^2$$

  This says that $a^2$ is even. So $a$ is also even, and $a=2k$ for some integer
  $k$. Then we get
  $$2b^2=4k^2$$

  So
  $$b^2=2k^2$$

  But this says that $b^2$ is even, so $b$ is even. It follows that 2 is a common
  factor for $a$ and $b$. This contradicts our assumption of $a$ and $b$, and
  hence $\sqrt2$ is not rational.
\end{proof}

\Principle{1.2.1}{Well-ordering Property of $\N$}\label{cd7c4d1}

Every non-empty subset $S$ of $\N$ has a \textit{least (or minimum)} element.
Formally,
$$
  \exists m\in S:\ \forall s\in S,\ m\leq s
$$

Note that $S$ may not have a largest element.

\Theorem{1.2.2}{Induction on natural numbers}\label{a824f8c}

Let $S\subseteq\N$. If we have
\begin{enumerati}
  \item $1\in S$, and
  \item for every $k\in\N$, $k\in S\implies k+1\in S$.
\end{enumerati}

Then $S=\N$.

\begin{proof}
  Suppose that $S\neq\N$. Then its complement $\N\setminus S\neq\emptyset$

  By the \href{cd7c4d1}{well-ordering property of $\N$}, there exists a least
  element $m\in\N\setminus S$.

  By (i), we have $m\neq 1$ and hence $m\geq2$. Thus, $m-1\in\N$. Since $m$ is
  the smallest natural number \textit{not} in $S$, we have $m-1\in S$. But by
  (ii), $m=(m-1)+1\in S$, which is a contradition to $m\in\N\setminus S$.
\end{proof}

\Theorem{1.2.3}{Principle of Mathematical Induction}\label{b51ca45}

For each $n\in\N$, let $P(n)$ be a statement about $n$. Suppose that
\begin{enumerati}
  \item $P(1)$ is true, and
  \item for every $k\in\N$, if $P(k)$ is true, then $P(k+1)$ is true.
\end{enumerati}

Observe that $1$ can be replaced with any natural number $n_0$, but we would
have only proved that $P$ is true for all natural numbers $\geq n_0$.

\begin{proof}
  Apply \href{a824f8c}{induction on natural numbers} on the set
  $$
    \set{n\in\N}{P(n)\text{ is true}}
  $$
\end{proof}

\Remark{1.3.1}{Algebraic Properties of $\R$}\label{bf61f02}

The binary operation \textbf{addition} on the set $\R$ satisfies the following
properties, for all $a,b,c\in\R$:

\begin{itemize}
  \item [(\textbf{A1})] \textit{(Commutativity)} $a+b=b+a$
  \item [(\textbf{A2})] \textit{(Associativity)} $(a+b)+c=a+(b+c)$
  \item [(\textbf{A3})] \textit{(Existence of additive identity)} $\exists0\in\R:\ a+0=0+a=a$
  \item [(\textbf{A4})] \textit{(Existence of inverse element)} $\forall
        x\in\R,\ \exists {-}x\in\R:\ x+({-}x)=({-}x)+x=0$
\end{itemize}

The binary operation \textbf{multiplication} on $\R$ satisfies the following
properties, for all $a,b,c\in\R$:

\begin{itemize}
  \item [(\textbf{M1})] \textit{(Commutativity)} $a\cdot b=b\cdot a$
  \item [(\textbf{M2})] \textit{(Associativity)} $(a\cdot b)\cdot c=a\cdot
        (b\cdot c)$
  \item [(\textbf{M3})] \textit{(Existence of multiplicative identity)} $\exists1\in\R:\ a\cdot 1=1\cdot a=a$
  \item [(\textbf{M4})] \textit{(Existence of inverse element)} $\forall
        x\in\R,\ \exists 1/x\in\R:\ x\cdot(1/x)=(1/x)\cdot x=1$
\end{itemize}

In addition, the two binary operations satisfy the following property:
\begin{itemize}
  \item [(\textbf{D})] \textit{(Distributivity of multiplication over addition)}
        $$
          a\cdot(b+c)=a\cdot b+a\cdot c\with\forall a,b,c\in\R
        $$
\end{itemize}

Because of (\textbf{A1})-(\textbf{A4}), (\textbf{M1})-(\textbf{M4}), and
(\textbf{D}), we say that $(\R,+,\cdot)$ forms a \textbf{field}.

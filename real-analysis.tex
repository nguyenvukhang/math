\section{Real Analysis}\label{cd94400}

\Definition{1.1.1}{Number systems}\label{d52c6b7}

\begin{enumerati}
  \item $\N:=$ set of all natural numbers $\{1,2,3,\ldots\}$
  \item $\Z:=$ set of all integers $\{\ldots,-2,-1,0,1,2,\ldots\}$
  \item $\Q:=$ set of all rational numbers $\Set{\dfrac pq}{p,q\in\Z,q\neq0}$
  \item $\mathbb R:=$ set of all real numbers
\end{enumerati}

We have
$$
  \N\subseteq\Z\subseteq\Q\subseteq\R.
$$

The set of irrational numbers is denoted by $\R\setminus\Q$.

\Theorem{1.1.2}{$\sqrt2$ is an irrational number}\label{c2585a1}

\begin{proof}
  Suppose $\sqrt2$ is rational. Then we can write
  $$\sqrt2=\frac ab$$

  where $a$ and $b$ are integers with no common factor other than 1. Then
  $$2=\frac{a^2}{b^2}$$

  and
  $$2b^2=a^2$$

  This says that $a^2$ is even. So $a$ is also even, and $a=2k$ for some integer
  $k$. Then we get
  $$2b^2=4k^2$$

  So
  $$b^2=2k^2$$

  But this says that $b^2$ is even, so $b$ is even. It follows that 2 is a common
  factor for $a$ and $b$. This contradicts our assumption of $a$ and $b$, and
  hence $\sqrt2$ is not rational.
\end{proof}

\Principle{1.2.1}{Well-ordering Property of $\mathbb N$}\label{cd7c4d1}

Every non-empty subset $S$ of $\N$ has a \textit{least (or minimum)} element.
Formally,
$$
  \exists m\in S:\ \forall s\in S,\ m\leq s
$$

Note that $S$ may not have a largest element.

\Theorem{1.2.2}{Induction on natural numbers}\label{a824f8c}

Let $S\subseteq\N$. If we have
\begin{enumerati}
  \item $1\in S$, and
  \item for every $k\in\N$, $k\in S\implies k+1\in S$.
\end{enumerati}

Then $S=\N$.

\begin{proof}
  Suppose that $S\neq\N$. Then its complement $\N\setminus S\neq\emptyset$

  By the \href{cd7c4d1}{well-ordering property of $\N$}, there exists a least
  element $m\in\N\setminus S$.

  By (i), we have $m\neq 1$ and hence $m\geq2$. Thus, $m-1\in\N$. Since $m$ is
  the smallest natural number \textit{not} in $S$, we have $m-1\in S$. But by
  (ii), $m=(m-1)+1\in S$, which is a contradition to $m\in\N\setminus S$.
\end{proof}

\Theorem{1.2.3}{Principle of Mathematical Induction}\label{b51ca45}

For each $n\in\N$, let $P(n)$ be a statement about $n$. Suppose that
\begin{enumerati}
  \item $P(1)$ is true, and
  \item for every $k\in\N$, if $P(k)$ is true, then $P(k+1)$ is true.
\end{enumerati}

Observe that $1$ can be replaced with any natural number $n_0$, but we would
have only proved that $P$ is true for all natural numbers $\geq n_0$.

\begin{proof}
  Apply \href{a824f8c}{induction on natural numbers} on the set
  $$
    \set{n\in\N}{P(n)\text{ is true}}
  $$
\end{proof}

\Remark{1.3.1}{Field properties of $\mathbb R$}\label{bf61f02}

The binary operation \textbf{addition} on the set $\R$ satisfies the following
properties, for all $a,b,c\in\R$:

\begin{itemize}
  \item [(\textbf{A1})] \textit{(Commutativity)} $a+b=b+a$
  \item [(\textbf{A2})] \textit{(Associativity)} $(a+b)+c=a+(b+c)$
  \item [(\textbf{A3})] \textit{(Existence of additive identity)} $\exists0\in\R:\ a+0=0+a=a$
  \item [(\textbf{A4})] \textit{(Existence of additive inverse)} $\forall
        x\in\R,\ \exists -x\in\R:\ x+(-x)=(-x)+x=0$
\end{itemize}

The binary operation \textbf{multiplication} on $\R$ satisfies the following
properties, for all $a,b,c\in\R$:

\begin{itemize}
  \item [(\textbf{M1})] \textit{(Commutativity)} $a\cdot b=b\cdot a$
  \item [(\textbf{M2})] \textit{(Associativity)} $(a\cdot b)\cdot c=a\cdot
        (b\cdot c)$
  \item [(\textbf{M3})] \textit{(Existence of multiplicative identity)} $\exists1\in\R:\ a\cdot 1=1\cdot a=a$
  \item [(\textbf{M4})] \textit{(Existence of multiplicative inverse)} $\forall
        x\in\R\sans0,\ \exists 1/x\in\R:\ x\cdot(1/x)=(1/x)\cdot x=1$
\end{itemize}

In addition, the two binary operations satisfy the following property:
\begin{itemize}
  \item [(\textbf{D})] \textit{(Distributivity of multiplication over addition)}
        $$
          a\cdot(b+c)=a\cdot b+a\cdot c\with\forall a,b,c\in\R
        $$
\end{itemize}

Because of (\textbf{A1})-(\textbf{A4}), (\textbf{M1})-(\textbf{M4}), and
(\textbf{D}), we say that $(\R,+,\cdot)$ forms a \textbf{field}.

\Theorem{1.3.2}{Results from the \href{bf61f02}{field properties of $\mathbb R$}}\label{a1bdcab}

For any $a,b,c\in\R$,

\begin{enumerati}
  \item \textit{(Uniqueness of additive inverse)} If $a+b=0$, then $b=-a$
  \item \textit{(Uniqueness of multiplicative inverse)} If $a\cdot b=1$ and
  $a\neq0$, then $b=\frac1a$.
  \item If $a+b=b$, then $a=0$.
  \item If $b\neq0$ and $a\cdot b=b$, then $a=1$.
  \item $a\cdot0=0$.
  \item If $a\cdot b=0$, then $a=0$ or $b=0$.
  \item \textit{(Cancellative property)} If $a\neq0$ and $a\cdot b=a\cdot c$, then $b=c$.
\end{enumerati}

\begin{proof}
  Let $a,b,c\in\R$.

  Proof of (i): Suppose $a+b=0$. Then
  \begin{align*}
    a+b+(-a) &=0+(-a)                                              \\
    a-(-a)+b &=-a\with\desc{(commutativity and additive identity)} \\
    0+b      &=-a\with\desc{(additive inverse)}                    \\
    b        &=-a\with\desc{(additive identity)}
  \end{align*}

  Proof of (ii): Suppose $a\cdot b=1$ and $a\neq0$. Then
  \begin{align*}
    a\cdot b\cdot(1/a) &=1\cdot(1/a)                                                \\
    a\cdot(1/a)\cdot b &=1/a\with\desc{(commutativity and multiplicative identity)} \\
    1\cdot b           &=1/a\with\desc{(multiplicative inverse)}                    \\
    b                  &=1/a\with\desc{(multiplicative identity)}
  \end{align*}

  Proof of (iii): Suppose $a+b=b$. Then
  \begin{align*}
    a+b+(-b) &=b+(-b)                           \\
    a+0      &=0\with\desc{(additive inverse)}  \\
    a        &=0\with\desc{(additive identity)}
  \end{align*}

  Proof of (iv): Suppose $a\cdot b=b$ and $b\neq0$. Then
  \begin{align*}
    a\cdot b\cdot(1/b) &=b\cdot(1/b)                            \\
    a\cdot1            &=1\with\desc{(multiplicative inverse)}  \\
    a                  &=1\with\desc{(multiplicative identity)}
  \end{align*}

  Proof of (v):
  \begin{align*}
    a\cdot0 &= a\cdot0+0\with\desc{(additive identity)}                       \\
            &= a\cdot0+[(a\cdot0)+(-(a\cdot0))]\with\desc{(additive inverse)} \\
            &= (a\cdot0+a\cdot0)+(-(a\cdot0))\with\desc{(associativity)}      \\
            &= a\cdot(0+0)+(-(a\cdot0))\with\desc{(distributivity)}           \\
            &= a\cdot0+(-(a\cdot0))\with\desc{(additive identity)}            \\
            &= 0\with\desc{(additive inverse)}                                \\
  \end{align*}

  Proof of (vi): Suppose $a\cdot b=0$. Now if $a=0$ then we are done. So suppose
  that $a\neq0$. Then $1/a$ exists.
  \begin{align*}
    a\cdot b\cdot(1/a) &=0\cdot(1/a)                            \\
    a\cdot b\cdot(1/a) &=0\with\desc{(result (v))}              \\
    a\cdot(1/a)\cdot b &=0\with\desc{(commutativity)}           \\
    1\cdot b           &=0\with\desc{(multiplicative inverse)}  \\
    b                  &=0\with\desc{(multiplicative identity)}
  \end{align*}

  Proof of (vii): Suppose that $a\neq0$ and $a\cdot b=a\cdot c$. By
  \href{bf61f02}{(\textbf{M4})} that $1/a$ exists. Then
  \begin{align*}
    (1/a)\cdot(a\cdot b)  &=(1/a)\cdot(a\cdot c)                             \\
    ((1/a)\cdot a)\cdot b &=((1/a)\cdot a)\cdot c\with\desc{(commutativity)} \\
    1\cdot b              &=1\cdot c\with\desc{(multiplicative inverse)}     \\
    b                     &= c\with\desc{(multiplicative identity)}
  \end{align*}
\end{proof}

\Remark{1.3.3}{Order properties of $\mathbb R$}\label{d49c63e}

There is a binary relation $>$ on $\R$ which has the following properties (with
$a,b,c\in\R$):

\begin{enumerati}
  \item [(\textbf{O1})] If $a>b$, then $a+c>b+c$.
  \item [(\textbf{O2})] If $a>0$ and $b>0$, then $a\cdot b>0$.
  \item [(\textbf{O3})] \textit{(Trichotomy Property)} If $a,b\in\R$, then
  exactly one of the following holds:
  $$a>b,\quad a=b,\quad b>a$$
  \item [(\textbf{O4})] \textit{(Transitive Property)} If $a>b$ and $b>c$, then
  $a>c$.
\end{enumerati}

\Theorem{1.3.7}{}\label{b069294}

If $a\in\R$ is such that $0\leq a<\epsilon$ for every positive number
$\epsilon$, then $a=0$.

\begin{proof}
  Since $a\geq0$, by definition either $a>0$ or $a=0$. Suppose $a>0$. Then
  $\frac a2>0$.

  Now let $\epsilon:=\frac a2$. Then by assumption, $a<\epsilon=\frac a2$
  \begin{align*}
    a<\frac a2 &\implies 2\cdot a<2\cdot\frac a2=a \\
               &\implies 2a-a<0                    \\
               &\implies a<0
  \end{align*}

  This contradict that $a>0$. Hence we must have $a=0$.
\end{proof}

\Theorem{1.6.2}{Triangle inequality for $\mathbb R$}\label{f1288ad}

For all $a,b\in\R$, we have
$$
  |a+b|\leq|a|+|b|
$$

\begin{proof}
  We have $-|a|\leq a\leq|a|$ and $-|b|\leq b\leq|b|$. Adding, we have
  \begin{align*}
    -(|a|+|b|)\leq a+b\leq |a|+|b|
  \end{align*}

  which implies that $|a+b|\leq|a|+|b|$.
\end{proof}

\Corollary{1.6.3}{Corollaries of triangle inequality for $\mathbb R$}\label{f699f4d}

\begin{enumerati}
  \item $\big||a|-|b|\big|\leq|a-b|$
  \item $|a-b|\leq|a|+|b|$
  \item $\big||a|-|b|\big|\leq|a+b|$
\end{enumerati}

\begin{proof}
  By the \href{f1288ad}{triangle inequality},
  $$
    |a|=|(a-b)+b|\leq|a-b|+|b|
  $$

  So
  \begin{equation*}
    |a|-|b|\leq|a-b|\Tag{*}
  \end{equation*}

  By symmetry, we also have
  $$
    |b|-|a|\leq|b-a|
  $$

  which can be rewritten as
  \begin{equation*}
    -(|a|-|b|)\leq|a-b|\Tag{**}
  \end{equation*}

  With $(*)$ and $(**)$, we have (i).

  (ii) is obtained by using $-b$ in the place of $b$ in the
  \href{f1288ad}{triangle inequality}

  (iii) follows from using $-b$ in the place of $b$ in (i).
\end{proof}

\Definition{2.1.1}{Boundedness}\label{e4698be}

A non-empty set $S\subseteq\R$ is said to be \textbf{bounded above} if there
exists some $M\in\R$ such that
$$
  x\leq M,\with\forall x\in S.
$$

Such an $M$ is called an \textbf{upper bound} of $S$.

On the other hand, $S$ is said to be \textbf{bounded below} if there exists
some $m\in\R$ such that
$$
  m\leq x,\with\forall x\in S.
$$

Such an $m$ is called a \textbf{lower bound} of $S$.

If $S$ is both bounded above and bounded below, then we simply call it
\textbf{bounded}.

Equivalently, a set $S$ is bounded if there exists $M\geq0$ such that
$$
  |x|\leq M,\with\forall x\in S
$$

\Definition{2.2.1}{Maximum and minimum of a subset of $\mathbb R$}\label{c3ec51c}

For a non-empty $S\subseteq\R$, one defines the maximum of $S$ to be the
(necessarily unique) number $M$ such that
\begin{enumerati}
  \item $M\in S$, and
  \item $x\leq M$ for all $x\in S$.
\end{enumerati}

Similarly, the \textbf{minimum} of $S$ is the (necessarily unique) number $m$
such that
\begin{enumerati}
  \item $m\in S$, and
  \item $m\leq x$ for all $x\in S$.
\end{enumerati}

\Definition{2.3.1}{Supremum}\label{e6981e1}

Let $E\subseteq\R$ be non-empty. A real number $M\in\R$ is called the
\textbf{supremum} of $E$ (we write $\sup E$) if
\begin{enumerati}
  \item $M$ is an \href{e4698be}{upper bound} of $E$, and
  \item if $M'$ is an upper bound of $E$, then $M\leq M'$.
\end{enumerati}

\Lemma{2.3.2}{}\label{f77f162}

Let $E\subseteq\R$ be non-empty. Then $M=\sup E$ if and only if $M$ is an upper
bound of $E$ and for every $\epsilon>0$, there exists $x_\epsilon\in E$ such
that $M-\epsilon<x_\epsilon$.

\begin{proof}
  ($\implies$) Suppose $M=\sup E$. Let $\epsilon>0$. Then $M-\epsilon<M$. Since
  $M$ is the least upper bound of $E$ by definition, $M-\epsilon$ cannot be an
  upper bound for $E$. Hence there exists $x_\epsilon\in E$ such that
  $M-\epsilon<x_\epsilon$.

  ($\impliedby$) Suppose $M$ is an upper bound for $E$ and that there exists
  $x_\epsilon\in E$ such that $M-\epsilon<x_\epsilon$. Let $M'$ be an upper
  bound of $E$. Suppose on the contrary that $M'<M$. Then we let
  $\epsilon:=M-M'>0$. Then there exists $x_\epsilon\in E$ such that
  $$
    M'=M-(M-M')=M-\epsilon<x_\epsilon
  $$

  This contradicts the assumption that $M'$ is an upper bound for $E$. Hence we
  must have $M\leq M'$, making $M$ the least upper bound of $E$.
\end{proof}

\Lemma{2.3.3}{}\label{cd8e7c5}

If $A\subseteq B\subseteq\R$ and both $\sup A$ and $\sup B$ exist, then $\sup
A\leq\sup B$.

\begin{proof}
  $\sup B$ is an upper bound for $B$, but since $A\subseteq B$, $\sup B$ is an
  upper bound for $A$ as well. Since $\sup A$ is the least upper bound of $A$,
  we have $\sup A\leq\sup B$.
\end{proof}

\Definition{2.3.4}{Infimum}\label{ff16df6}

Let $E\subseteq\R$ be non-empty. A real number $m\in\R$ is called the
\textbf{infimum} of $E$ (we write $\inf E$) if
\begin{enumerati}
  \item $m$ is a \href{e4698be}{lower bound} of $E$, and
  \item if $m'$ is a lower bound of $E$, then $m'\leq m$.
\end{enumerati}

This is an analog to \href{e6981e1}{Definition 2.3.1}.

\Lemma{2.3.5}{}\label{fec9bdb}

Let $E\subseteq\R$ be non-empty. Then $m=\inf E$ if and only if $m$ is a lower
bound of $E$ and for every $\epsilon>0$, there exists $x_\epsilon\in E$ such
that $x_\epsilon<m+\epsilon$.

\begin{proof}
  Exercise (similar to proof of \href{f77f162}{Lemma 2.3.2})
\end{proof}

\Principle{2.3.6}{Completeness/supremum property of $\mathbb R$}\label{f330cf9}

Every non-empty subset of $\R$ which is bounded above has a supremum in $\R$.

In other words, if $E\subseteq\R$ is non-empty, then $\sup E$ exists.

\Remark{2.3.7}{}\label{b1dc879}

This marks the end of our assumptions, which are:

\begin{enumerati}
  \item the \href{bf61f02}{field properties of $\R$} (\textbf{A1})-(\textbf{A4}),
  (\textbf{M1})-(\textbf{M4}), and (\textbf{D}).
  \item the \href{d49c63e}{order properties of $\R$} (\textbf{O1})-(\textbf{O4}).
  \item the \href{f330cf9}{completeness property of $\R$}.
\end{enumerati}

With these we will build up other properties of $\R$.

\Theorem{2.3.8}{The infimum property of $\mathbb R$}\label{ab2a2fe}

Every non-empty subuset of $\R$ which is bounded below has an infimum in $\R$.

\begin{proof}

  Let $E\subseteq\R$ be non-empty and bounded below by $b\in\R$. Let
  $A:=\set{-x}{x\in E}$. Then $A\subseteq\R$ and is non-empty. For all $x\in E$,
  $b\leq x$ and hence $-x\leq-b$. And so $-b$ is an upper bound for $A$.

  Since $A$ is non-empty and bounded above, by the \href{f330cf9}{supremum
  property of $\R$}, $A$ has a supremum $M\in\R$. We claim that
  \begin{equation*}
    \inf E=-\sup A=-M.\Tag{*}
  \end{equation*}

  Since $M$ is an upper bound for $A$,
  \begin{align*}
    -x &\leq M,\with\forall{-x}\in A \\
    -x &\leq M,\with\forall{x}\in E  \\
    -M &\leq x,\with\forall{x}\in E
  \end{align*}

  Hence $-M$ is a lower bound for $E$.

  Now let $m$ be another lower bound of $E$. Then $-m$ is an upper bound of $A$.
  Since $M=\sup A$, we have $M\leq-m$. So $m\leq-M$. Hence $-M$ is indeed the
  greatest lower bound of $E$. This proves $(*)$.
\end{proof}

\Result{2.3.9}{}\label{f426fd0}

Let $A,B\subseteq\R$ be non-empty sets, and let
$$
  C:=\set{a+b}{a\in A,\ b\in B}
$$

Then $\sup C=\sup A+\sup B$.

\begin{proof}
  Let $c\in C$. Then $c=a+b$ for some $a\in A$ and $b\in B$. Now since
  $a\leq\sup A$ and $b\leq\sup B$, we have
  $$
    c=a+b\leq\sup A+\sup B.
  $$

  Hence $\sup A+\sup B$ is an upper bound of $C$.

  Next, let $M$ be an upper bound of $C$. Then for all $a\in A$ and $b\in B$,
  $$
    a+b\leq M
  $$

  and thus $a\leq M-b$. So then for each $b\in B$, $M-b$ is an upper bound for
  $A$. Consequently, $\sup A\leq M-b$, and we have
  $$
    b\leq M-\sup A\with\forall b\in B.
  $$

  Which now implies that $M-\sup A$ is an upper bound for $B$, so
  $$
    \sup B\leq M-\sup A
  $$

  and thus
  $$
    \sup A+\sup B\leq M
  $$

  Showing that $\sup A+\sup B$ is indeed the least upper bound for $C$. Hence
  $\sup C=\sup A+\sup B$.
\end{proof}

\Theorem{2.4.1}{Archimedean property of $\mathbb R$}\label{fbc2289}

For any $x\in\R$, there exists $n_x\in\N$ such that $x<n_x$.

Alternatively, any $x\in\R$ is not an upper bound for $\N$.

In other words, $\N$ is not bounded above in $\R$.

\begin{proof}
  Suppose on the contrary that the Archimedean property of $\R$ does not hold.

  Then there exists some $x\in\R$ such that $n\leq x$ for all $n\in\N$. That is,
  the non-empty set $\N$ is bounded above.

  By the \href{f330cf9}{completeness property of $\R$}, $M=\sup\N$ exists.

  By \href{f77f162}{Lemma 2.3.2} (using $\epsilon:=1$), there exists $\bar
  n\in\N$ such that $M-1<\bar n$. Then $M<\bar n+1$. But $\bar n+1\in\N$. This
  contradicts that $M$ is an upper bound of $\N$.
\end{proof}

\Corollary{2.4.2}{}\label{d845856}

Let $A\subseteq\R$ be given by $A:=\set{\dfrac1n}{n\in\N}$. Then
\begin{enumerati}
  \item $\inf A=0$, and
  \item given any $\epsilon>0$, there exists $n_\epsilon\in\N$ such that
  $0<\dfrac1{n_\epsilon}<\epsilon$.
\end{enumerati}

Note that (ii) is significant because it claims that for any positive real
number, there exists a \textit{rational} number between it and zero.

\begin{proof}
  Proving (i): For any $x\in A$, $x=\frac1n$ for some $n\in\N$, and thus $x>0$.
  Thus $0$ is a lower bound of $A$. Now suppose $m'$ is another lower bound for
  $A$. Then
  \begin{equation*}
    m'\leq\frac1n\with\forall n\in\N\Tag{*}
  \end{equation*}

  If $m'>0$, then $1/m'>0$. By the \href{fbc2289}{Archimedean property of $\R$},
  there exists $\bar n\in\N$ such that
  $$
    \frac1{m'}<\bar n\implies\frac1{\bar n}<m'
  $$

  which contradicts $(*)$. Hence we must have $m'\leq0$, which implies that 0 is
  the greatest lower bound of $A$.

  Proving (ii): from (i) since $\epsilon>0=\inf A$, $\epsilon$ is not a lower
  bound of $A$. That is, there is an element of $A$ smaller than $\epsilon$:
  $$
    \exists n_\epsilon\in\N:\ \frac1{n_\epsilon}<\epsilon
  $$

  This completes the proof.
\end{proof}

\Corollary{2.4.3}{Existence of the floor of a real number}\label{abc7dbd}

Let $x\in\R$. Then there exists a unique $m\in\Z$ such that
$$
  m\leq x<m+1
$$

We denote $m$ by $\floor x$.

\begin{proof}
  For this proof, we shall consider two cases:

  Case 1: $x\geq 1$. Consider the set
  $$
    S:=\set{n\in\N}{n>x-1}\subseteq\N
  $$

  We claim that $\floor x$ is minimum of this set.

  By the \href{fbc2289}{Archimedean property of $\R$}, we know that $S$ is
  non-empty. Then by the \href{cd7c4d1}{well-ordering property of $\N$}, it
  follows that $S$ has a minimum element, which we denote by $m$. Since $m\in S$,
  it follows that $m\in\N$ and
  \begin{equation*}
    m>x-1\implies x<m+1\Tag{*}
  \end{equation*}

  Next we show that $m\leq x$. Suppose on the contrary that $m>x$. Then
  \begin{align*}
    m>x\geq1
     &\implies m-1>0\quad\text{and}\quad m-1>x-1    \\
     &\implies m-1\in\N\quad\text{and}\quad m-1>x-1 \\
     &\implies m-1\in S
  \end{align*}

  But this contradicts that $m=\min S$. Hence $m\leq x$. Together with $(*)$, we
  have
  $$
    m\leq x<m+1.
  $$

  Case 2: $x<1$. It follows from the \href{fbc2289}{Archimedean property of $\R$}
  that there exists $k\in\N$ such that
  $$
    1-x<k
  $$

  which implies that $x+k>1$. Then from Case 1 applied to $x+k$, there exists
  $m'\in\Z$ such that
  $$
    m'\leq x+k<m'+1
  $$

  which is then
  $$
    m'-k\leq x<m'-k+1
  $$

  Let $m:=m'-k\in\Z$. then we have $m\leq x<m+1$. Thus we have proved existence.

  Next, on to uniqueness. Let $m_1,m_2\in\Z$ be such that
  $$
    m_1\leq x<m_1+1\quad\text{and}\quad m_2\leq x<m_2+1
  $$

  Then we have
  $$
    m_1\leq x<m_2+1 \implies m_1-m_2<1
  $$

  and by symmetry, $m_2-m_1<1$. Hence
  $$
    -1<m_1-m_2<1
  $$

  But since $m_1,m_2\in\Z$, we have $m_1-m_2\in\Z$ and thus $m_1-m_2=0$. Hence
  $m_1=m_2$ and this completes the proof for uniqueness.
\end{proof}

\Lemma{2.4.3}{}\label{b88beb7}

There exists a unique positive real number $a$ such that $a^2=2$, without
assuming the existence of $\sqrt2\in\R$.

\begin{proof}
  \textit{(Existence)} Consider the set
  $$
    S:=\set{x\in\R}{x\geq0\text{ and }x^2<2}\subseteq\R
  $$

  We claim that $(\sup S)^2=2$.

  $S$ is non-empty since $1\in S$. Also, $S$ is bounded above (by 2, for
  instance). Hence by the \href{f330cf9}{completeness property of $\R$}, $\sup
  S$ exists in $\R$. Let $a:=\sup S\in\R$.

  We know that $a>0$ since $1\in S$, and hence $a$ is positive, as desired.

  It remains to show that $a^2=2$. By the \href{d49c63e}{trichotomy property of
  $\R$}, we just have to exclude the possibilities
  $$
    \text{Case 1: }a^2<2\quad\text{and}\quad\text{Case 2: }a^2>2
  $$

  \paragraph{Case 1: $a^2<2$.}

  We will argue that there exists some $n\in\N$ such that $(a+\frac1n)^2<2$,
  which implies that $(a+\frac1n)^2\in S$, which then implies that $a=\sup S$ is
  not an upper bound of $S$.

  Observe that
  \begin{equation*}
    \Big(a+\frac1n\Big)^2=a^2+\frac{2a}n+\frac1{n^2}\leq
    a^2+\frac{2a}n+\frac1n=a^2+\frac{2a+1}n\Tag{1}
  \end{equation*}

  since $\dfrac1{n^2}\leq\dfrac1n$ for any $n\in\N$. As $a^2<2$, we have
  \begin{equation*}
    a^2+\frac{2a+1}n<2\iff n>\frac{2a+1}{2-a^2}\Tag{2}
  \end{equation*}

  Since $a^2<2$, we have $\dfrac{2a+1}{2-a^2}\in\R$. Thus by the
  \href{fbc2289}{Archimedean property of $\R$}, there exists $n\in\N$ satisfying
  $$
    n>\frac{2a+1}{2-a^2}.
  $$

  Fixing this $n$, and together with (2) and then (1), we have
  $(a+\frac1{n})^2<2$. Hence Case 1 is not possible.

  \paragraph{Case 2: $a^2>2$.}

  We claim that there exists some $n\in\N$ such that $a-\frac1n$ is an upper
  bound of $S$, breaking the fact that $a$ is the least upper bound of $S$. We
  proceed by
  \begin{enumerati}
    \item Find $n\in\N$ such that $(a-\frac1n)^2>2$.
    \item Show that $x\leq a-\frac1n$ for all $x\in S$.
  \end{enumerati}

  Step (i): Note that
  \begin{equation*}
    \Big(a-\frac1n\Big)^2=a^2-\frac{2a}n+\frac1{n^2}>a^2-\frac{2a}n\Tag{3}
  \end{equation*}

  On the other hand, we have
  \begin{equation*}
    a^2-\frac{2a}n>2\iff\frac1n<\frac{a^2-2}{2a}\Tag{4}
  \end{equation*}

  Since $a^2>2$ and $a>0$, we have $\dfrac{a^2-2}{2a}>0$, and by
  \href{d845856}{Corollary 2.4.2(ii)}, there exists $n\in\N$ such that
  $\dfrac1n<\dfrac{a^2-2}{2a}$.

  Fixing this $n$, and together with (4) and then (3), we have $(a-\frac1n)^2>2$

  Step (ii): For all $x\in S$, we have $x\geq0$ and $x^2<2$. Thus,
  \begin{equation*}
    \Big(a-\frac1n\Big)^2-x^2>2-2=0\implies\Big(a-\frac1n+x\Big)\Big(a-\frac1n-x\Big)>0
  \end{equation*}

  Note that $a>1$, $\frac1n\leq 1$, $x>0$, and thus $a-\dfrac1n+x>0$. Hence we
  must have
  $$
    a-\frac1n-x>0
  $$

  and thus $x<a-\frac1n$. This completes the contradiction of Case 2.

  Hence we must have $a^2=2$.

  \textit{(Uniqueness)} Suppose $a,b\in\R$ with $a>0$ and $b>0$ such that
  $a^2=2$ and $b^2=2$. Then
  \begin{equation*}
    a^2-b^2=2-2=0\implies(a-b)(a+b)=0
  \end{equation*}

  Since $a>0$ and $b>0$, it follows that $a+b>0$ and in particular $a+b\neq0$.
  Hence we must have $a-b=0$, which means that $a=b$.
\end{proof}

\Theorem{2.4.5}{Existence of the positive $k$-th root of a positive real number}\label{c70a9ac}

Let $c>0$ and $k\in\N$. Then there exists a unique $a\in\R$ with $a^k=c$.

\begin{proof}
  The proof is similar to the \href{b88beb7}{square root case}. Let
  $$
    S:=\set{t\in\R}{t>0\text{ and }t^k<c}
  $$

  Then one can show that $1\in S$ if $c>1$, and $\frac c2\in S$ if $c\leq1$
  (hence $S$ is non-empty). Moreover, $c$ is an upper bound of $S$ if $c>1$, and
  1 is an upper bound of $S$ if $c\leq1$ (hence $S$ is bounded above). By the
  \href{f330cf9}{supremum property of $\R$}, $a=\sup S$ exists. We claim that
  $a^k=c$. To justify this claim, one shows that it is impossible to have $a^k<c$
  or $a^k>c$. Again, refer to the \href{b88beb7}{square root case} for
  inspiration.
\end{proof}

\Theorem{2.4.6}{Density Theorem}\label{d0c9c52}

For any $x,y\in\R$ satisfying $x<y$, there exists a $r\in\Q$ such that
$$
  x<r<y
$$

\begin{proof}
  Since $x<y$, we have $y-x>0$ and thus by \href{d845856}{Corollary 2.4.2(ii)},
  there exists $n\in\N$ such that
  \begin{align*}
    y-x>\frac1n &\implies ny-nx>1        \\
                &\implies nx+1<ny\Tag{*}
  \end{align*}

  Then by \href{abc7dbd}{Corollary 2.4.3}, the floor $\floor{nx}\in\Z$ exists and
  it satisfies
  $$
    \floor{nx}\leq nx <\floor{nx}+1\implies nx<\floor{nx}\leq nx+1
  $$

  Together with $(*)$, we have
  $$
    nx<\floor{nx}+1<ny
  $$

  and thus
  $$
    x<\frac{\floor{nx}+1}{n}<y
  $$

  Hence by setting $r:=\dfrac{\floor{nx}+1}{n}$, we have
  $$
    r\in\Q\text{ and }x<r<y
  $$
\end{proof}

\Example{2.4.6}{}\label{ade99b7}

Let $E:=\set{x\in\Q}{x<\sqrt3}$. Then $\sup E=\sqrt3$.

\begin{proof}
  By definition of $E$, $x\leq\sqrt3$ for all $x\in E$. Thus, $E$ is bounded
  above (by $\sqrt3$). Also, since $0\in E$, $E$ is non-empty. Thus by the
  \href{f330cf9}{completeness property of $\R$}, $\sup E$ exists in $\R$.

  Since $\sqrt3$ is an upper bound of $E$, we must have $\sup E\leq\sqrt3$.

  Suppose that $\sup E\neq\sqrt3$. Then $\sup E<\sqrt3$. By the
  \href{d0c9c52}{Density Theorem}, there exists $r\in\Q$ such that
  \begin{equation*}
    \sup E<r<\sqrt3\Tag{*}
  \end{equation*}

  Since $r\in\Q$ and $r<\sqrt3$, it follows that $r\in E$. But this and $(*)$
  contradicts the fact that $\sup E$ is an upper bound for $E$.

  Hence we must have $\sup E=\sqrt3$.
\end{proof}

\Corollary{2.4.6}{}\label{d4d76b6}

Let $\alpha\in\R$, and let
$$
  E_\alpha:=\set{x\in\Q}{x<\alpha}\subseteq\Q
$$

Then $\sup E_\alpha=\alpha$.

\begin{proof}
  Exercise. (Similar to \href{ade99b7}{Example 2.4.6})
\end{proof}

\Corollary{2.4.7}{}\label{b0d86cf}

If $a,b\in\R$ such that $a<b$, then there exists $x\in\R\setminus\Q$ such that
$a<x<b$.

\begin{proof}
  If $a<b$, then $a<\dfrac{a+b}2<b$ and thus
  $$
    \frac a{\sqrt2}<\frac{a+b}{2\sqrt2}<\frac b{\sqrt2}
  $$

  By the \href{d0c9c52}{density theorem}, there exist $r_1,r_2\in\Q$ such that
  $$
    \frac a{\sqrt2}<r_1<\frac{a+b}{2\sqrt2}<r_2<\frac b{\sqrt2}
  $$

  At least one of $r_1,r_2$ is non-zero. Call it $r$. Then we have $r\in\Q\sans0$
  and
  $$
    \frac a{\sqrt2}<r<\frac b{\sqrt2}
  $$

  Hence we have $a<r\sqrt2<b$. $r\sqrt2$ is because $\sqrt2$ is irrational (by
  \href{c2585a1}{Theorem 1.1.2}), and by a \href{d9d3f10}{known result}, the
  product of a rational number and an irrational number is irrational.
\end{proof}

\Corollary{2.4.8}{}\label{e43d143}

If an interval $I\subset\R$ has at least two elements, then $I$ contains
infinitely many rational numbers and infinitely many irrational numbers.

\begin{proof}
  \def\all{\iter{x_1}{x_n}}

  Assume that $I$ contains finitely many rational numbers. Enumerate all of them
  by $\all\in I$ in order of increasing value:
  $$
    x_1<x_2<\ldots<x_n.
  $$
  Also, by assumption we have $n\geq2$.

  By the \href{d0c9c52}{density theorem}, there exists $r\in\Q$ such that
  $$
    x_1<r<x_2
  $$

  Note that since $I$ is an interval, we have $r\in I$. But clearly $r$ is not
  equal to any of the $\all$ previously identified. This contradicts the
  assumption that $\all$ are all the numbers in $I$.

  Hence $I$ must contain infinitely many rational numbers.

  The case with irrational numbers is completely analog to this, but instead of
  the density theorem we use \href{b0d86cf}{Corollary 2.4.7}.
\end{proof}

\Definition{2.4.9}{Dense sets}\label{e929c5e}

The set $D\subseteq\R$ is said to be \textbf{dense} in $\R$ if for any
$a,b\in\R$ with $a<b$, we have $D\cap(a,b)\neq\emptyset$.

In other words, $\exists x\in D$ such that $a<x<b$.

\Remark{2.4.10}{}\label{bb3cf6b}

\begin{enumerati}
  \item By the \href{d0c9c52}{density theorem}, $\Q$ is dense in $\R$.
  \item By \href{b0d86cf}{Corollary 2.4.7}, $\R\setminus\Q$ is dense in $\R$.
\end{enumerati}

\Definition{2.5.1}{Intervals}\label{c65e94a}

An \textbf{interval} is a subset $I$ of $\R$ with the following (equivalent)
properties:
\begin{itemize}
  \item if $x\leq t\leq y$ and $x,y\in I$, then $t\in I$.
  \item if $x,y\in I$ and $x\leq y$, then $[x,y]\subseteq I$.
\end{itemize}

\Definition{3.1.1}{Sequences}\label{b5fa0e4}

A \textbf{sequence} in $\R$ is a function $X:\N\to\R$.

The numbers $\set{X(n)}{n\in\N}$ are called the \textbf{terms} of the sequence.
For each $n\in\N$, $X(n)$ is called the $n$-th term of the sequence.

\paragraph{Notation.}

We usually write $x_n$ for $X(n)$ and denote the sequence $X$ by any one of
$$
  \{x_n\},\ \{x_n\}_{n=1}^\infty,\ \{x_n\}_{n\in\N},\ \{x_n\}_\N
$$

\Definition{3.1.2}{Constant sequence}\label{d661313}

A constant sequence is of the form
$$
  \{c,c,c,\ldots\}
$$

for some constant $c\in\R$.

\Definition{3.1.3}{Neighborhoods}\label{ba35f12}

Let $a\in\R$ and $\epsilon>0$. The \textbf{$\epsilon$-neighborhood of $a$} is
the set
$$
  B_\epsilon(a)=\set{x\in\R}{|x-a|<\epsilon}
$$

or alternatively, $(a-\epsilon,a+\epsilon)$.

\Definition{3.1.4}{Limit}\label{e565120}

We say that $\bar x$ is the \textbf{limit} of $\{x_n\}$ if for every
$\epsilon>0$, there exist $K=K(\epsilon)\in\N$ such that
$$
  n\geq K\implies|x_n-\bar x|<\epsilon
$$

or equivalently,
$$
  \forall n\geq K,\ |x_n-\bar x|<\epsilon
$$

or,
$$
  \forall n\geq K,\ x_n\in B_\epsilon(\bar x)
$$

\paragraph{Remark}

Here we write $K=K(\epsilon)$ to signify that $K$ depends on $\epsilon$.

\Definition{3.1.5}{Convergence}\label{de3e28a}

\begin{enumerati}
  \item If $\bar x$ is the limit of $\{x_n\}$, then we also say that $\{x_n\}$
  \textbf{converges} to $\bar x$, and we write
  $$
    \lim_{n\to\infty}x_n=\bar x
  $$

  or `` $x_n\to\bar x$ as $n\to\infty$ " or simply `` $x_n\to\bar x$ ".
  \item We say that a sequence $\{x_n\}$ \textbf{converges} if it converges to a
  (finite) limit $\bar x\in\R$; and that it \textbf{diverges} if it does not
  converge (to a finite limit).
\end{enumerati}

\Theorem{3.1.6}{Uniqueness of limit}\label{e6b43e0}

If $\{x_n\}$ converges, then it has exactly one limit.

\begin{proof}
  Suppose $x$ and $x'$ are limits of $\{x_n\}$. Let $\epsilon>0$ be
  arbitrarily given, and let $\bar\epsilon=\epsilon/2$.

  Since $x_n\to x$, there exists $K_1\in\N$ such that
  $$
    n\geq K_1\implies |x_n-x|<\bar\epsilon
  $$

  Similarly since $x_n\to x'$, there exists $K_2\in\N$ such that
  $$
    n\geq K_2\implies |x_n-x'|<\bar\epsilon
  $$

  Then let $K:=\max\{K_1,K_2\}\in\N$. Then for all $n\geq K$,
  \begin{align*}
    |x-x'| &= |(x-x_n)-(x_n-x')|                                                  \\
           &\leq|x-x_n|-|x_n-x'|\with\desc{(\href{f1288ad}{triangle inequality})} \\
           &<\bar\epsilon+\bar\epsilon                                            \\
           &=\epsilon
  \end{align*}

  Since $\epsilon>0$ is arbitrary, by \href{1.3.7}{Theorem 1.3.7} we have
  $|x-x'|=0$ and hence $x=x'$.
\end{proof}

\Definition{3.2.1}{Bounded sequence}\label{d5ed299}

The boundedness of a sequence $\{x_n\}$ is determined by the set
$$
  \set{x_n}{n\in\N}
$$

and the definitions stated \href{e4698be}{here}.

\Theorem{3.2.2}{Convergence implies boundedness}\label{d8148e6}

Every convergent sequence is bounded.

\begin{proof}
  Let $\{x_n\}$ be a convergent sequence and $\lim_{n\to\infty}x_n=\bar x$. Put
  $\epsilon:=1$. Then there exists $K\in\N$ such that
  $$
    |x_n-\bar x|<1,\with\forall n\geq K
  $$

  Thus when $n\geq K$,
  \begin{align*}
    |x_n| &= |(x_n-\bar x)+\bar x|                                                    \\
          &\leq|x_n-\bar x|+|\bar x|\with\desc{(\href{f1288ad}{triangle inequality})} \\
          &\leq1+|\bar x|
  \end{align*}

  Let $M=\max\{\iter{x_1}{x_{K-1}},1+|\bar x|\}$. Then
  $$
    |x_n|\leq M,\with\forall n\in\N
  $$

  So $\{x_n\}$ is bounded.
\end{proof}

\Corollary{3.2.3}{}\label{a46c1d7}

\begin{enumerati}
  \item Every unbounded sequence is divergent (contrapositive of \href{d8148e6}{Theorem
  3.2.2})
  \item Boundedness does not imply convergence. Consider the bounded sequence $\{x_n\}$
  defined by $x_n=(-1)^n$.
\end{enumerati}

\Theorem{3.2.2}{Limit arithmetic}\label{d13a5e7}

If $\displaystyle\lim_{n\to\infty}x_n=\bar x$ and
$\displaystyle\lim_{n\to\infty}y_n=\bar y$, then

\begin{enumerati}
  \def\l{\displaystyle\lim_{n\to\infty}}
  \item $\l(x_n+y_n)=\bar x+\bar y$
  \item $\l(x_n-y_n)=\bar x-\bar y$
  \item $\l(x_n\cdot y_n)=\bar x\cdot\bar y$
  \item $\l(x_n/y_n)=\bar x/\bar y$, provided $y_n\neq0$ for all $n\in\N$, and
  $\bar y\neq0$.
\end{enumerati}

In short, the arithmetic operators $+$, $-$, $\times$, $\div$ are preserved
upon taking limits.

Note that these require both $\{x_n\}$ and $\{y_n\}$ to converge.

\Corollary{3.2.3}{}\label{c182ece}

If $\{x_n\}$ converges and $k\in\N$, then
$$
  \lim_{n\to\infty} (x_n)^k=\Big(\lim_{n\to\infty}x_n\Big)^k
$$

\Theorem{3.2.4}{Squeeze Theorem}\label{c3364d9}

If $x_n\leq y_n\leq z_n$ for all $n\in\N$ and
$\displaystyle\lim_{n\to\infty}x_n=\lim_{n\to\infty}z_n=a$, then
$$
  \lim_{n\to\infty}y_n=a
$$

Note that we can weaken the condition on $n$ to just $n\in\N,\ n\geq K_0$ for
some fixed $K_0$.

\begin{proof}
  Let $\epsilon>0$ be given. Since $x_n\to a$, there exists $K_1\in\N$ such that
  for all $n\geq K_1$,
  \begin{equation*}
    |x_n-a|<\epsilon \implies-\epsilon<x_n-a\Tag{*}
  \end{equation*}

  Since $z_n\to a$, there exists $K_1\in\N$ such that for all $n\geq K_1$,
  \begin{equation*}
    |z_n-a|<\epsilon \implies z_n-a<\epsilon\Tag{**}
  \end{equation*}

  Let $K:=\max\{K_1,K_2\}\in\N$. If we used the weaker condition, put
  $K:=\max\{K_0,K_1,K_2\}$.

  Then for all $n\geq K$, we have
  \begin{align*}
    x_n\leq y_n\leq z_n
     &\implies x_n-a\leq y_n-a\leq z_n-a                                   \\
     &\implies -\epsilon<y_n-a<\epsilon\with\desc{(from $(*)$ and $(**)$)} \\
     &\implies |y_n-a|<\epsilon
  \end{align*}

  Hence we also have $\displaystyle\lim_{n\to\infty}y_n=a$.
\end{proof}

\Theorem{3.2.5}{}\label{a9a7a2f}

If $|x_n|\to0$, then $x_n\to0$.

\begin{proof}
  Let $\epsilon>0$ be given. Since $|x_n|\to0$, it follows that there exists
  $K\in\N$ such that
  $$
    n\geq K\implies \big||x_n|-0\big|<\epsilon
  $$

  But $\big||x_n|-0\big|=|x_n-0|$, and hence we have
  $$
    n\geq K\implies |x_n-0|=\big||x_n|-0\big|<\epsilon
  $$

  Hence $x_n\to0$.
\end{proof}

\section{Real Analysis}\label{cd94400}

\Definition{1.1.1}{Number systems}\label{d52c6b7}

\begin{enumerati}
  \item $\N:=$ set of all natural numbers $\{1,2,3,\ldots\}$
  \item $\Z:=$ set of all integers $\{\ldots,-2,-1,0,1,2,\ldots\}$
  \item $\mathbb Q:=$ set of all rational numbers $\Set{\dfrac pq}{p,q\in\Z,q\neq0}$
  \item $\mathbb R:=$ set of all real numbers
\end{enumerati}

We have
$$
  \N\subseteq\Z\subseteq\mathbb Q\subseteq\R.
$$

The set of irrational numbers is denoted by $\R\setminus\mathbb Q$.

\Theorem{1.1.2}{$\sqrt2$ is an irrational number}\label{c2585a1}

\begin{proof}
  Suppose $\sqrt2$ is rational. Then we can write
  $$\sqrt2=\frac ab$$

  where $a$ and $b$ are integers with no common factor other than 1. Then
  $$2=\frac{a^2}{b^2}$$

  and
  $$2b^2=a^2$$

  This says that $a^2$ is even. So $a$ is also even, and $a=2k$ for some integer
  $k$. Then we get
  $$2b^2=4k^2$$

  So
  $$b^2=2k^2$$

  But this says that $b^2$ is even, so $b$ is even. It follows that 2 is a common
  factor for $a$ and $b$. This contradicts our assumption of $a$ and $b$, and
  hence $\sqrt2$ is not rational.
\end{proof}

\Principle{1.2.1}{Well-ordering Property of $\N$}\label{cd7c4d1}

Every non-empty subset $S$ of $\N$ has a \textit{least (or minimum)} element.
Formally,
$$
  \exists m\in S:\ \forall s\in S,\ m\leq s
$$

Note that $S$ may not have a largest element.

\Theorem{1.2.2}{Induction on natural numbers}\label{a824f8c}

Let $S\subseteq\N$. If we have
\begin{enumerati}
  \item $1\in S$, and
  \item for every $k\in\N$, $k\in S\implies k+1\in S$.
\end{enumerati}

Then $S=\N$.

\begin{proof}
  Suppose that $S\neq\N$. Then its complement $\N\setminus S\neq\emptyset$

  By the \href{cd7c4d1}{well-ordering property of $\N$}, there exists a least
  element $m\in\N\setminus S$.

  By (i), we have $m\neq 1$ and hence $m\geq2$. Thus, $m-1\in\N$. Since $m$ is
  the smallest natural number \textit{not} in $S$, we have $m-1\in S$. But by
  (ii), $m=(m-1)+1\in S$, which is a contradition to $m\in\N\setminus S$.
\end{proof}

\Theorem{1.2.3}{Principle of Mathematical Induction}\label{b51ca45}

For each $n\in\N$, let $P(n)$ be a statement about $n$. Suppose that
\begin{enumerati}
  \item $P(1)$ is true, and
  \item for every $k\in\N$, if $P(k)$ is true, then $P(k+1)$ is true.
\end{enumerati}

Observe that $1$ can be replaced with any natural number $n_0$, but we would
have only proved that $P$ is true for all natural numbers $\geq n_0$.

\begin{proof}
  Apply \href{a824f8c}{induction on natural numbers} on the set
  $$
    \set{n\in\N}{P(n)\text{ is true}}
  $$
\end{proof}

\Remark{1.3.1}{Algebraic Properties of $\R$}\label{bf61f02}

The binary operation \textbf{addition} on the set $\R$ satisfies the following
properties, for all $a,b,c\in\R$:

\begin{itemize}
  \item [(\textbf{A1})] \textit{(Commutativity)} $a+b=b+a$
  \item [(\textbf{A2})] \textit{(Associativity)} $(a+b)+c=a+(b+c)$
  \item [(\textbf{A3})] \textit{(Existence of additive identity)} $\exists0\in\R:\ a+0=0+a=a$
  \item [(\textbf{A4})] \textit{(Existence of additive inverse)} $\forall
        x\in\R,\ \exists {-}x\in\R:\ x+({-}x)=({-}x)+x=0$
\end{itemize}

The binary operation \textbf{multiplication} on $\R$ satisfies the following
properties, for all $a,b,c\in\R$:

\begin{itemize}
  \item [(\textbf{M1})] \textit{(Commutativity)} $a\cdot b=b\cdot a$
  \item [(\textbf{M2})] \textit{(Associativity)} $(a\cdot b)\cdot c=a\cdot
        (b\cdot c)$
  \item [(\textbf{M3})] \textit{(Existence of multiplicative identity)} $\exists1\in\R:\ a\cdot 1=1\cdot a=a$
  \item [(\textbf{M4})] \textit{(Existence of multiplicative inverse)} $\forall
        x\in\R\sans0,\ \exists 1/x\in\R:\ x\cdot(1/x)=(1/x)\cdot x=1$
\end{itemize}

In addition, the two binary operations satisfy the following property:
\begin{itemize}
  \item [(\textbf{D})] \textit{(Distributivity of multiplication over addition)}
        $$
          a\cdot(b+c)=a\cdot b+a\cdot c\with\forall a,b,c\in\R
        $$
\end{itemize}

Because of (\textbf{A1})-(\textbf{A4}), (\textbf{M1})-(\textbf{M4}), and
(\textbf{D}), we say that $(\R,+,\cdot)$ forms a \textbf{field}.

\Theorem{1.3.2}{Results from the algebraic properties of $\mathbb R$}\label{a1bdcab}

For any $a,b,c\in\R$,

\begin{enumerati}
  \item \textit{(Uniqueness of additive inverse)} If $a+b=0$, then $b={-}a$
  \item \textit{(Uniqueness of multiplicative inverse)} If $a\cdot b=1$ and
  $a\neq0$, then $b=\frac1a$.
  \item If $a+b=b$, then $a=0$.
  \item If $b\neq0$ and $a\cdot b=b$, then $a=1$.
  \item $a\cdot0=0$.
  \item If $a\cdot b=0$, then $a=0$ or $b=0$.
  \item \textit{(Cancellative property)} If $a\neq0$ and $a\cdot b=a\cdot c$, then $b=c$.
\end{enumerati}

\begin{proof}
  \def\a{{-}a}\def\b{{-}b}
  Let $a,b,c\in\R$.

  Proof of (i): Suppose $a+b=0$. Then
  \begin{align*}
    a+b+(\a) &=0+(\a)                                              \\
    a-(\a)+b &=\a\with\desc{(commutativity and additive identity)} \\
    0+b      &=\a\with\desc{(additive inverse)}                    \\
    b        &=\a\with\desc{(additive identity)}
  \end{align*}

  Proof of (ii): Suppose $a\cdot b=1$ and $a\neq0$. Then
  \begin{align*}
    a\cdot b\cdot(1/a) &=1\cdot(1/a)                                                \\
    a\cdot(1/a)\cdot b &=1/a\with\desc{(commutativity and multiplicative identity)} \\
    1\cdot b           &=1/a\with\desc{(multiplicative inverse)}                    \\
    b                  &=1/a\with\desc{(multiplicative identity)}
  \end{align*}

  Proof of (iii): Suppose $a+b=b$. Then
  \begin{align*}
    a+b+(\b) &=b+(\b)                           \\
    a+0      &=0\with\desc{(additive inverse)}  \\
    a        &=0\with\desc{(additive identity)}
  \end{align*}

  Proof of (iv): Suppose $a\cdot b=b$ and $b\neq0$. Then
  \begin{align*}
    a\cdot b\cdot(1/b) &=b\cdot(1/b)                            \\
    a\cdot1            &=1\with\desc{(multiplicative inverse)}  \\
    a                  &=1\with\desc{(multiplicative identity)}
  \end{align*}

  Proof of (v):
  \begin{align*}
    a\cdot0 &= a\cdot0+0\with\desc{(additive identity)}                         \\
            &= a\cdot0+[(a\cdot0)+({-}(a\cdot0))]\with\desc{(additive inverse)} \\
            &= (a\cdot0+a\cdot0)+({-}(a\cdot0))\with\desc{(associativity)}      \\
            &= a\cdot(0+0)+({-}(a\cdot0))\with\desc{(distributivity)}           \\
            &= a\cdot0+({-}(a\cdot0))\with\desc{(additive identity)}            \\
            &= 0\with\desc{(additive inverse)}                                  \\
  \end{align*}

  Proof of (vi): Suppose $a\cdot b=0$. Now if $a=0$ then we are done. So suppose
  that $a\neq0$. Then $1/a$ exists.
  \begin{align*}
    a\cdot b\cdot(1/a) &=0\cdot(1/a)                            \\
    a\cdot b\cdot(1/a) &=0\with\desc{(result (v))}              \\
    a\cdot(1/a)\cdot b &=0\with\desc{(commutativity)}           \\
    1\cdot b           &=0\with\desc{(multiplicative inverse)}  \\
    b                  &=0\with\desc{(multiplicative identity)}
  \end{align*}

  Proof of (vii): Suppose that $a\neq0$ and $a\cdot b=a\cdot c$. By
  \href{bf61f02}{(\textbf{M4})} that $1/a$ exists. Then
  \begin{align*}
    (1/a)\cdot(a\cdot b)  &=(1/a)\cdot(a\cdot c)                             \\
    ((1/a)\cdot a)\cdot b &=((1/a)\cdot a)\cdot c\with\desc{(commutativity)} \\
    1\cdot b              &=1\cdot c\with\desc{(multiplicative inverse)}     \\
    b                     &= c\with\desc{(multiplicative identity)}
  \end{align*}
\end{proof}

\Theorem{1.6.2}{Triangle inequality for $\mathbb R$}\label{f1288ad}

For all $a,b\in\R$, we have
$$
  |a+b|\leq|a|+|b|
$$

\begin{proof}
  We have ${-}|a|\leq a\leq|a|$ and ${-}|b|\leq b\leq|b|$. Adding, we have
  \begin{align*}
    -(|a|+|b|)\leq a+b\leq |a|+|b|
  \end{align*}

  which implies that $|a+b|\leq|a|+|b|$.
\end{proof}

\Corollary{1.6.3}{Corollaries of triangle inequality for $\mathbb R$}\label{f699f4d}

\begin{enumerati}
  \item $\big||a|-|b|\big|\leq|a-b|$
  \item $|a-b|\leq|a|+|b|$
  \item $\big||a|-|b|\big|\leq|a+b|$
\end{enumerati}

\begin{proof}
  By the \href{f1288ad}{triangle inequality},
  $$
    |a|=|(a-b)+b|\leq|a-b|+|b|
  $$

  So
  \begin{equation*}
    |a|-|b|\leq|a-b|\Tag{*}
  \end{equation*}

  By symmetry, we also have
  $$
    |b|-|a|\leq|b-a|
  $$

  which can be rewritten as
  \begin{equation*}
    -(|a|-|b|)\leq|a-b|\Tag{**}
  \end{equation*}

  With $(*)$ and $(**)$, we have (i).

  (ii) is obtained by using ${-}b$ in the place of $b$ in the
  \href{f1288ad}{triangle inequality}

  (iii) follows from using ${-}b$ in the place of $b$ in (i).
\end{proof}

\Definition{2.1.1}{Boundedness}\label{e4698be}

A non-empty set $S\subseteq\R$ is said to be \textbf{bounded above} if there
exists some $M\in\R$ such that
$$
  x\leq M,\with\forall x\in S.
$$

Such an $M$ is called an \textbf{upper bound} of $S$.

On the other hand, $S$ is said to be \textbf{bounded below} if there exists
some $m\in\R$ such that
$$
  m\leq x,\with\forall x\in S.
$$

Such an $m$ is called a \textbf{lower bound} of $S$.

If $S$ is both bounded above and bounded below, then we simply call it
\textbf{bounded}.

\Definition{2.2.1}{Maximum and minimum of a subset of $\R$}\label{c3ec51c}

For a non-empty $S\subseteq\R$, one defines the maximum of $S$ to be the
(necessarily unique) number $M$ such that
\begin{enumerati}
  \item $M\in S$, and
  \item $x\leq M$ for all $x\in S$.
\end{enumerati}

Similarly, the \textbf{minimum} of $S$ is the (necessarily unique) number $m$
such that
\begin{enumerati}
  \item $m\in S$, and
  \item $m\leq x$ for all $x\in S$.
\end{enumerati}

\Definition{2.3.1}{Supremum}\label{e6981e1}

Let $E\subseteq\R$ be non-empty. A real number $M\in\R$ is called the
\textbf{supremum} of $E$ if
\begin{enumerati}
  \item $M$ is an \href{e4698be}{upper bound} of $E$, and
  \item if $M'$ is an upper bound of $E$, then $M\leq M'$.
\end{enumerati}

\Lemma{2.3.2}{}\label{f77f162}

Let $E\subseteq\R$ be non-empty. Then $M=\sup E$ if and only if $M$ is an upper
bound of $E$ and for every $\epsilon>0$, there exists $x_\epsilon\in E$ such
that $M-\epsilon<x_\epsilon$.

\begin{proof}
  ($\implies$) Suppose $M=\sup E$. Let $\epsilon>0$. Then $M-\epsilon<M$. Since
  $M$ is the least upper bound of $E$ by definition, $M-\epsilon$ cannot be an
  upper bound for $E$. Hence there exists $x_\epsilon\in E$ such that
  $M-\epsilon<x_\epsilon$.

  ($\impliedby$) Suppose $M$ is an upper bound for $E$ and that there exists
  $x_\epsilon\in E$ such that $M-\epsilon<x_\epsilon$. Let $M'$ be an upper
  bound of $E$. Suppose on the contrary that $M'<M$. Then we let
  $\epsilon:=M-M'>0$. Then there exists $x_\epsilon\in E$ such that
  $$
    M'=M-(M-M')=M-\epsilon<x_\epsilon
  $$

  This contradicts the assumption that $M'$ is an upper bound for $E$. Hence we
  must have $M\leq M'$, making $M$ the least upper bound of $E$.
\end{proof}

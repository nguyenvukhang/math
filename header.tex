\documentclass{article}

% for coloring links in hyperref
\usepackage{xcolor}
\definecolor{blue}{HTML}{0284C7}
\definecolor{green}{HTML}{10B981}
\definecolor{gray}{HTML}{CBD5E1}
\definecolor{lightgray}{HTML}{E2E8F0}
\definecolor{pink}{HTML}{DB2777}
\def\blue#1{\textcolor{blue}{#1}}
\def\green#1{\textcolor{green}{#1}}
\def\gray#1{\textcolor{gray}{#1}}
\def\lightgray#1{\textcolor{lightgray}{#1}}

% for linking to sections within the same document
% provides `\hyperref[<target>]{<display text>}`
\usepackage[colorlinks=true,linkcolor=pink]{hyperref}
\hypersetup{pageanchor=false}

% for adding super powers to lists {itemize,enumerate}
\usepackage{tex_packages/enumitem}

% requires [enumitem]
\setlist[itemize]{itemsep=1pt,topsep=0pt}

\usepackage{amsmath,amsfonts,amssymb}
\usepackage[b5paper,
	left=0.7in,right=0.7in,
	top=0.5in,bottom=0.5in,
	footskip=10.25pt
]{geometry}

\usepackage{fancyhdr}
\fancypagestyle{plain}{%
	\fancyhf{} % clear all header and footer fields
  \fancyfoot[R]{\footnotesize\lightgray[\gray{\thepage}\lightgray]}
	\renewcommand{\headrulewidth}{0pt}
	\renewcommand{\footrulewidth}{0pt}}
\pagestyle{plain}

% remove paragraph indent
\setlength{\parindent}{0pt}

% set line spacing
\setlength{\parskip}{0.5em}

% Typing macros. Keep these to an absolute minimum to maintain
% compatibility of the source.
\def\N{\mathbb{N}}
\def\Z{\mathbb{Z}}
\def\R{\mathbb{R}}
\def\C{\mathbb{C}}
\def\Re{\text{Re}}
\def\Quad{\quad\quad}
\def\quadd{\quad\quad\quad}
\def\Quadd{\quad\quad\quad\quad}
\def\with{\quad}
\def\norm#1{\left\lVert{#1}\right\rVert}
\def\QED{\hfill$\square$}
\def\iter#1#2{{#1},\ldots,{#2}}
\def\epsilon{\varepsilon}
\def\Set#1#2{\left%
\{{#1}\;\middle\vert\;{#2}\right\}}
\def\Tag#1{\tag*{($#1$)}}
\def\notimplies{\;\not\nobreak\!\!\!\!\implies}
\def\href#1#2{\hyperref[#1]{#2}}
\def\inner#1#2{\langle{#1},{#2}\rangle}

% brackets in increasing size:
% \big( \Big( \bigg( \Bigg(

% don't number sections at all
% `\section{}` behaves like `\section*{}` now
\setcounter{secnumdepth}{0}

\def\section#1{\subsubsection{\LARGE #1}}
\def\subsection#1{\subsubsection{\Large #1}}

% for enabling/disabling specific fields
\newenvironment{proof}{\paragraph{Proof.}} {\QED}
\newenvironment{compute}{\paragraph{Compute.}} {\QED}
\newenvironment{enumerata}{\begin{enumerate}[label=(\alph*)]}{\end{enumerate}}
\newenvironment{enumerati}{\begin{enumerate}[label=(\roman*)]}{\end{enumerate}}
\renewenvironment{align}{\begin{align*}}{\end{align*}}
\renewenvironment{gather}{\begin{gather*}}{\end{gather*}}
\renewenvironment{equation}{\begin{equation*}}{\end{equation*}}

% auto-generated from bin/generate-headers.py
\def\Algorithm#1#2{\subsubsection{Algorithm {#1}\quad{\normalfont\ifx&#2&\else(\fi#2\ifx&#2&\else)\fi}}}
\def\Corollary#1#2{\subsubsection{Corollary {#1}\quad{\normalfont\ifx&#2&\else(\fi#2\ifx&#2&\else)\fi}}}
\def\Definition#1#2{\subsubsection{Definition {#1}\quad{\normalfont\ifx&#2&\else(\fi#2\ifx&#2&\else)\fi}}}
\def\Example#1#2{\subsubsection{Example {#1}\quad{\normalfont\ifx&#2&\else(\fi#2\ifx&#2&\else)\fi}}}
\def\Exercise#1#2{\subsubsection{Exercise {#1}\quad{\normalfont\ifx&#2&\else(\fi#2\ifx&#2&\else)\fi}}}
\def\Lemma#1#2{\subsubsection{Lemma {#1}\quad{\normalfont\ifx&#2&\else(\fi#2\ifx&#2&\else)\fi}}}
\def\Problem#1#2{\subsubsection{Problem {#1}\quad{\normalfont\ifx&#2&\else(\fi#2\ifx&#2&\else)\fi}}}
\def\Proposition#1#2{\subsubsection{Proposition {#1}\quad{\normalfont\ifx&#2&\else(\fi#2\ifx&#2&\else)\fi}}}
\def\Remark#1#2{\subsubsection{Remark {#1}\quad{\normalfont\ifx&#2&\else(\fi#2\ifx&#2&\else)\fi}}}
\def\Result#1#2{\subsubsection{Result {#1}\quad{\normalfont\ifx&#2&\else(\fi#2\ifx&#2&\else)\fi}}}
\def\Theorem#1#2{\subsubsection{Theorem {#1}\quad{\normalfont\ifx&#2&\else(\fi#2\ifx&#2&\else)\fi}}}

% custom math operators
\DeclareMathOperator*{\argmin}{argmin}